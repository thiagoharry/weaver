\input tex/epsf.tex
\font\sixteen=cmbx16
\font\twelve=cmr12
\font\fonteautor=cmbx12
\font\fonteemail=cmtt10
\font\twelvenegit=cmbxti12
\font\twelvebold=cmbx12
\font\trezebold=cmbx13
\font\twelveit=cmsl12
\font\monodoze=cmtt12
\font\it=cmti12
\voffset=0,959994cm % 3,5cm de margem superior e 2,5cm inferior
\parskip=6pt

\def\titulo#1{{\noindent\sixteen\hbox to\hsize{\hfill#1\hfill}}}
\def\autor#1{{\noindent\fonteautor\hbox to\hsize{\hfill#1\hfill}}}
\def\email#1{{\noindent\fonteemail\hbox to\hsize{\hfill#1\hfill}}}
\def\negrito#1{{\twelvebold#1}}
\def\italico#1{{\twelveit#1}}
\def\monoespaco#1{{\monodoze#1}}
\def\iniciocodigo{\lineskip=0pt\parskip=0pt}
\def\fimcodigo{\twelve\parskip=0pt plus 1pt\lineskip=1pt}

\long\def\abstract#1{\parshape 10 0.8cm 13.4cm 0.8cm 13.4cm
0.8cm 13.4cm 0.8cm 13.4cm 0.8cm 13.4cm 0.8cm 13.4cm 0.8cm 13.4cm
0.8cm 13.4cm 0.8cm 13.4cm 0.8cm 13.4cm
\noindent{{\twelvenegit Abstract: }\twelveit #1}}

\def\resumo#1{\parshape  10 0.8cm 13.4cm 0.8cm 13.4cm
0.8cm 13.4cm 0.8cm 13.4cm 0.8cm 13.4cm 0.8cm 13.4cm 0.8cm 13.4cm
0.8cm 13.4cm 0.8cm 13.4cm 0.8cm 13.4cm
\noindent{{\twelvenegit Resumo: }\twelveit #1}}

\def\secao#1{\vskip12pt\noindent{\trezebold#1}\parshape 1 0cm 15cm}
\def\subsecao#1{\vskip12pt\noindent{\twelvebold#1}}
\def\referencia#1{\vskip6pt\parshape 5 0cm 15cm 0.5cm 14.5cm 0.5cm 14.5cm
0.5cm 14.5cm 0.5cm 14.5cm {\twelve\noindent#1}}

%@* .

\twelve
\vskip12pt
\titulo{The Weaver Program}
\vskip12pt
\autor{Thiago Leucz Astrizi}
\vskip6pt
\email{thiago@@bitbitbit.com.br}
\vskip6pt

\abstract{This article describes using literary programming the
  program Weaver. This program is a project manager for the Weaver
  Game Engine. If a user wants to create a new game with the Weaver
  Game Engine, they use this program to create the directory structure
  for a new game project. They also use this program to add new source
  files and shader files to a game project. And to update a project
  with a more recent Weaver version installed in the computer. The
  presenting code in C is cross-platform and should work under
  Windows, Linux, OpenBSD and possibly other Unix variants.}


\secao{1. Introduction}

A game engine is made by a set of libraries and functions that helps a
game creation offering common functionalities for this kind of
development. But besides the libraries and functions, there should
exist a manager responsible for creating some code which uses the
library in a correct way and executes the necessary initializations.


The Weaver Game Engine has very strict prerequisites about how the
directory with a game project should be organized. To follow these
requisites, this program is necessary. It initializesin a correct way
the directory structure in a new project. It adds new source files
with the correct code to ensure the code integration. And controlling
the project in this way, it also knows how to perform updates in the
libraries for more recent versions.

This program usage is by the command line. For example, if a user
types ``{\tt weaver pong}'', a new directory structure like in the
following image will be created.

\imagem{cweb/diagrams/project_dir.eps}

The following sections in this document are organized in the following
way. Section 2 is about this software license. Section 3 lists all the
variables that control its execution. Section 4 defines some macros
used in the program structure. Section 5 lists all the auxiliary
functions defined. Sections 6 is how the variables are
initialized. Section 7 is about the software use cases and how they
are implemented after we have all the variables with the correct
value.

\secao{2. Copyright and licensing}

The software license is the GNU General Public License version 3:

\espaco{5mm}\linha
\alinhaverbatim
Copyright (c) Thiago Leucz Astrizi 2015

This program is free software: you can redistribute it and/or
modify it under the terms of the GNU Affero General Public License as
published by the Free Software Foundation, either version 3 of
the License, or (at your option) any later version.

This program is distributed in the hope that it will be useful,
but WITHOUT ANY WARRANTY; without even the implied warranty of
MERCHANTABILITY or FITNESS FOR A PARTICULAR PURPOSE.  See the
GNU Affero General Public License for more details.

You should have received a copy of the GNU Affero General Public
License along with this program.  If not, see
<http://www.gnu.org/licenses/>.
\alinhanormal

\linha\espaco{5mm}

The complete version of the license can be obtained with the source
code or checking the link above.

\secao{3. Variables and software structure}

Weaver execution depends of the following variables:

|inside_weaver_directory|: If the program is invoked inside a Weaver
 project directory.

|argument|: The first argument, or NULL if doesn't exist.

|argument2|: The second argument, or NULL if doesn't exist.

|project_version_major|: If we are in a Weaver project, which is the
 major version number of the program which created the project? For
 example, if wea are in a project created by Weaver 0.5, the major
 version is 0. In tests version, the value is always 0.

|project_version_minor|: If we are in a Weaver project, the minor
 version number of the program which created the project. For example,
 if Weaver 0.5 created the current project, this number is 5. In test
 versions, the value is always 0.

|weaver_version_major|: The major version of this program.

|weaver_version_minor|: The minor version of this program.

|arg_is_path|: If the first argument exists and is an absolute or
 relative path in the filesystem.

|arg_is_valid_project|: If the first argument exists and would be
 considered a valid Weaver project name.

|arg_is_valid_module|: If the first argument exists and would be
 considered a valid module name in a Weaver project.

|arg_is_valid_plugin|: If the second argument exists and would be
 considered a valid plugin name in a Weaver project.

|arg_is_valid_function|: If the second argument exists and if it would
 be considered a valid name for a main loop and for a new file in a
 Weaver project.

|project_path|: If we are inside a Weaver project, which is the path
 for its base directory (where is the Makefile)?

|have_arg|: If the program is invoked with an argument.

|shared_dir|: The path to the directory where are the shared files
 from Weaver installation. The default is ``{\tt
 /usr/local/share/weaver}'' in Unix systems and the ``Program Files''
 folder in Windows. This can be changed in the program building
 defining the macro {\tt WEAVER\_DIR}.

|author_name|,|project_name| and |year|: The name of the user which is
 executing the program, the current project name (if we are inside a
 Weaver project directory) and the current year. This is important for
 copyright messages creation.

|return_value|: If the program is interrupted in this exact moment,
 what the program should return?

The software general structure with all the variables declarations is:


\iniciocodigo
@(src/weaver.c@>=
@<Headers Included in Weaver Program@>
@<Weaver Program Macros@>
@<Weaver Auxiliary Functions@>
int main(int argc, char **argv){@/
  int return_value = 0; /* Return value. */
  bool inside_weaver_directory = false, arg_is_path = false,
    arg_is_valid_project = false, arg_is_valid_module = false,
    have_arg = false, arg_is_valid_plugin = false,
    arg_is_valid_function = false; /* Boolean variables. */
  unsigned int project_version_major = 0, project_version_minor = 0,
    weaver_version_major = 0, weaver_version_minor = 0,
    year = 0;
  /* Strings UTF-8: */
  char *argument = NULL, *project_path = NULL, *shared_dir = NULL,
    *author_name = NULL, *project_name = NULL, *argument2 = NULL;
  @<Initialization@>
  @<Use Case 1: Printing help (create project)@>
  @<Use Case 2: Printing management help@>
  @<Use Case 3: Print version@>
  @<Use Case 4: Updating Weaver project@>
  @<Use Case 5: Create new module@>
  @<Use Case 6: Create new project@>
  @<Use Case 7: Create new plugin@>
  @<Use Case 8: Create new shader@>
  @<Use Case 9: Create new main loop@>
END_OF_PROGRAM:
  @<Finishing@>
  return return_value;
}
@
\fimcodigo

\secao{4. Macros and Headers in Weaver Program}

This program needs some macros. The first one shall store a string
with the program version; This version could be formed just by letters
(if a test version) or by digits followed by a dot and more digits
(without whitespaces) if this is a final version of the program.

For the second macro, observe that in the program structure above,
exists a label called |END_OF_PROGRAM| in the finishing part. We can
reach the label following the program normal execution, if nothing
wrong happens. Otherwise, if an error happens, we can reach that label
by an unconditional jump after printing the error message and
adjusting the program return value. Treating this error condition with
these actions is the second macro responsability.

We also could finish the program prematurely, but not because some
error happened. The third macro will treat this case:

\iniciocodigo
@<Weaver Program Macros@>=
#define VERSION "Alpha"
#define W_ERROR() {perror(NULL); return_value = 1; goto END_OF_PROGRAM;}
#define END() goto END_OF_PROGRAM;
@
\fimcodigo

We are using the library function \monoespaco{perror}, so we need to
include the header \monoespaco{stdio.h}, which will also bring us
other useful functions to print in the screen or in files and to open
and close files. We also should insert support for boolean values and
the standard library with functions like |exit| utilized in the
program structure.

If we are running in Windows, we want so supress some warnings about
the use of functions \monoespaco{fopen} no lugar
de \monoespaco{fopen\_s}, por exemplo. Because the safe functions in
Windows are not portable and we use the old functions in a controlled
way. To supress the warning we need to define the specific macro below
before including other headers.

\iniciocodigo
@<Headers Included in Weaver Program@>=
#if defined(_WIN32)
#define _CRT_SECURE_NO_WARNING
#endif
#include <stdio.h> // printf, fprintf, fopen, fclose, fgets, fgetc, perror
#include <stdbool.h> // bool, true, false
#include <stdlib.h> // free, exit, getenv
@
\fimcodigo

\secao{5. Auxiliary Functions}

Here we list some functions which we should use in the program to
facilitate its description.

\subsecao{5.1. path\_up: Manipulating Paths}

To manipulate directory tree paths, we define an auxiliary function
which receives a path and erases the last characters until two ``/''
are erased. So in ``/home/alice/project/dir/'', it returns
``/home/alice/project'', goind one level up in the directory tree.

But in Windows systems, the separator isn't ``/'', but ``\\''. So we
should treat the separator differently according with the Operating
System:

\iniciocodigo
@<Weaver Auxiliary Functions@>=
void path_up(char *path){
#if !defined(_WIN32)
  char separator = '/';
#else
  char separator = '\\';
#endif
  int erased = 0;
  char *p = path;
  while(*p != '\0') p ++; // Vai até o fim
  while(erased < 2 && p != path){
    p --;
    if(*p == separator) erased ++;
    *p = '\0'; // Apaga
  }
}
@
\fimcodigo

Notice that if the function get a string without two separators, we
erase all the string. In this program we will limit this function
usage to strings with path for files outside the root directory, which
are not the root directory themselves and for directories ended by the
separator character. So we should always respect the limit of minimal
two separators in paths. Example: ``/etc/'' and ``/tmp/file.txt''.

\subsecao{5.2. directory\_exists: Arquivo existe e é diretório}

To check if the directory \monoespaco{.weaver} exists, we define
|directory_exist(x)| as a function which gets a file path and returns
1 if |x| is an existing directory, -1 if |x| is an existing file and 0
otherwise. Fist let's create macros to make explicit the meaning of
return values:

\iniciocodigo
@<Weaver Program Macros@>+=
#define DONT_EXIST         0
#define EXISTS_AND_IS_DIR   1
#define EXISTS_AND_IS_FILE -1
@
\fimcodigo

\iniciocodigo
@<Weaver Auxiliary Functions@>+=
int directory_exist(char *dir){
#if !defined(_WIN32)
  // Unix:
  struct stat s; // Stores if the file exists
  int err; // Checagem de erros
  err = stat(dir, &s); // It exists?
  if(err == -1) return DONT_EXIST;
  if(S_ISDIR(s.st_mode)) return EXISTS_AND_IS_DIR;
  return EXISTS_AND_IS_FILE;
#else
  // Windows:
  DWORD dwAttrib = GetFileAttributes(dir);
  if(dwAttrib == INVALID_FILE_ATTRIBUTES) return DONT_EXIST;
  if(!(dwAttrib & FILE_ATTRIBUTE_DIRECTORY)) return EXISTS_AND_IS_FILE;
  else return EXISTS_AND_IS_DIR;
#endif
}
@
\fimcodigo

Depending of the Operating System, we should utilize different
functions and need different headers:

\iniciocodigo
@<Headers Included in Weaver Program@>=
#if !defined(_WIN32)
#include <sys/types.h> // stat, getuid, getpwuid, mkdir
#include <sys/stat.h> // stat, mkdir
#else
#include <windows.h> // GetFileAttributes, ...
#endif
@
\fimcodigo

\subsecao{5.3. concatenate: Concatenate strings}

This function gets an arbitrary number of strings, but the last string
must be |NULL| or the empty string. And it returns the concatenation
of all the strings passed as argument. The function will always
allocate a new string, which should be freed before the program
ending.

Example: |concatenate("tes", " ", "t", "")| returns |"tes t"|.

\iniciocodigo
@<Weaver Auxiliary Functions@>+=
char *concatenate(char *string, ...){
  va_list arguments;
  char *new_string, *current_string = string;
  size_t current_size = strlen(string) + 1;
  char *realloc_return;
  va_start(arguments, string);
  new_string = (char *) malloc(current_size);
  if(new_string == NULL) return NULL;
  // Copying first string:
  memcpy(new_string, string, current_size);
  while(current_string != NULL && current_string[0] != '\0'){
    size_t increment_length, last_length;
    current_string = va_arg(arguments, char *);
    increment_length = strlen(current_string);
    last_length = current_size;
    current_size += increment_length;
    realloc_return = (char *) realloc(new_string, current_size);
    if(realloc_return == NULL){
      free(new_string);
      return NULL;
    }
    new_string = realloc_return;
    // Copying next string:
    memcpy(&(new_string[last_length-1]), current_string, increment_length + 1);
  }
  return new_string;
}
@
\fimcodigo

This is a dangerous function that always should be invoked passing as
last argument an empty string or NULL.

This function usage requires the following headers:

\iniciocodigo
@<Headers Included in Weaver Program@>=
#include <string.h> // strcmp, strcat, strcpy, strncmp
#include <stdarg.h> // va_start, va_arg
@
\fimcodigo

\subsecao{5.4. basename: Get a file name given its path}

This function already exists in Unix systems. Given a complete path
for a file, it returns a string with the file name. It doesn't neet to
allocate a new string, it can just return a pointer for the filename
inside the path string. Let's define it for Windows and other systems
without a |basename| function:

\iniciocodigo
@<Weaver Auxiliary Functions@>+=
#if defined(_WIN32)
char *basename(char *path){
  char *p = path;
  char *last_delimiter = NULL;
  while(*p != '\0'){
    if(*p == '\\')
      last_delimiter = p;
    p ++;
  }
  if(last_delimiter != NULL)
    return last_delimiter + 1;
  else
    return path;
}
#endif
@
\fimcodigo


In Unix Systems, we don't need to define this function, we just
include its header:

\iniciocodigo
@<Headers Included in Weaver Program@>=
#if !defined(_WIN32)
#include <libgen.h>
#endif
@
\fimcodigo

\subsecao{5.5. copy\_single\_file: Copy single file to target directory}

The function |copy_single_file| copies the file which path is the
first argument to the target directory which path is the second
argument. It returns 1 if successful or 0 otherwise.

\iniciocodigo
@<Weaver Auxiliary Functions@>+=
int copy_single_file(char *file, char *directory){
  int block_size, bytes_read;
  char *buffer, *file_dst;
  FILE *orig, *dst;
  // Inicializa 'block_size':
  @<Discover block size@>
  buffer = (char *) malloc(block_size); // Allocating buffer for copy
  if(buffer == NULL) return 0;
  file_dst = concatenate(directory, "/", basename(file), "");
  if(file_dst == NULL) return 0;
  orig = fopen(file, "r"); // Open origin file
  if(orig == NULL){
    free(buffer);
    free(file_dst);
    return 0;
  }
  dst = fopen(file_dst, "w"); // Open destiny file
  if(dst == NULL){
    fclose(orig);
    free(buffer);
    free(file_dst);
    return 0;
  }
  while((bytes_read = fread(buffer, 1, block_size, orig)) > 0){
    fwrite(buffer, 1, bytes_read, dst); // Copy origin to destiny
  }
  fclose(orig);
  fclose(dst);
  free(file_dst);
  free(buffer);
  return 1;
}
@
\fimcodigo

It's more efficient when the buffer used in the copy has the same size
than a block in the filesystem. To get the correct valuem we use this
code in Unix systems:

\iniciocodigo
@<Discover block size@>=
#if !defined(_WIN32)
{
  struct stat s;
  stat(directory, &s);
  block_size = s.st_blksize;
  if(block_size <= 0){
    block_size = 4096;
  }
}
#endif
@
\fimcodigo

In Windows we just assume that the size is 4KB:

\iniciocodigo
@<Discover block size@>+=
#if defined(_WIN32)
  block_size = 4096;
#endif
@
\fimcodigo

\subsecao{5.6. copy\_files: Copy all source files to destiny}

With a function to copy a single file, we need to define a function to
copy all the files inside a directory recursivelly. This requises some
work, as we need to list all the content in a directory to get its
files. How to do this depends of the Operating System.

In Unix systems we use the function |readdir| to read the content of
directories:

\iniciocodigo
@<Weaver Auxiliary Functions@>+=
#if !defined(_WIN32)
int copy_files(char *orig, char *dst){
  DIR *d = NULL;
  struct dirent *dir;
  d = opendir(orig);
  if(d){
    while((dir = readdir(d)) != NULL){ // Loop to read each file
      char *file;
      file = concatenate(orig, "/", dir -> d_name, "");
      if(file == NULL){
        return 0;
      }
#if (defined(__linux__) || defined(_BSD_SOURCE)) && defined(DT_DIR)
      // If we support DT_DIR, we don't need the funcion 'stat':
      if(dir -> d_type == DT_DIR){
#else
      struct stat s;
      int err;
      err = stat(file, &s);
      if(err == -1) return 0;
      if(S_ISDIR(s.st_mode)){
#endif
      // If we are dealing with a subdirectory:
        char *new_dst;
        new_dst = concatenate(dst, "/", dir -> d_name, "");
        if(new_dst == NULL){
          return 0;
        }
        if(strcmp(dir -> d_name, ".") && strcmp(dir -> d_name, "..")){
          if(directory_exist(new_dst) == DONT_EXIST) mkdir(new_dst, 0755);
          if(copy_files(file, new_dst) == 0){
            free(new_dst);
            free(file);
            closedir(d);
            return 0;
          }
        }
        free(new_dst);
      }
      else{
        // If we get a regular file:
        if(copy_single_file(file, dst) == 0){
          free(file);
          closedir(d);
          return 0;
        }
      }
    free(file);
    } // End of loop to read each file
    closedir(d);
  }
  return 1;
}
#endif
@
\fimcodigo

And this requires the following headers:

\iniciocodigo
@<Headers Included in Weaver Program@>=
#if !defined(_WIN32)
#include <dirent.h> // readdir, opendir, closedir
#endif
@
\fimcodigo

In Windows we don't need new headers. The function definition becames
the following:

\iniciocodigo
@<Weaver Auxiliary Functions@>+=
#if defined(_WIN32)
int copy_files(char *orig, char *dst){
  char *path, *search_path;
  WIN32_FIND_DATA file;
  HANDLE dir = NULL;
  search_path = concatenate(orig, "\\*", "");
  if(search_path == NULL)
    return 0;
  dir = FindFirstFile(search_path, &file);
  if(dir != INVALID_HANDLE_VALUE){
    // The first file shall be '.' and should be safely ignored
    do{
      if(strcmp(file.cFileName, ".") && strcmp(file.cFileName, "..")){
        path = concatenate(orig, "\\", file.cFileName, "");
        if(path == NULL){
          free(search_path);
          return 0;
        }
        if(file.dwFileAttributes & FILE_ATTRIBUTE_DIRECTORY){
          char *dst_path;
          dst_path = concatenate(dst, "\\", file.cFileName, "");
          if(directory_exist(dst_path) == DONT_EXIST)
            CreateDirectoryA(dst_path, NULL);
          if(copy_files(path, dst_path) == 0){
            free(dst_path);
            free(path);
            FindClose(dir);
            free(search_path);
            return 0;
          }
          free(dst_path);
        }
        else{ // file
          if(copy_single_file(path, dst) == 0){
            free(path);
            FindClose(dir);
            free(search_path);
            return 0;
          }
        }
        free(path);
      }
    }while(FindNextFile(dir, &file));
  }
  free(search_path);
  FindClose(dir);
  return 1;
}
#endif
@
\fimcodigo

\subsecao{5.7. write\_copyright: Write copyright messages in files}

By default Weaver projects are licensed under GNU GPL 3. As codes
under this license are copied and utilized statically in new projects,
the new projects needs the same license or a compatible one.

The code is very simple and requires just some parameters as the
author name and the current year:

\iniciocodigo
@<Weaver Auxiliary Functions@>+=
void write_copyright(FILE *fp, char *author_name, char *project_name, int year){
  char license[] = "/*\nCopyright (c) %s, %d\n\nThis file is part of %s.\n\n%s\
 is free software: you can redistribute it and/or modify\nit under the terms of\
 the GNU Affero General Public License as published by\nthe Free Software\ 
 Foundation, either version 3 of the License, or\n(at your option) any later\
 version.\n\n\
%s is distributed in the hope that it will be useful,\nbut WITHOUT ANY\
  WARRANTY; without even the implied warranty of\nMERCHANTABILITY or FITNESS\
  FOR A PARTICULAR PURPOSE.  See the\nGNU Affero General Public License for more\
  details.\n\nYou should have received a copy of the GNU Affero General Public License\
\nalong with %s. If not, see <http://www.gnu.org/licenses/>.\n*/\n\n";
  fprintf(fp, license, author_name, year, project_name, project_name,
          project_name, project_name);
}
@
\fimcodigo

\subsecao{5.8. create\_dir: Create new directories}

This function is responsible for creating a list of directories. This
is a very simple thing, but should be encapsulated in a function
because of differences between Operating Systems about how to do this
task.

This function must receive as argument a variable number of strings,
but the last argument must be an empty string or NULL. Earch argument
except the last represents a path. The function will create the
directory with the specified path. By default we use
``\monoespaco{/}'' as separator, so the function shall work both in
Unix systems as in Windows. In the later, the
function \monoespaco{CreateDirectoryA} accepts paths represented in
Unix notation.

In Unix systems we need to specify the maximum permissions in the
directory in terms os reading, writing and execution. The Operating
System can then accept our recommended permissions, or ensure more
restrictive ones depending of configuration. In Windows the permission
logic is more hyerarquical, so we just use the same permissions as the
parent directory.

In case of error, we return -1. Otherwise, we return 1.

The function definition is:

\iniciocodigo
@<Weaver Auxiliary Functions@>+=
int create_dir(char *string, ...){
  char *current_string;
  va_list arguments;
  va_start(arguments, string);
  int err = 1;
  current_string = string;
  while(current_string != NULL && current_string[0] != '\0' && err != -1){
#if !defined(_WIN32)
    err = mkdir(current_string, S_IRWXU | S_IRWXG | S_IROTH);
#else
    if(!CreateDirectoryA(current_string, NULL))
      err = -1;
#endif
    current_string = va_arg(arguments, char *);
  }
  return err;
}
@
\fimcodigo

\subsecao{5.9. append\_file: Concatenate file contents}

This is an unusual function, it was designed to solve efficiently a
single use case, sacrificing the consistency of its interface and its
ease of use. It gets as argument a pointer for a target file already
opened (usually we want to use this function after we opened a file to
write the copyright notice), as second argument it gets the path of
parent directory of an origin file and as the third argument it gets
the origin file name.

Its definition is:

\iniciocodigo
@<Weaver Auxiliary Functions@>+=
int append_file(FILE *fp, char *dir, char *file){
  int block_size, bytes_read;
  char *buffer, *directory = ".";
  char *path = concatenate(dir, file, "");
  if(path == NULL) return 0;
  FILE *origin;
  @<Discover block size@>
  buffer = (char *) malloc(block_size);
  if(buffer == NULL){
    free(path);
    return 0;
  }
  origin = fopen(path, "r");
  if(origin == NULL){
    free(buffer);
    free(path);
    return 0;
  }
  while((bytes_read = fread(buffer, 1, block_size, origin)) > 0){
    fwrite(buffer, 1, bytes_read, fp);
  }
  fclose(origin);
  free(buffer);
  free(path);
  return 1;
}
@
\fimcodigo

\secao{6. Variable Initialization}

\subsecao{6.1. inside\_weaver\_directory e project\_path: Where we are}

The first variable is |inside_weaver_directory|, which stores |false|
if the program was invoked outside a Weaver project directory and
|true| otherwise.

How should we detect if we are in a Weaver project directory? It's
simple. They are directories which contains in them or in an ancestor
directory a hidden directory named \monoespaco{.weaver}. If we find
this directory, we can also adjust the variable |project_path| to
point to where is this directory. If we don't find it, we are outside
a Weaver directory and we don't need to change these variables default
value, which are |false| and |NULL|.

In short, we need a loop with the following characteristics:

\negrito{Invariant}: The variable |complete_path| must always store
the complete path of the directory \monoespaco{.weaver} if this
file hypothetically existed in the current directory.

\negrito{Initialization:} We initialize |complete_path| to be valid
when we are in or initial current directory.

\negrito{Maintenance:} In each iteration we check if we found a
termination condition. If not, we change to the current directory
parent, always updating the variables to keep valid the invariant.

\negrito{Termination}: We terminate the loop if one of the following
3 conditions occur:

a) |complete_path == "/.weaver"|: We can't go to a parent directory
because we are already in the root of the filesystem. It means that we
aren't in a Weaver directory.

b) |complete_path == "C:\\.weaver"|: In fact, the initial letter could
be ``D'', ``E'' or any other letter, not just ``C''. It also could be
``\\.weaver''. This means that we are in the root of a Windows
filesystem (the last case without a drive letter represents a network
directory) and we aren't in a Weaver directory.

c) |complete_path == "./.weaver"| and this file exists and is a
directory: In this case, we were inside a Weaver directory. We can
also update |project_path| to store the current path.

The initialization of these variables is then:

\iniciocodigo
@<Initialization@>=
char *path = NULL, *complete_path = NULL;
#if !defined(_WIN32)
path = getcwd(NULL, 0); // Unix
#else
{ // Windows
  DWORD bsize;
  bsize = GetCurrentDirectory(0, NULL);
  path = (char *) malloc(bsize);
  GetCurrentDirectory(bsize, path);
}
#endif
if(path == NULL) W_ERROR();
complete_path = concatenate(path, "/.weaver", "");
free(path);
if(complete_path == NULL) W_ERROR();
@
\fimcodigo

To get the current directory, we need the header:

\iniciocodigo
@<Headers Included in Weaver Program@>=
#if !defined(_WIN32)
#include <unistd.h> // get_current_dir_name, getcwd, stat, chdir, getuid
#endif
@
\fimcodigo

Now we define the described loop:

\iniciocodigo
@<Initialization@>+=
{
  // Testa se chegamos ao fim:
  while(strcmp(complete_path, "/.weaver") &&
        strcmp(complete_path, "\\.weaver") &&
        strcmp(complete_path + 1, ":\\.weaver")){
    if(directory_exist(complete_path) == EXISTS_AND_IS_DIR){
      inside_weaver_directory = true;
      complete_path[strlen(complete_path) - 7] = '\0'; // Apaga o '.weaver'
      project_path = concatenate(complete_path, "");
      if(project_path == NULL){ free(complete_path); W_ERROR(); }
      break;
    }
    else{
      path_up(complete_path);
#ifdef __OpenBSD__
      {
        size_t tmp_size = strlen(complete_path);
        strlcat(complete_path, "/.weaver", tmp_size + 9);
      }
#else
      strcat(complete_path, "/.weaver");
#endif
    }
  }
  free(complete_path);
}
@
\fimcodigo

We allocated memory to |project_path|, so in the end of program we
need to free this memory:

\iniciocodigo
@<Finishing@>=
if(project_path != NULL) free(project_path);
@
\fimcodigo

\subsecao{6.2. weaver\_version\_major e weaver\_version\_minor:
Program Version}

To discover the current program version, we can just check the macro
|VERSION|. Then we get the majorand minor version number parsing the
digits separated by a dot (if they exist). If we can't find a dot in
the version name, this is a test version and the minor and major
version number must be treated as 0. So we can just use |atoi|
function and this requirement will be fullfilled:

\iniciocodigo
@<Initialization@>+=
{
  char *p = VERSION;
  while(*p != '.' && *p != '\0') p ++;
  if(*p == '.') p ++;
  weaver_version_major = atoi(VERSION);
  weaver_version_minor = atoi(p);
}
@
\fimcodigo

\subsecao{6.3. project\_version\_major e project\_version\_minor:
Project Version}

If we are inside a Weaver project, we need to initialize these
variables with Weaver major and minor version used to create the
project, or update the project if it was updated in the past. This can
be obtained checking the file \italico{.weaver/version} inside the
Weaver directory. If we aren't in a Weaver directory, we don't need to
get these values. The major and minor version usually is separated by
a dot, following the same rules than in previous subsection.

\iniciocodigo
@<Initialization@>+=
if(inside_weaver_directory){
  FILE *fp;
  char *p, version[10];
  char *file_path = concatenate(project_path, ".weaver/version", "");
  if(file_path == NULL) W_ERROR();
  fp = fopen(file_path, "r");
  free(file_path);
  if(fp == NULL) W_ERROR();
  p = fgets(version, 10, fp);
  if(p == NULL){ fclose(fp); W_ERROR(); }
  while(*p != '.' && *p != '\0') p ++;
  if(*p == '.') p ++;
  project_version_major = atoi(version);
  project_version_minor = atoi(p);
  fclose(fp);
}
@
\fimcodigo

\subsecao{6.4. have\_arg, argument e argument2: Invocation Arguments}

The easiest to initialize variables. We just check for |argc| and
|argv|.

\iniciocodigo
@<Initialization@>+=
have_arg = (argc > 1);
if(have_arg) argument = argv[1];
if(argc > 2) argument2 = argv[2];
@
\fimcodigo

\subsecao{6.5. arg\_is\_path: If the argument is a directory}

If we have a first argument, we need to check if it's a path for a
directory where is a Weaver project. For this, we just
concatenate \monoespaco{/.weaver} in the first argument and check if
this file exist.

\iniciocodigo
@<Initialization@>+=
if(have_arg){
  char *buffer = concatenate(argument, "/.weaver", "");
  if(buffer == NULL) W_ERROR();
  if(directory_exist(buffer) == EXISTS_AND_IS_DIR){
    arg_is_path = 1;
  }
  free(buffer);
}
@
\fimcodigo

\subsecao{6.6. shared\_dir: Where the files are installed}

The variable |shared_dir| shall contain where are the installed shared
files. These files are libraries to be inserted statically and models
of source code. If the macro \monoespaco{WEAVER\_DIR} exists because
it was defined during compilation, this will be the path where are
these files. Other wise, se use \monoespaco{/usr/local/share/weaver}
in Unix systems and \monoespaco{\%PROGRAMFILES\%\\weaver} in Windows
systems.

@<Initialization@>+=
{
#ifdef WEAVER_DIR
  shared_dir = concatenate(WEAVER_DIR, "");
#else
#if !defined(_WIN32)
  shared_dir = concatenate("/usr/local/share/weaver/", ""); // Unix
#else
  { // Windows
    char *temp_buf = NULL;
    DWORD bsize = GetEnvironmentVariable("ProgramFiles", temp_buf, 0);
    temp_buf = (char *) malloc(bsize);
    GetEnvironmentVariable("ProgramFiles", temp_buf, bsize);
    shared_dir = concatenate(temp_buf, "\\weaver\\", "");
    free(temp_buf);
  }
#endif
#endif
  if(shared_dir == NULL) W_ERROR();
}
@
\fimcodigo

With this code we allow the user to choose the install directory
during compilation. This usually is more common in Unix systems than
in Windows where programs are expected to be kept in the same place.

In Windows the code is longer because we need to determine manually
where to store the files. The path can change depending of system
language, drive unit, or if the program is 32 or 64 bits.

In both cases, during program finalization, we need to free the memory
allocated to |shared_dir|:

\iniciocodigo
@<Finishing@>+=
if(shared_dir != NULL) free(shared_dir);
@
\fimcodigo

\subsecao{6.7. arg\_is\_valid\_project:
If the argument is a project name}

The next question is if what we got as argument, if we got something,
could be a valid Weaver project name or not. To answer this, three
conditions need to be fullfilled:

1) The project basename must be formed only by alphanumeric characters
and underline (a ``/'' or ``\\'' can appear in the path, not in the
basename).

2) A file with the same name can't already exist in the place
indicated for the new project.

3) The project can't have the same name as some file that exists in
the base directory of a Weaver project (``Makefile'', for
example). Otherwise, during compilation we would overwrite important
files.

For this, we use the following code:

\iniciocodigo
@<Initialization@>+=
if(have_arg && !arg_is_path){
  char *buffer;
  char *base = basename(argument);
  int size = strlen(base);
  int i;
  // Checking for invalid characters
  for(i = 0; i < size; i ++){
    if(!isalnum(base[i]) && base[i] != '_'){
      goto NOT_VALID;
    }
  }
  // Checking if file exists:
  if(directory_exist(argument) != DONT_EXIST){
    goto NOT_VALID;
  }
  // Checking for name conflicts:
  buffer = concatenate(shared_dir, "project/", base, "");
  if(buffer == NULL) W_ERROR();
  if(directory_exist(buffer) != DONT_EXIST){
    free(buffer);
    goto NOT_VALID;
  }
  free(buffer);
  arg_is_valid_project = true;
}
NOT_VALID:
@
\fimcodigo

To check for alphanumeric characters, we include the following header:

\iniciocodigo
@<Headers Included in Weaver Program@>=
#include <ctype.h> // isalnum
@
\fimcodigo

\subsecao{6.8. arg\_is\_valid\_module: If the argument could be a module name}

Checking if the first argument could be a valid module name makes
sense only if we are inside a Weaver directory (otherwise we wouldn't
care about modules) and if we have a first argument. In this case the
argument is a valid module name if it's formed by just alphanumeric
characters, underline and if doesn't exist a file with the same name
and extension \monoespaco{.c} or \monoespaco{.h} in \monoespaco{src/}:

\iniciocodigo
@<Initialization@>+=
if(have_arg && inside_weaver_directory){
  char *buffer;
  int i, size;
  size = strlen(argument);
  // Checando caracteres inválidos no nome:
  for(i = 0; i < size; i ++){
    if(!isalnum(argument[i]) && argument[i] != '_'){
      goto NOT_VALID_MODULE;
    }
  }
  // Checking name conflicts:
  buffer = concatenate(project_path, "src/", argument, ".c", "");
  if(buffer == NULL) W_ERROR();
  if(directory_exist(buffer) != DONT_EXIST){
    free(buffer);
    goto NOT_VALID_MODULE;
  }
  buffer[strlen(buffer) - 1] = 'h';
  if(directory_exist(buffer) != DONT_EXIST){
    free(buffer);
    goto NOT_VALID_MODULE;
  }
  free(buffer);
  arg_is_valid_module = true;
}
NOT_VALID_MODULE:
@
\fimcodigo

\subsecao{6.9. arg\_is\_valid\_plugin: If the argument could be a plugin name}

An argument is a valid plugin name if it's formed only by alphanumeric
characters and underline and if doesn't exist in the
directory \monoespaco{plugin} a file with the same name and
extension \monoespaco{.c} or \monoespaco{.h}. We also must be inside a
Weaver directory.

\iniciocodigo
@<Initialization@>+=
if(argument2 != NULL && inside_weaver_directory){
  int i, size;
  char *buffer;
  size = strlen(argument2);
  // Checando caracteres inválidos no nome:
  for(i = 0; i < size; i ++){
    if(!isalnum(argument2[i]) && argument2[i] != '_'){
      goto NOT_VALID_PLUGIN;
    }
  }
  // Checando se já existe plugin com mesmo nome:
  buffer = concatenate(project_path, "plugins/", argument2, ".c", "");
  if(buffer == NULL) W_ERROR();
  if(directory_exist(buffer) != DONT_EXIST){
    free(buffer);
    goto NOT_VALID_PLUGIN;
  }
  free(buffer);
  arg_is_valid_plugin = true;
}
NOT_VALID_PLUGIN:
@
\fimcodigo

\subsecao{6.10. arg\_is\_valid\_function: If the second argument could
be the name of a main loop}

This variable will be true if exists a second argument formed only by
alphanumeric characters or underline. And also the first character
must me a letter and it can't have the same name than a reserved word
in C or C++. The last checked versions for these languages is the
draft C++20 and C11.

\iniciocodigo
@<Initialization@>+=
if(argument2 != NULL && inside_weaver_directory &&
   !strcmp(argument, "--loop")){
  int i, size;
  char *buffer;
  // First character isn't digit
  if(isdigit(argument2[0]))
    goto NOT_VALID_FUNCTION;
  size = strlen(argument2);
  // Checking invalid characters
  for(i = 0; i < size; i ++){
    if(!isalnum(argument2[i]) && argument2[i] != '_'){
      goto NOT_VALID_PLUGIN;
    }
  }
  // Checking if file already exist
  buffer = concatenate(project_path, "src/", argument2, ".c", "");
  if(buffer == NULL) W_ERROR();
  if(directory_exist(buffer) != DONT_EXIST){
    free(buffer);
    goto NOT_VALID_FUNCTION;
  }
  buffer[strlen(buffer)-1] = 'h';
  if(directory_exist(buffer) != DONT_EXIST){
    free(buffer);
    goto NOT_VALID_FUNCTION;
  }
  free(buffer);
  // Checking reserved words
  const char *reserved[] = {"alignas", "alignof", "and", "and_eq",
                            "asm", "auto", "bitand", "bitor", "bool",
                            "break", "case", "catch", "char", "char8_t",
                            "char16_t", "char32_t", "class", "compl",
                            "concept", "const", "consteval", "constexpr",
                            "constinit", "const_cast", "continue",
                            "co_await", "co_return", "co_yield",
                            "decltype", "default", "delete", "do",
                            "double", "dynamic_cast", "else", "enum",
                            "explicit", "export", "extern", "false",
                            "float", "for", "friend", "goto", "if",
                            "inline", "int", "long", "mutable",
                            "namespace", "new", "noexcept", "not",
                            "not_eq", "nullptr", "operator", "or",
                            "or_eq", "private", "protected", "public",
                            "register", "reinterpret_cast", "requires",
                            "restrict", "return", "short", "signed",
                            "sizeof", "static", "static_assert",
                            "static_cast", "struct", "switch", "template",
                            "this", "thread_local", "throw", "true",
                            "try", "typedef", "typeid", "typename",
                            "union", "unsigned", "using", "virtual",
                            "void", "volatile", "xor", "xor_eq",
                            "wchar_t", "while", NULL};
  for(i = 0; reserved[i] != NULL; i ++)
    if(!strcmp(argument2, reserved[i]))
      goto NOT_VALID_FUNCTION;
  arg_is_valid_function = true;
}
NOT_VALID_FUNCTION:
@

\subsecao{6.11. author\_name: Code creator name}

The variable |author_name| must contain the name of the user which is
invoking the program. This information is useful to generate copyright
notices in source files of new modules.

Initialize this information is different in Unix and Windows systems.
In Unix systems we begin getting their UID. Then we get login
information with |getpwuid|. If the user has a registered name
in \monoespaco{/etc/passwd}, we use this name. Otherwise, we just use
the login name.

\iniciocodigo
@<Initialization@>+=
#if !defined(_WIN32)
{
  struct passwd *login;
  int size;
  char *string_to_copy;
  login = getpwuid(getuid()); // Get user data
  if(login == NULL) W_ERROR();
  size = strlen(login -> pw_gecos);
  if(size > 0)
    string_to_copy = login -> pw_gecos;
  else
    string_to_copy = login -> pw_name;
  size = strlen(string_to_copy);
  author_name = (char *) malloc(size + 1);
  if(author_name == NULL) W_ERROR();
#ifdef __OpenBSD__
  strlcpy(author_name, string_to_copy, size + 1);
#else
  strcpy(author_name, string_to_copy);
#endif
}
#endif
@
\fimcodigo

In Windows the name can be obtained with the function
|GetUserNameExA|. In the first invocation we try to get the buffer
size to store the name and in the second we get the name. In case of
error, we try the more ancient function |GetUserNameA| which will
return a simpler username instead of the full name. And in this case,
we allocate in the buffer space to store the biggest valid username.

\iniciocodigo
@<Initialization@>+=
#if defined(_WIN32)
{
  int size = 0;
  GetUserNameExA(NameDisplay, author_name, &size);
  if(GetLastError() == ERROR_MORE_DATA){
    if(size == 0)
      size = 64;
    author_name = (char *) malloc(size);
    if(GetUserNameExA(NameDisplay, author_name, &size) == 0){
      size = UNLEN + 1;
      author_name = (char *) malloc(size);
      GetUserNameA(author_name, &size);
    }
  }
  else{
    size = UNLEN + 1;
    author_name = (char *) malloc(size);
    GetUserNameA(author_name, &size);
  }
}
#endif
@
\fimcodigo

After we need to free the memory allocated to |author_name|:

\iniciocodigo
@<Finishing@>+=
if(author_name != NULL) free(author_name);
@
\fimcodigo

For this code work, we need to insert the correct library according
with the Operating System. In Unix is \monoespaco{pwd.h} to
get \monoespaco{getpwuid} and in Windows we insert the security
header.

@<Headers Included in Weaver Program@>=
#if !defined(_WIN32)
#include <pwd.h> // getpwuid
#else
#define SECURITY_WIN32
#include <Security.h>
#include <Lmcons.h>
#endif
@

\subsecao{6.12. project\_name: The project name}

Talking about the project name makes sense only when we are in a
Weaver project directory. In this case, the project name can be found
in the file \monoespaco{.weaver/name} inside the base directory:

\iniciocodigo
@<Initialization@>+=
if(inside_weaver_directory){
  FILE *fp;
  char *c;
#if !defined(_WIN32)
  char *filename = concatenate(project_path, ".weaver/name", "");
#else
  char *filename = concatenate(project_path, ".weaver\name", "");
#endif
  if(filename == NULL) W_ERROR();
  project_name = (char *) malloc(256);
  if(project_name == NULL){
    free(filename);
    W_ERROR();
  }
  fp = fopen(filename, "r");
  if(fp == NULL){
    free(filename);
    W_ERROR();
  }
  c = fgets(project_name, 256, fp);
  fclose(fp);
  free(filename);
  if(c == NULL) W_ERROR();
  project_name[strlen(project_name)-1] = '\0';
  project_name = realloc(project_name, strlen(project_name) + 1);
  if(project_name == NULL) W_ERROR();
}
@
\fimcodigo

After we need to free the memory allocated to |project_name|:

\iniciocodigo
@<Finishing@>+=
if(project_name != NULL) free(project_name);
@
\fimcodigo

\subsecao{6.13. year: Current year}

The current year can be obtained with the function |localtime| in
any Operating System:

\iniciocodigo
@<Initialization@>+=
{
  time_t current_time;
  struct tm *date;
  time(&current_time);
  date = localtime(&current_time);
  year = date -> tm_year + 1900;
}
@
\fimcodigo

The prerequisite is include the headers with time functions:

\iniciocodigo
@<Headers Included in Weaver Program@>=
#include <time.h> // localtime, time
@
\fimcodigo

\secao{7. Use Cases}

\subsecao{7.1. Printing help about project creation}

The first use case happens when Weaver is called outside a Weaver
project directory and the invocation is without arguments or with the
argument \monoespaco{--help}. In this case, we assume that the user
doesn't know entirelly how to use the program and print a help
message. The message will be something like this:

\alinhaverbatim
    .  .   You are outside a Weaver Directory.
   ./  \\.  The following command uses are available:
   \\\\  //
   \\\\()//  weaver
   .={}=.      Print this message and exits.
  / /`'\\ \\
  ` \\  / '  weaver PROJECT_NAME
     `'        Creates a new Weaver Directory with a new
               project.
\alinhanormal

This is made with the following code:

\iniciocodigo
@<Use Case 1: Printing help (create project)@>=
if(!inside_weaver_directory && (!have_arg || !strcmp(argument, "--help"))){
  printf("    .  .     You are outside a Weaver Directory.\n"
  "   .|  |.    The following command uses are available:\n"
  "   ||  ||\n"
  "   \\\\()//  weaver\n"
  "   .={}=.      Print this message and exits.\n"
  "  / /`'\\ \\\n"
  "  ` \\  / '  weaver PROJECT_NAME\n"
  "     `'        Creates a new Weaver Directory with a new\n"
  "               project.\n");
  END();
}
@
\fimcodigo


\subsecao{7.2. Printing help about project management}

The second use case is also very simple. It happens when we are
already in a Weaver directory and we call Weaver without arguments or
with a \monoespaco{--help}. We assume in this case that the user wants
instructions about creation of new modules or other parts of a
project. The printed message will be:

\alinhaverbatim
       \\              You are inside a Weaver Directory.
        \\______/      The following command uses are available:
        /\\____/\\
       / /\\__/\\ \\       weaver
    __/_/_/\\/\\_\\_\\___     Prints this message and exits.
      \\ \\ \\/\\/ / /
       \\ \\/__\\/ /       weaver NAME
        \\/____\\/          Creates NAME.c and NAME.h, updating
        /      \\          the Makefile and headers
       /
                          weaver --loop NAME
                           Creates a new main loop in a new file src/NAME.c

                          weaver --plugin NAME
                           Creates new plugin in plugin/NAME.c

                          weaver --shader NAME
                           Creates a new shader directory in shaders/
\alinhanormal

We get this with the code:

\iniciocodigo
@<Use Case 2: Printing management help@>=
if(inside_weaver_directory && (!have_arg || !strcmp(argument, "--help"))){
  printf("       \\                You are inside a Weaver Directory.\n"
  "        \\______/        The following command uses are available:\n"
  "        /\\____/\\\n"
  "       / /\\__/\\ \\       weaver\n"
  "    __/_/_/\\/\\_\\_\\___     Prints this message and exits.\n"
  "      \\ \\ \\/\\/ / /\n"
  "       \\ \\/__\\/ /       weaver NAME\n"
  "        \\/____\\/          Creates NAME.c and NAME.h, updating\n"
  "        /      \\          the Makefile and headers\n"
  "       /\n"
  "                        weaver --loop NAME\n"
  "                         Creates a new main loop in a new file src/NAME.c\n\n"
  "                        weaver --plugin NAME\n"
  "                         Creates a new plugin in plugin/NAME.c\n\n"
  "                        weaver --shader NAME\n"
  "                         Creates a new shader directory in shaders/\n");
  END();
}
@
\fimcodigo

\subsecao{7.3. Showing the Weaver version}

A use case even simpler. It happens when a user calls Weaver with the
argument \monoespaco{--version}:

\iniciocodigo
@<Use Case 3: Print version@>=
if(have_arg && !strcmp(argument, "--version")){
  printf("Weaver\t%s\n", VERSION);
  END();
}
@
\fimcodigo

\subsecao{7.4. Updating existing Weaver projects}

This use case happens when the user pass as argument an absolute or
relative path to a Weaver directory. We assume in this case that the
user wants toupdate the project passed as argument.Perhaps it was
created with an older version and they want to update to a newer
version.

Naturally, this will be done only if the Weaver version installed in
the machine is equal or higher than the version used in the
project. Or if we have installed a test version. In the later case, we
assume that the user wants to try the experimental version. Ignoring
test versions, it's not possible to make downgrades, going from 0.2
version to 0.1.

If the project version and the installed version is the same, we still
update. This is a useful way to corrrect a project if someone erases
or corrupt an important source file in \monoespaco{src/weaver/}.

Experimental versions always are identified as having a name formed
only by alphabetic characters (``Alpha'' ou ``Beta'', for
example). Stable versions are formed by one or more digits and a dot
followed by one or more digits (the majr and minor version). As the
version numbers are interpreted using |atoi|, if we are using an
experimental version, we identify the major and minor version as 0.

Projects in experimental versions always are updated, even if the
installed version is older than the project experimental version.

Updating consists in copying all the files in the Weaver shared
directory in \monoespaco{project/src/weaver} to the directory 
\monoespaco{src/weaver} inside the project. For this we use the
auxiliary functions defined in previous sections.

\iniciocodigo
@<Use Case 4: Updating Weaver project@>=
if(arg_is_path){
  if((weaver_version_major == 0 && weaver_version_minor == 0) ||
     (weaver_version_major > project_version_major) ||
     (weaver_version_major == project_version_major &&
      weaver_version_minor >= project_version_minor)){
    char *buffer, *buffer2;
    // |buffer| <- SHARED_DIR/project/src/weaver
    buffer = concatenate(shared_dir, "project/src/weaver/", "");
    if(buffer == NULL) W_ERROR();
    // |buffer2| <- PROJECT_DIR/src/weaver/
    buffer2 = concatenate(argument, "/src/weaver/", "");
    if(buffer2 == NULL){
      free(buffer);
      W_ERROR();
    }
    if(copy_files(buffer, buffer2) == 0){
      free(buffer);
      free(buffer2);
      W_ERROR();
    }
    free(buffer);
    free(buffer2);
  }
  END();
}
@
\fimcodigo

\subsecao{7.5. Adding a new module in a Weaver project}

If we are inside a Weaver project directory and we got an argument, we
assume that this argument is a new module for our game. If the
argument is a valid module name, we create the new module with this
name. Otherwise we print an error message and exit.

Creating a new module involves:

a) Creating base files \monoespaco{.c} and \monoespaco{.h}, giving
them the same name as the new module.

b) Adding in both files a copyright and licensing comment with author
name, project name and current year.

c) Adding in \monoespaco{.h} macro code to prevent the header of being
inserted more than one time and including the module header in the
file with \monoespaco{.c} extension.

d) Including the module header with extension \monoespaco{.h} in the
file \monoespaco{src/includes.h} to ensure that its functions and
structures can be acessed from all the other source files.

The code for this is:

\iniciocodigo
@<Use Case 5: Create new module@>=
if(inside_weaver_directory && have_arg &&
   strcmp(argument, "--plugin") && strcmp(argument, "--shader") &&
   strcmp(argument, "--loop")){
  if(arg_is_valid_module){
    char *filename;
    FILE *fp;
    // Creating module.c
    filename = concatenate(project_path, "src/", argument, ".c", "");
    if(filename == NULL) W_ERROR();
    fp = fopen(filename, "w");
    if(fp == NULL){
      free(filename);
      W_ERROR();
    }
    write_copyright(fp, author_name, project_name, year);
    fprintf(fp, "#include \"%s.h\"", argument);
    fclose(fp);
    filename[strlen(filename)-1] = 'h'; // Creating module.h
    fp = fopen(filename, "w");
    if(fp == NULL){
      free(filename);
      W_ERROR();
    }
    write_copyright(fp, author_name, project_name, year);
    fprintf(fp, "#ifndef _%s_h_\n", argument);
    fprintf(fp, "#define _%s_h_\n\n#include \"weaver/weaver.h\"\n",
            argument);
    fprintf(fp, "#include \"includes.h\"\n\n#endif");
    fclose(fp);
    free(filename);
    // Updating src/includes.h to insert modulo.h:
    fp = fopen("src/includes.h", "a");
    fprintf(fp, "#include \"%s.h\"\n", argument);
    fclose(fp);
  }
  else{
    fprintf(stderr, "ERROR: This module name is invalid.\n");
    return_value = 1;
  }
  END();
}
@
\fimcodigo

\subsecao{7.6. Creating a new project}

Creating a new Weaver project involves creating a directory with the
project name, copying to there all the content of
directory \monoespaco{project} in Weaver shared directory and creating
a directory \monoespaco{.weaver} with the project data. We also create
a \monoespaco{src/game.c} and \monoespaco{src/game.h} adding the
copyright comment there and copying inside them the basic structure
from the files \monoespaco{basefile.c} and \monoespaco{basefile.h}.
We also create an empy \monoespaco{src/includes.h} to be written in
the future when new modules will be created.

\iniciocodigo
@<Use Case 6: Create new project@>=
if(! inside_weaver_directory && have_arg){
  if(arg_is_valid_project){
    int err;
    char *dir_name;
    FILE *fp;
    err = create_dir(argument, NULL);
    if(err == -1) W_ERROR();
#if !defined(_WIN32)
    err = chdir(argument);
#else
    err = _chdir(argument);
#endif
    if(err == -1) W_ERROR();
    err = create_dir(".weaver", "conf", "tex", "src", "src/weaver",
                     "fonts", "image", "sound", "models", "music",
                     "plugins", "src/misc", "src/misc/sqlite",
                     "compiled_plugins", "shaders", "");
    if(err == -1) W_ERROR();
    dir_name = concatenate(shared_dir, "project", "");
    if(dir_name == NULL) W_ERROR();
    if(copy_files(dir_name, ".") == 0){
      free(dir_name);
      W_ERROR();
    }
    free(dir_name); // Creating file with version number
    fp = fopen(".weaver/version", "w");
    fprintf(fp, "%s\n", VERSION);
    fclose(fp); // Creating file with project name
    fp = fopen(".weaver/name", "w");
    fprintf(fp, "%s\n", basename(argv[1]));
    fclose(fp);
    fp = fopen("src/game.c", "w");
    if(fp == NULL) W_ERROR();
    write_copyright(fp, author_name, argument, year);
    if(append_file(fp, shared_dir, "basefile.c") == 0) W_ERROR();
    fclose(fp);
    fp = fopen("src/game.h", "w");
    if(fp == NULL) W_ERROR();
    write_copyright(fp, author_name, argument, year);
    if(append_file(fp, shared_dir, "basefile.h") == 0) W_ERROR();
    fclose(fp);
    fp = fopen("src/includes.h", "w");
    write_copyright(fp, author_name, argument, year);
    fprintf(fp, "\n#include \"weaver/weaver.h\"\n");
    fprintf(fp, "\n#include \"game.h\"\n");
    fclose(fp);
  }
  else{
    fprintf(stderr, "ERROR: %s is not a valid project name.", argument);
    return_value = 1;
  }
  END();
}
@
\fimcodigo

\subsecao{7.7. Creating a new plugin}

This use case is invoked when we have two arguments, the first one is
|"--plugin"| and the second one is the new plugin name, which can't
conflict with the name of an already existing plugin in the directory
\monoespaco{plugins/}. We must be inside a Weaver project directory to
create a plugin:

\iniciocodigo
@<Use Case 7: Create new plugin@>=
if(inside_weaver_directory && have_arg && !strcmp(argument, "--plugin") &&
   arg_is_valid_plugin){
  char *buffer;
  FILE *fp;
  /* Creating the file: */
  buffer = concatenate("plugins/", argument2, ".c", "");
  if(buffer == NULL) W_ERROR();
  fp = fopen(buffer, "w");
  if(fp == NULL) W_ERROR();
  write_copyright(fp, author_name, project_name, year);
  fprintf(fp, "#include \"../src/weaver/weaver.h\"\n\n");
  fprintf(fp, "void _init_plugin_%s(W_PLUGIN){\n\n}\n\n", argument2);
  fprintf(fp, "void _fini_plugin_%s(W_PLUGIN){\n\n}\n\n", argument2);
  fprintf(fp, "void _run_plugin_%s(W_PLUGIN){\n\n}\n\n", argument2);
  fprintf(fp, "void _enable_plugin_%s(W_PLUGIN){\n\n}\n\n", argument2);
  fprintf(fp, "void _disable_plugin_%s(W_PLUGIN){\n\n}\n", argument2);
  fclose(fp);
  free(buffer);
  END();
}
@
\fimcodigo

\subsecao{7.8. Creating a new shader}

This use case is similar to the previous one, but with some
differences. All shader will be a new file in GLSL format inside the
directory \monoespaco{shaders}. Also their names will always be in the
format of regular expression \monoespaco{[0-9][0-9]*-.*}. The digit(s)
in the firszt part must be unique for each shader in the same
project. And the numbers must always be sequential, starting in 1 and
incrementing with each new shader.

This use case will be invoked when our first argument is
|``--shader''| and the second one is any name. We don't need to
require names in a particular format because the numeric restrictions
over the initial digits ensure that each shader will have a unique and
nonconflicting name.

To ensure this restriction, the code will count the number of files
with GLSL extension in the shader directory and create a new shader
with the name \monoespaco{DD-XX.glsl}, where \monoespaco{DD} is the
number of existing files plus 1 and \monoespaco{XX} is the name chosen
as the second argument got by the program. But if we find holes in the
shader numbering, for example if exists a shader 3 without a shader 2,
we will choode \monoespaco{DD} to fill the hole.

After setting the shader numbering, we create it as an empty file and
copy the content from an existing file in the shared directory where
Weaver is installed.

Depois de descobrir a numeração do novo shader, basta criarmos ele
como um arquivo vazio e depois copiarmos o conteúdo de um modelo já
existente em nosso diretório de instalação.

The code of this use case is:

\iniciocodigo
@<Use Case 8: Create new shader@>=
if(inside_weaver_directory && have_arg && !strcmp(argument, "--shader") &&
   argument2 != NULL){
    FILE *fp;
    size_t tmp_size, number = 0;
    char *buffer;
    @<Shader: Count number of files and get shader number@>
    // Creating the shader:
    tmp_size = number / 10 + 7 + strlen(argument2);
    buffer = (char *) malloc(tmp_size);
    if(buffer == NULL) W_ERROR();
    buffer[0] = '\0';
    snprintf(buffer, tmp_size, "%d-%s.glsl", (int) number, argument2);
    fp = fopen(buffer, "w");
    if(fp == NULL){
        free(buffer);
        W_ERROR();
    }
    if(append_file(fp, shared_dir, "shader.glsl") == 0) W_ERROR();
    fclose(fp);
    free(buffer);
    END();
}
@
\fimcodigo

The part where we count the number of files is different in Unix and
in Windows because of different API to deal with the filesystem. What
we will do is iterate over the files in the
directory \monoespaco{shaders} and for each file which isn't a
directory, has the extension GLSL and has a name which starts with a
positive number, we mark its number as true in a boolean vector
initially set as entirely false. After iterating over all files we
check the boolean vector searching for the first position marked as
false. This position is a number still not used for a shader and its
index is our chosen number.

Aditional complexities involves esuring that our boolean vector has a
good size. We initially allocate it with 128 positions, but if we find
a shader with bigger number, we realloc it to deal with the biggger
number.

The code for this in Linux is:

\iniciocodigo
@<Shader: Count number of files and get shader number@>=
#if !defined(_WIN32)
{
  size_t i, max_number = 0;
  DIR *shader_dir;
  struct dirent *dp;
  char *p;
  bool *exists;
  size_t exists_size = 128;
  shader_dir = opendir("shaders/");
  if(shader_dir == NULL)
    W_ERROR();
  exists = (bool *) malloc(sizeof(bool) * exists_size);
  if(exists == NULL){
    closedir(shader_dir);
    W_ERROR();
  }
  for(i = 0; i < exists_size; i ++)
    exists[i] = false;
  while((dp = readdir(shader_dir)) != NULL){
    if(dp -> d_name == NULL) continue;
    if(dp -> d_name[0] == '.') continue;
    if(dp -> d_name[0] == '\0') continue;
    buffer = concatenate("shaders/", dp -> d_name, "");
    if(buffer == NULL) W_ERROR();
    if(directory_exist(buffer) != EXISTS_AND_IS_FILE){
      free(buffer);
      continue;
    }
    for(p = buffer; *p != '\0'; p ++);
    p -= 5;
    if(strcmp(p, ".glsl") && strcmp(p, ".GLSL")){
      free(buffer);
      continue;
    }
    number = atoi(buffer);
    if(number == 0){
      free(buffer);
      continue;
    }
    if(number > max_number)
      max_number = number;
    if(number > exists_size){
      if(number > exists_size * 2)
        exists_size = number;
      else
        exists_size *= 2;
      exists = (bool *) realloc(exists, exists_size * sizeof(bool));
      if(exists == NULL){
        free(buffer);
        closedir(shader_dir);
        W_ERROR();
      }
      for(i = exists_size / 2; i < exists_size; i ++)
        exists[i] = false;
    }
    exists[number - 1] = true;
    free(buffer);
  }
  closedir(shader_dir);
  for(i = 0; i <= max_number; i ++)
    if(exists[i] == false){
      break;
    }
  free(exists);
}
#endif
@
\fimcodigo

In Windows the code to deal with files is different, but the other
parts are the same:

\iniciocodigo
@<Shader: Count number of files and get shader number@>=
#if defined(_WIN32)
{
  int i;
  char *p;
  bool *exists;
  size_t exists_size = 128;
  int number, max_number = 0;
  WIN32_FIND_DATA file;
  HANDLE shader_dir = NULL;
  shader_dir = FindFirstFile("shaders\\", &file);
  if(shader_dir == INVALID_HANDLE_VALUE)
    W_ERROR();
  exists = (bool *) malloc(sizeof(bool) * exists_size);
  if(exists == NULL){
    FindClose(shader_dir);
    W_ERROR();
  }
  for(i = 0; i < exists_size; i ++)
    exists[i] = false;
  do{
    if(file.cFileName == NULL) continue;
    if(file.cFileName[0] == '.') continue;
    if(file.cFileName[0] == '\0') continue;
    buffer = concatenate("shaders\\", file.cFileName, "");
    if(buffer == NULL) W_ERROR();
    if(directory_exist(buffer) != EXISTS_AND_IS_FILE){
      free(buffer);
      continue;
    }
    for(p = buffer; *p != '\0'; p ++);
    p -= 5;
    if(strcmp(p, ".glsl") && strcmp(p, ".GLSL")){
      free(buffer);
      continue;
    }
    number = atoi(buffer);
    if(number == 0){
      free(buffer);
      continue;
    }
    if(number > max_number)
      max_number = number;
    if(number > exists_size){
      if(number > exists_size * 2)
        exists_size = number;
      else
        exists_size *= 2;
      exists = (bool *) realloc(exists, exists_size * sizeof(bool));
      if(exists == NULL){
        free(buffer);
        FindClose(shader_dir);
        W_ERROR();
      }
      for(i = exists_size / 2; i < exists_size; i ++)
        exists[i] = false;
    }
    exists[number - 1] = true;
    free(buffer);
  }while(FindNextFile(shader_dir, &file) != 0);
  FindClose(shader_dir);
  for(i = 0; i <= max_number; i ++)
  if(exists[i] == false){
    break;
  }
  free(exists);
}
#endif
@
\fimcodigo

\subsecao{7.9. Creating a new main loop}

This use case happens when the second argument is
\monoespaco{--loop} and when the next argument is a valid name
for a C and C++ function. If it's not, we print an error message to
warn the user.

In this case we can't just copy the content of a base file to our new
file as we did previously in the code for a new module, because we
need to define a function with the given name in the file. So we
create the file and write inside it the content hard-coded in the
following code:

\iniciocodigo
@<Use Case 9: Create new main loop@>=
if(inside_weaver_directory && !strcmp(argument, "--loop")){
  if(!arg_is_valid_function){
    if(argument2 == NULL)
      fprintf(stderr,
              "ERROR: You should pass a name for your new loop.\n");
    else
      fprintf(stderr, "ERROR: %s not a valid loop name.\n", argument2);
    W_ERROR();
  }
  char *filename;
  FILE *fp;
  // Creating LOOP_NAME.c
  filename = concatenate(project_path, "src/", argument2, ".c", "");
  if(filename == NULL) W_ERROR();
  fp = fopen(filename, "w");
  if(fp == NULL){
    free(filename);
    W_ERROR();
  }
  write_copyright(fp, author_name, project_name, year);
  fprintf(fp, "#include \"%s.h\"\n\n", argument2);
  fprintf(fp, "MAIN_LOOP %s(void){\n", argument2);
  fprintf(fp, " LOOP_INIT:\n\n");
  fprintf(fp, " LOOP_BODY:\n");
  fprintf(fp, "  if(W.keyboard[W_ANY])\n");
  fprintf(fp, "    Wexit_loop();\n");
  fprintf(fp, " LOOP_END:\n");
  fprintf(fp, "  return;\n");
  fprintf(fp, "}\n");
  fclose(fp);
  // Creating LOOP_NAME.h
  filename[strlen(filename)-1] = 'h';
  fp = fopen(filename, "w");
  if(fp == NULL){
    free(filename);
    W_ERROR();
  }
  write_copyright(fp, author_name, project_name, year);
  fprintf(fp, "#ifndef _%s_h_\n", argument2);
  fprintf(fp, "#define _%s_h_\n#include \"weaver/weaver.h\"\n\n", argument2);
  fprintf(fp, "#include \"includes.h\"\n\n");
  fprintf(fp, "MAIN_LOOP %s(void);\n\n", argument2);
  fprintf(fp, "#endif\n");
  fclose(fp);
  free(filename);
  // Updating src/includes.h
  fp = fopen("src/includes.h", "a");
  fprintf(fp, "#include \"%s.h\"\n", argument2);
  fclose(fp);  
}
@
\fimcodigo

\secao{8. Conclusion}

This end the necessary code to create the Weaver program, a manager of
eaver projects.

The progam described here, in fact, doesn't contain everything to
manage a project. There's somo work too in the Weaver installer, which
in Unix is a Makefile and in Windows systems is a MSIX package.

The Weaver library code also is part of the Weaver engine, but will
have their code described in other articles.

There's also some base code for shaders and new projects which should
be find in the directory \monoespaco{project} in the source code.

And finally there's also the complex code of Makefiles and Window's
\monoespaco{.vcxproj} files, where more management code can be found.

\fim
