\input tex/epsf.tex
\font\sixteen=cmbx16
\font\twelve=cmr12
\font\fonteautor=cmbx12
\font\fonteemail=cmtt10
\font\twelvenegit=cmbxti12
\font\twelvebold=cmbx12
\font\trezebold=cmbx13
\font\twelveit=cmsl12
\font\monodoze=cmtt12
\font\it=cmti12
\voffset=0,959994cm % 3,5cm de margem superior e 2,5cm inferior
\parskip=6pt

\def\titulo#1{{\noindent\sixteen\hbox to\hsize{\hfill#1\hfill}}}
\def\autor#1{{\noindent\fonteautor\hbox to\hsize{\hfill#1\hfill}}}
\def\email#1{{\noindent\fonteemail\hbox to\hsize{\hfill#1\hfill}}}
\def\negrito#1{{\twelvebold#1}}
\def\italico#1{{\twelveit#1}}
\def\monoespaco#1{{\monodoze#1}}
\def\iniciocodigo{\lineskip=0pt\parskip=0pt}
\def\fimcodigo{\twelve\parskip=0pt plus 1pt\lineskip=1pt}

\long\def\abstract#1{\parshape 10 0.8cm 13.4cm 0.8cm 13.4cm
0.8cm 13.4cm 0.8cm 13.4cm 0.8cm 13.4cm 0.8cm 13.4cm 0.8cm 13.4cm
0.8cm 13.4cm 0.8cm 13.4cm 0.8cm 13.4cm
\noindent{{\twelvenegit Abstract: }\twelveit #1}}

\def\resumo#1{\parshape  10 0.8cm 13.4cm 0.8cm 13.4cm
0.8cm 13.4cm 0.8cm 13.4cm 0.8cm 13.4cm 0.8cm 13.4cm 0.8cm 13.4cm
0.8cm 13.4cm 0.8cm 13.4cm 0.8cm 13.4cm
\noindent{{\twelvenegit Resumo: }\twelveit #1}}

\def\secao#1{\vskip12pt\noindent{\trezebold#1}\parshape 1 0cm 15cm}
\def\subsecao#1{\vskip12pt\noindent{\twelvebold#1}}
\def\referencia#1{\vskip6pt\parshape 5 0cm 15cm 0.5cm 14.5cm 0.5cm 14.5cm
0.5cm 14.5cm 0.5cm 14.5cm {\twelve\noindent#1}}

%@* .

\twelve
\vskip12pt
\titulo{The Weaver API}
\vskip12pt
\autor{Thiago Leucz Astrizi}
\vskip6pt
\email{thiago@@bitbitbit.com.br}
\vskip6pt

\abstract{This article describes using literary programming the
  Weaver API. Weaver is a game engine and this API are how programmers
  interact with the engine in their game projects. Besides the API,
  this article also covers how the configuration file is interpreted
  and how game loops should be managed in a game project. The API is
  portable code which should work under OpenBSD, Linux, Windows and
  Web Assembly environments.}

\secao{1. Introduction}

\subsecao{1.1. File Organization}

When a user types \monoespaco{weaver PROJECT} at command line, a
directory with a new Weaver project is created. Inside the directory,
the file with the main loop is in \monoespaco{src/game.c} and inside
them we find:

\linha
\alinhaverbatim
#include "game.h"

void main_loop(void){ // The game main loop
 LOOP_INIT: // Initialization code 

 LOOP_BODY: // Code executed each iteration
    if(W.keyboard[W_ANY])
        Wexit_loop();
 LOOP_END: // Code executed at finalization
    return;
}

int main(void){
  Winit(); // Initializes Weaver
  Wloop(main_loop); // Enter a new game loop
  return 0;
}
\alinhanormal
\linha

And inside \monoespaco{src/game.h}, we find:

\linha
\alinhaverbatim
#ifndef _game_h_
#define _game_h_

#include "weaver/weaver.h"
#include "includes.h"

struct _game_struct{
  // You can customise this structure declaring variables here.
  // But don't change it's name. And access it by its pointer: W.game
  int whatever; // <- This variable is here to prevent compiler errors
} _game;

void main_loop(void);

#endif
\alinhanormal
\linha

In this file there's an struct which can be customized by the user and
which is where variables with global states should be declared. The
variables declared here are also variables which will be saved in a
player's save file and which will be sent over a network to inform
clients about the game state. This struct should be referenced by the
variable \monoespaco{W.game}. This also gives us information that the
API will be organized in a way that will exist a \monoespaco{struct}
called \monoespaco{W} where the API resources will be centralized.

The file \monoespaco{includes.h} is just a header which includes in
the project all the other headers of modules created by the user (each
module is a C source file and a header).

All the API code shall be present or be included by macros in the
files \monoespaco{weaver.c} and \monoespaco{weaver.h}. The
\monoespaco{weaver.h} organization is:

\iniciocodigo
@(project/src/weaver/weaver.h@>=
#ifndef _weaver_h_
#define _weaver_h_
#ifdef __cplusplus
  extern "C" {
#endif
#include "../../conf/conf.h"
@<Global Structure@>
@<Weaver Headers@>
@<Weaver Macros@>
#ifdef __cplusplus
  }
#endif
#endif
@
\fimcodigo

Notar que incluímos no cabeçalho o arquivo de
configuração \monoespaco{conf.h} responsável por controlar e
configurar o comportamento de nosso motor.

\subsecao{1.2. A Estrutura W}

A tal ``estrutura global'' referenciada nos códigos acima é
o \monoespaco{struct} chamado \monoespaco{W}. Já mencionamos no
comentário que colocaremos nele
o \monoespaco{struct \_game\_struct \_game} que definimos
em \monoespaco{game.h}. Podemos então começar a definir tal estrutura:

\iniciocodigo
@<Global Structure@>=
// Esta estrutura conterá todas as variáveis e funções definidas pela
// API Weaver:
extern struct _weaver_struct{
  struct _game_struct *game;
  @<Variáveis Weaver@>
  //@<Funções Weaver@>
} W;
@
\fimcodigo

Notar que além de \monoespaco{W.game}, existirão outras variáveis
presentes dentro desta estrutura. Basicamente iremos centralizar
dentro dela todas as funções públicas da nossa API. Só não estarão
nela funções que começam com ``\_'', e que são consideradas internas e
não deveriam ser usadas por programadores utilizando a API. Desta
forma deixamos bem delimitado o que faz parte da API e também evitamos
poluir com nomes o ``namespace'' global de programas em C.

Também definiremos aqui a estrutura geral de nosso
arquivo \monoespaco{weaver.c}:

\iniciocodigo
@(project/src/weaver/weaver.c@>=
#include "weaver.h"
#include "../game.h"
@<API Weaver: Definições@>
@<API Weaver: Funções@>
@<API Weaver: Base@>
@
\fimcodigo

Nas definições declaramos novos tipos de estruturas de dados que forem
necessárias. A primeira coisa que já podemos definir é a
estrutura \monoespaco{W}, a qual é apenas declarada no cabeçalho:

\iniciocodigo
@<API Weaver: Definições@>=
struct _weaver_struct W;
@
\fimcodigo

Na parte de funções definimos as funções a serem usadas. Já a última
partedo arquivo contém as funções mais básicas da API. As únicas que
não são colocadas dentro da estrutura \monoespaco{W}. Elas são a
função de inicialização e a função que encerra o funcionamento do
motor.

\subsecao{1.3. Funções de Inicialização e Finalização}

Uma das coisas que a função de inicialização faz é inicializar os
valores da estrutura \monoespaco{W}:

\iniciocodigo
@<Weaver Headers@>=
void Winit(void);
@
\fimcodigo

\iniciocodigo
@<API Weaver: Base@>=
void Winit(void){
  W.game = &_game;
  @<API Weaver: Inicialização@>
}
@
\fimcodigo

A função de finalização deve desalocar qualquer memória pendente,
finalizar o uso de recursos, e deve também fechar o programa
informando que tudo correu bem se assim realmente ocorreu:

\iniciocodigo
@<Weaver Headers@>=
void Wexit(void);
@
\fimcodigo

\iniciocodigo
@<API Weaver: Base@>=
void Wexit(void){
  //@<API Weaver: Finalização@>
  exit(0);
}
@
\fimcodigo

O uso da função \monoespaco{exit} nos obriga a inserir o cabeçalho:

\iniciocodigo
@<Weaver Headers@>=
#include <stdlib.h>
@
\fimcodigo

Os demais códigos que serão executados durante a inicialização e
finalização serão descritos ao longo do artigo.

\secao{2. Contagem de Tempo}

Weaver mede o tempo em microssegundos ($10^{-6}$s) e armazena a sua
contagem de tempo em pelo menos 64 bits de memória. Weaver serve para
criar programas que executam dentro de um laço principal. Então além
do tempo total em microssegundos desde que o programa inicializou,
também armazenamos a diferença de tempo entre a iteração atual do
programa e a última.

Então para começar devemos ter um lugar onde devemos armazenar a
última medida de tempo que fizemos. Usaremos para isso uma variável
global. No Windows usamos um dos tipos específicos para representar
inteiros grandes (e incluimos o cabeçalho necessário para usá-lo)
e nos demais sistemas usamos uma estrutura de valor de tempo de
alta resolução.

\iniciocodigo
@<Weaver Headers@>=
#if defined(_WIN32)
#include <windows.h>
LARGE_INTEGER _last_time;
#else
#include <sys/time.h>
struct timeval _last_time;
#endif
@
\fimcodigo

A ideia é que esta variável armazene sempre a última medida de
tempo. Ela é inicializada com a primeira medida de tempo na
inicialização:

\iniciocodigo
@<API Weaver: Inicialização@>=
#if defined(_WIN32)
QueryPerformanceCounter(&_last_time);
#else
gettimeofday(&_last_time, NULL);
#endif
@
\fimcodigo

Após a inicialização, todas as outras atualizações desta variável
deverão ser feitas por meio da função declarada abaixo:

\iniciocodigo
@<Weaver Headers@>+=
unsigned long _update_time(void);
@
\fimcodigo

Tal função irá ler o tempo atual e armazenar na variável. Ela irá
sempre retornar a diferença de tempo em microssegundos entre a leitura
atual e a última. Em sistemas Unix faremos isso exatamente da maneira
recomendada pelo manual da GNU Glbc de modo a tornar a subtração de
tempo mais portável e funcional mesmo que os elementos da
estrutura \monoespaco{timeval} sejam armazenadas como ``unsigned''.  A
desvantagem é que o código se torna menos claro. O código fica sendo:

\iniciocodigo
@<API Weaver: Definições@>=
#if !defined(_WIN32)
unsigned long _update_time(void){
  int nsec;
  unsigned long result;
  struct timeval _current_time;
  gettimeofday(&_current_time, NULL);
  // Aqui temos algo equivalente ao "vai um" do algoritmo da subtração:
  if(_current_time.tv_usec < _last_time.tv_usec){
    nsec = (_last_time.tv_usec - _current_time.tv_usec) / 1000000 + 1;
    _last_time.tv_usec -= 1000000 * nsec;
    _last_time.tv_sec += nsec;
  }
  if(_current_time.tv_usec - _last_time.tv_usec > 1000000){
    nsec = (_current_time.tv_usec - _last_time.tv_usec) / 1000000;
    _last_time.tv_usec += 1000000 * nsec;
    _last_time.tv_sec -= nsec;
  }
  if(_current_time.tv_sec < _last_time.tv_sec){
    // Overflow
    result = (_current_time.tv_sec - _last_time.tv_sec) * (-1000000);
    // Sempre positivo:
    result += (_current_time.tv_usec - _last_time.tv_usec);
  }
  else{
    result = (_current_time.tv_sec - _last_time.tv_sec) * 1000000;
    result += (_current_time.tv_usec - _last_time.tv_usec);
  }
  _last_time.tv_sec = _current_time.tv_sec;
  _last_time.tv_usec = _current_time.tv_usec;
  return result;
}
#endif
@
\fimcodigo

Em sistema Windows, já existe uma função que trata o tempo como sendo
em microssegundos exatamente no formato que já era usado antes mesmo
de portarmos Weaver para Windows na versão beta. Por causa disso, a
função se torna muito mais simples:

\iniciocodigo
@<API Weaver: Definições@>=
#if defined(_WIN32)
unsigned long _update_time(void){
  LARGE_INTEGER prev;
  prev.QuadPart = _last_time.QuadPart;
  QueryPerformanceCounter(&_last_time);
  return (_last_time.QuadPart - prev.QuadPart);
}
#endif
@
\fimcodigo

\secao{3. Os loops principais.}

Todos os jogos são organizados dentro de loops, ou laços
principais. Eles são basicamente um código que fica iterando
indefinidamente até que uma condição nos leve a outro loop principal,
ou então ao fim do programa.

Como foi mostrado no código inicial do \monoespaco{game.c}, um loop
principal deve ser declarado como:

\alinhaverbatim
void nome\_do\_loop(void){
 LOOP\_INIT: // Código executado na inicialização

 LOOP\_BODY: // Código executado a cada iteração
    if(W.keyboard[W\_ANY])
        Wexit\_loop();
 LOOP\_END: // Código executado na finalização
    return;
}
\alinhanormal

Antes de entendermos como devemos entrar em um loop principal
corretamente, é importante descrever como este loop é
executado. Nota-se que ele possui uma região de inicialização, de
execução e finalização demarcada por rótulos escritos em letras
maiúsculas.

Interpretar isso é bastante simples. É perfeitamente possível
interpretar o código acima como:

\alinhaverbatim
void nome\_do\_loop(void){
  // LOOP\_INIT
  for(;;){
    // LOOP\_BODY
    if(W.keyboard[W\_ANY])
        Wexit\_loop();
  }
  // LOOP\_END
}
\alinhanormal

Mas embora esta interpretação seja suficientemente adequada em certos
contextos, não é assim que este código será traduzido. Não é em todos
os ambientes em que é possível executar um loop infinito sem fazer com
que a interface do jogo trave. Um exemplo é o ambiente WebAssembly de
um navegador de Internet, onde os laços principais de um programa só
podem ser executados se forem corretamente declarados como tais e a
execução deles ocorre não por meio de um laço infinito, mas pela
contínua chamada de uma função de laço principal.

Por causa disso, para tornar o código mais portável devemos encarar
toda execução de um loop principal como no código abaixo:

\alinhaverbatim
for(;;)
  nome\_do\_loop();
\alinhanormal

E dentro da função de loop principal nós não colocamos um laço
explícito. Ao invés disso, nós decidimos qual parte da função deve ser
executada com ajuda dos rótulos inseridos, os quais na verdade são
macros com lógica adicional embutida junto a alguns
comandos \monoespaco{goto} que decidem o que será executado a cada vez
que a função é chamada.

Uma consequência disso é que não é possível lidar com variáveis
declaradas dentro da inicialização de um loop principal. Elas ,só
teriam um valor correto dentro da inicialização, e dentro do corpo do
loop principal seu valor seria indefinido. Por exemplo, o seguinte
código terá resultado indefinido e talvez não imprima nada na tela:

\alinhaverbatim
// ERRADO
void loop(void){
LOOP\_INIT:
  int var = 5;
LOOP\_BODY:
  if(var == 5)
    printf("var == 5\\n");
LOOP\_END:
}
\alinhanormal

Já o seguinte código iria imprimir na saída padrão a cada iteração do
loop:

\alinhaverbatim
// CERTO
static int var;

void loop(void){
LOOP\_INIT:
  var = 5;
LOOP\_BODY:
  if(var == 5)
    printf("var == 5\\n");
LOOP\_END:
}
\alinhanormal

Outra coisa que deve ser levada em conta é que as macros utilizadas
escondem outro detalhe importante: não é apenas um laço principal que
executamos, mas existem dois simultâneos. Um laço principal executa
com uma frequência fixa: o laço que cuida da física e da lógica do
jogo. Outro laço, nós apenas fazemos executar o mais rápido que der no
hardware atual: o laço responsável por renderizar coisas na tela.

Idealmente para cada iteração do laço de renderização executamos uma
ou mais iterações de nosso laço de física e lógica. Isso significa que
podemos renderizar com uma frequência maior que executamos a iteração
responsável por realmente mover objetos, detectar colisões e ler
entrada do usuário. Para que a cada vez nós não renderizemos
exatamente a mesma imagem, o que derrotaria o propósito de fazermos
isso, nós interpolamos a posição dos objetos da tela de acordo com
seus valores de aceleração, velocidade e posição.

Garantindo que a nossa física e lógica do jogo execute sempre em
intervalos constantes, nós garantimos o determinismo necessário para
podermos sincronizar partidas por meio de redes como a Internet. E
renderizando os objetos na tela o mais rápido que podemos com ajuda de
interpolação nos dá a experiência visual mais suave e natural que for
possível.

Uma referência e maiores detalhes de como implementar isso pode ser
encontrado em [Fiedler 2004]. Nossa implementação será como mostrado
na referência, com exceção de que nosso código será muito menos
transparente por ter que estar contido dentro de macros sem usar loops
explícitos.

Vamos agora definir nas subseções seguintes o que exatamente deverá
existir em cada uma das macros que devem estar presentes em todo laço
principal.

\subsecao{3.1. Inicialização do Loop.}

Isso é o que fará a macro \monoespaco{LOOP\_INIT}:

Primeiro devemos checar variáveis que determinam se devemos encerrar o
laço. Se estivermos, mas ainda houverem recursos sendo carregados
(imagens, vídeos, shaders, sons), apenas retornamos da função. Se não
houver nada sendo carregado, mas ainda não executamos a finalização
deste laço, pulamos para executar a finalização. Se não há nada a ser
carregado e já executamos a finalização, aí sim encerramos de vez o
laço. Depois checamos se esta função está sendo chamada pela primeira
vez. Em caso afirmativo, apenas continuamos a execução. Em caso
negativo, fazemos um desvio incondicional para não termos que executar
novamente o código de inicialização. Por fim, se não fizemos o desvio,
faremos com que a variável |W.loop_name| passe a ser uma string com o
nome do laço principal atual.

Saber se devemos continuar executando ou não um laço é algo que pode
ser controlado por uma variável global, não sendo nem necessário se
preocupar com semáforos. Afinal, somente um laço irá executar em um
dado momento. O mesmo pode ser feito com a variável que determina se
estamos na primeira execução de um laço (ou o começo de um laço) e se
já executamos a finalização. Vamos declarar ambas as variáveis:

\iniciocodigo
@<Weaver Headers@>+=
#include <stdbool.h>
bool _running_loop, _loop_begin, _loop_finalized;
@
\fimcodigo

E vamos inicializar elas:

\iniciocodigo
@<API Weaver: Inicialização@>+=
_running_loop = false;
_loop_begin = false;
_loop_finalized = false;
@
\fimcodigo

Saber se ainda estamos carregando arquivos (ou melhor, quantos
arquivos pendentes ainda estamos carregando) ou o nome do laço em que
estamos são duas informações que são úteis não só para a lógica
interna do motor, mas também para o seu usuário. Saber se o laço ainda
não terminou de carregar é útil para fornecer uma tela de
carregamento. Saber durante a execução o nome do laço em que estamos é
útil tanto para depuração como para podermos carregar recursos
diferentes dependendo do laço em que estamos. Por causa disso, ambas
as variáveis devem ser declaradas na estrutura |W|. O tamanho máximo
de nome de um laço que podemos armazenar pode ser personalizado com a
macro \monoespaco{W\_MAX\_LOOP\_NAME}.

\iniciocodigo
@<Variáveis Weaver@>+=
// Dentro da estrutura W:
#if !defined(W_MAX_LOOP_NAME)
#define W_MAX_LOOP_NAME 64
#endif
unsigned pending_files;
char loop_name[W_MAX_LOOP_NAME];
@
\fimcodigo

Na inicialização ajustamos tais variáveis como 0 e |NULL|
respectivamente:

\iniciocodigo
@<API Weaver: Inicialização@>+=
W.pending_files = 0;
W.loop_name[0] = '\0';
@
\fimcodigo

A função que usaremos para sair do laço será esta:

\iniciocodigo
@<Weaver Headers@>+=
#if !defined(_MSC_VER)
void _exit_loop(void) __attribute__ ((noreturn));
#else
__declspec(noreturn) void _exit_loop(void);
#endif


@
\fimcodigo

Mas não iremos definir ela ainda. Pelo cabeçalho nota-se que é uma
função que nunca retorna, tendo isso especificado tanto pela convenção
do GCC e Clang como pela convenção do Visual Studio. Isso porque o que
ela fará é apenas chamar o código do laço anterior, ou sair do
programa dependendo do caso.

Após descrever tudo isso, podemos enfim definir a macro de início de
laço:

\iniciocodigo
@<Weaver Headers@>+=
#define LOOP_INIT                                                   \
  if(!_running_loop){                                               \
    if(W.pending_files)                                             \
      return;                                                       \
    if(!_loop_finalized){                                           \
      _loop_finalized = true;                                       \
      goto _LOOP_FINALIZATION;                                      \
    }                                                              \
    _exit_loop();                                                   \
  }                                                                 \
  if(!_loop_begin)                                                   \
    goto _END_LOOP_INITIALIZATION;                                   \
  snprintf(W.loop_name, W_MAX_LOOP_NAME, "%s", __func__);            \
  _BEGIN_LOOP_INITIALIZATION
@
\fimcodigo

Terminamos com o
identificador \monoespaco{\_BEGIN\_LOOP\_INITIALIZATION}, o qual será
o verdadeiro nome do rótulo que existirá por trás de nossa macro.

\subsecao{3.2. Corpo do Loop.}

Isso é o que fará a macro \monoespaco{LOOP\_BODY}:

Ao chega nesta macro, devemos ajustar como falsa a informação de que é
nossa primeira execução do laço, pois assim não iremos executar a
inicialização novamente. Em seguida, aproveitamos para colocar um
desvio incondicional por trás de um \monoespaco{if} que garanta que
ele nunca seja executado para o rótulo que termina a macro
anterior. Esse desvio nunca irá ocorrer, mas isso previne que o
compilador reclame que o rótulo que encerra a última macro não é
usado. Em seguida, crimaos o verdadeiro rótulo que marca o fim da
inicialização e o começo da execução do corpo do laço. Neste começo de
corpo do laço nós medimos o tempo que passou desde o último laço e
armazenamos em um acumulador. Se este acumulador tiver um valor maior
que o período de tempo entre execuções do código de física e lógica,
então executamos o código presente no laço e o código associado à
física. Caso contrário, só ignoramos tudo e vamos para a finalização
onde apenas renderizamos na tela.  Caso tenha passado um longo tempo
entre cada iteração de laço, executamos mais de uma vez o código do
corpo do laço.

O acumulador que usamos para saber se devemos executar a lógica e a
física do jogo será chamado de \monoespaco{\_lag}. Declaramos ele
globalmente:

\iniciocodigo
@<Weaver Headers@>+=
unsigned long _lag;
@
\fimcodigo

E o inicializamos:

@<API Weaver: Inicialização@>+=
_lag = 0;
@

Existirão variáveis que poderão ser lidas pelo usuário com informações
de tempo. Uma delas (\monoespaco{W.t}) conterá a quantidae de
microssegundos desde que o jogo inicializou. Outra delas
(\monoespaco{W.dt}) conterá o intervalo entre execuções do motor de
física e da lógica do jogo. Ambas as variáveis precisam ser declaradas
na estrutura \monoespaco{W}:

\iniciocodigo
@<Variáveis Weaver@>+=
unsigned long long t;
unsigned long dt;
@
\fimcodigo

A primeira das variáveis, obviamente deve ser inicializada como
zero. A segunda deve ser inicializada como tendo o mesmo valor que uma
macro \monoespaco{W\_TIMESTEP} que pode ser definida pelo usuário para
que assim ele controle a granularidade de execução do código mais
pesado do motor de física e lógica do jogo. Se esta macro não for
definida, iremos assumir um valor de 40000 microssegundos. Isso
significa uma freuência de 25 Hz de execução do motor de física (25
vezes por segundo).

\iniciocodigo
@<API Weaver: Inicialização@>+=
#if !defined(W_TIMESTEP)
#define W_TIMESTEP 40000
#endif
W.dt = W_TIMESTEP;
W.t = 0;
@
\fimcodigo

O código da engine de física e lógica interna do jogo deve ficar
encapsulado em uma função chamada \monoespaco{\_update}:

\iniciocodigo
@<Weaver Headers@>+=
void _update(void);
@
\fimcodigo

Por hora ainda não iremos definir o código a ser executado nesta
função:

\iniciocodigo
@<API Weaver: Base@>+=
void _update(void){
  @<Código a executar todo loop@>
}
@
\fimcodigo

Mas com as definições que fizemos já podemos definir a nossa macro que
marca o começo do código do laço principal:

\iniciocodigo
@<Weaver Headers@>+=
#define LOOP_BODY                                            \
  _loop_begin =  false;                                      \
  if(_loop_begin)                                            \
    goto _BEGIN_LOOP_INITIALIZATION;                         \
_END_LOOP_INITIALIZATION:                                    \
  _lag += _update_time();                                    \
  while(_lag >= W.dt){                                       \
    _update();                                               \
_LABEL_0
@
\fimcodigo

Notar que a macro acima abre um laço com um \monoespaco{while}, mas
não o encerra. Ele deverá ser encerrado pelo código inserido pela
macro que delimita o fim do corpo do laço principal. A qual também
deverá decrementar a variável \monoespaco{\_lag} para que este não
seja um laço infinito.

\subsecao{3.3. Finalização do Loop.}

Isso é o que fará a macro \monoespaco{LOOP\_END}:

Primeiro para fornecer uma condição de parada para o laço começado na
macro anterior, decrementaremos de \monoespaco{\_lag} o valor
de \monoespaco{W.dt}. Em seguida, incrementaremos \monoespaco{W.t} com
tal quantidade de microssegundos. Só então encerramos o bloco do laço
dentro do qual estávamos. Fora deste laço, seguimos realizando aquilo
que deve ser feito independente de estarmos executando o motor de
física e de lógica de jogo ou não. Como são sempre atividades
relacionadas à renderização na tela, isso estará dentro de uma função
chamada \monoespaco{\_render}. Em seguida, simplesmente retornamos. Se
estamos executando o código aqui, é porque estávamos terminando de
executar o corpo do link. Depois do retorno colocamos um desvio para
um rótulo que nunca será executado e apenas nos protege de aviso do
compilador. E enfim colocamos um rótulo imediatamente antes da
finalização e que pode ser atingido somente por um desvio se
detectarmos que o laço acabou e precisamos rodar a finalização.

A única coisa nova que temos aqui então é a função de renderização:

\iniciocodigo
@<Weaver Headers@>+=
void _render(void);
@
\fimcodigo

Por hora ainda não iremos definir o código a ser executado nesta
função:

\iniciocodigo
@<API Weaver: Base@>+=
void _render(void){
  @<Código de renderização@>
}
@
\fimcodigo

E agora colocamos o código da macro:

\iniciocodigo
@<Weaver Headers@>+=
#define LOOP_END                                           \
    _lag -=  40000;                                        \
    W.t +=  40000;                                         \
  }                                                        \
  _render();                                               \
  return;                                                  \
  goto _LABEL_0;                                           \
_LOOP_FINALIZATION
@
\fimcodigo

\subsecao{3.4. Entrando em Laço Principal.}

Frequantemente estaremos trocando de laços principais ao longo de um
jogo. Mas nem sempre isso significa uma substituição completa. Alguns
laços principais ocorrem dentro de outros laços principais. Por
exemplo, quando acessamos um menu de configurações durante um jogo. Ou
quando em um RPG clássico por turnos mudamos para o modo de combate
após um encontro aleatório.

Podemos então formar uma pilha de laços principais, onde ao sairmos do
último laço voltamos para o que está empilhado imediatamente abaixo
dele ao invés de sairmos do jogo.

Sendo assim, existem duas formas de entrar em um laço principal. Uma
delas, por meio da função que definiremos chamada \monoespaco{Wloop} e
a segunda por meio da \monoespaco{Wsubloop}. A primeira envolve
substituirmos o laço principal atual por um novo. A segunda enolve
apenas criar um laço principal dentro do laço atual. A primeira é algo
que só poderemos fazer se não houverem arquivos pendentes sendo
carregados (possivelmente haverá uma tela de carregamento neste
caso). Por isso apenas nos asseguramos disso por mieo de um truque com
macros:

\iniciocodigo
@<Weaver Headers@>+=
#if !defined(_MSC_VER)
void _Wloop(void (*f)(void)) __attribute__ ((noreturn));
void Wsubloop(void (*f)(void)) __attribute__ ((noreturn));
#else
__declspec(noreturn) void _Wloop(void (*f)(void));
__declspec(noreturn) void Wsubloop(void (*f)(void));
#endif
#define Wloop(a) ((W.pending_files)?(false):(_Wloop(a)))
@
\fimcodigo

Nenhum dos dois tipos de função irá retornar jamais. Então
especificamos isso tanto na convenção de compiladores como Clang e GCC
como na do Visual Studio.

Vamos precisar de uma pilha de laços principais, que representaremos
por meio de uma sequência de ponteiros para funções. Essa nossa
sequência será um vetor com \monoespaco{W\_MAX\_SUBLOOP}
posições. Esta macro poderá ser controlada pelo usuário, mas se não
estiver definida, assumiremos que será 3. Além da pilha, vamos
precisar de uma variável para nos informar em qual profundidade da
pilha estamos no momento (\monoespaco{\_number\_of\_loops}).

A declaração destas duas variáveis será:

\iniciocodigo
@<Weaver Headers@>+=
#if !defined(W_MAX_SUBLOOP)
#define W_MAX_SUBLOOP 3
#endif
int _number_of_loops;
void (*_loop_stack[W_MAX_SUBLOOP]) (void);
@
\fimcodigo

E inicializamos a contagem do número de laços como zero:

\iniciocodigo
@<API Weaver: Inicialização@>+=
_number_of_loops = 0;
@
\fimcodigo

Entrar em um novo laço principal por meio de \monoespaco{Wloop}
envolve verificar se já estamos antes em um laço. Se for o caso,
cancelamos ele. Em seguida, carregamos o novo laço para a pilha e
ajustamos o valor da contagem de laços em execução. Atualizamos o
nosso valor de contagem de tempo e finalmente executamos o laço. Em
ambiente Web Assembly em navegador de Internet isso envolve chamar
diretamente uma função que estabelece nossa função escolhida como laço
principal. Nos demais, basta executar a função em
um \monoespaco{while} comum:

\iniciocodigo
@<API Weaver: Base@>+=
void _Wloop(void (*f)(void)){
  if(_number_of_loops > 0){
    @<Cancela Loop Principal@>
    _number_of_loops --;
  }
  @<Código antes de Loop e Subloop@>
  @<Código andes de Loop, não de Subloop@>
  _loop_stack[_number_of_loops] = f;
  _number_of_loops ++;
#if defined(__EMSCRIPTEN__)
  emscripten_set_main_loop(f, 0, 1);
#else
  while(1)
    f();
#endif
}
@
\fimcodigo

Cancelar um laço principal já existente envolve caso estejamos
executando em ambiente Web Assembly invocar a função que interrompe o
laço atual:

\iniciocodigo
@<Cancela Loop Principal@>=
#if defined(__EMSCRIPTEN__)
emscripten_cancel_main_loop();
#endif
@
\fimcodigo

Iniciar um novo subloop, ou sublaço, é bastante semelhante. Mas
tratamos de maneira diferente o nosso contador de laços, já que ele
realmente precisa ser incrementado. E também temos que checar se
tivemos um estouro na nossa pilha de laços:

\iniciocodigo
@<API Weaver: Definições@>+=
void Wsubloop(void (*f)(void)){
#if defined(__EMSCRIPTEN__)
    emscripten_cancel_main_loop();
#endif
  @<Código antes de Loop e Subloop@>
  @<Código andes de Subloop, não de Loop@>
  if(_number_of_loops >= W_MAX_SUBLOOP){
    fprintf(stderr, "Error: Max number of subloops achieved.\n");
    fprintf(stderr, "Please, increase W_MAX_SUBLOOP in conf/conf.h"
            " to a value bigger than %d.\n", W_MAX_SUBLOOP);
    exit(1);
  }
  _loop_stack[_number_of_loops] = f;
  _number_of_loops ++;
#if defined(__EMSCRIPTEN__)
  emscripten_set_main_loop(f, 0, 1);
#else
  while(1)
    f();
#endif
}
@
\fimcodigo

Vamos incluir o cabçalho para podermos imprimir mensagens de erro:

@<Weaver Headers@>+=
#include <stdio.h>
@

Uma coisa que faremos tanto antes de um laço como de um sublaço é
atualizar a variável \monoespaco{\_running\_loop} que avisa ao motor
que realmente estamos executando um laço ao invés de tentar sair dele,
ajustamos a variável que diz que estamos entrando no começo de um laço
e que portanto precisaremos executar o código de inicialização, a
variável que informa que a finalização do laço ainda não foi executada
e também atualizamos nosso contador de tempo:

\iniciocodigo
@<Código antes de Loop e Subloop@>=
_running_loop = true;
_loop_begin = true;
_loop_finalized = false;
_update_time();
@
\fimcodigo

\subsecao{3.4. Saindo do Laço Principal.}

Assim como existem duas funções para entrar em laços prinicpais,
existitão duas funções para sair. Uma delas apenas sai do laço
prinicpal atual, voltando para o próximo laço principal da pilha se
existit. A outra sai de todos os laços existentes e encerra o
programa.

A que sai de todos os laços já foi parcialmetne definida e é
a \monoespaco{Wexit}.

Sair de um laço atual já existente envolve ajustar a variável global
que diz que estamos executando o laço para um valor que significa que
queremos sair do laço:

\iniciocodigo
@<Weaver Headers@>+=
#define Wexit_loop() (_running_loop = false)
@
\fimcodigo

Se você verificar novamente o código inserido pelas macros presentes
em laços principais, verá que se esta variável for falsa e não existir
nenhum recurso ou arquivo ainda sendo carregado, então será chamada a
função \monoespaco{\_exit\_loop}:

\iniciocodigo
@<Weaver Headers@>+=
#if !defined(_MSC_VER)
void _exit_loop(void) __attribute__ ((noreturn));
#else
__declspec(noreturn) void _exit_loop(void);
#endif
@
\fimcodigo

Já o código desta função envolve cancelar o laço atual e checar se
existe outro na pilha. Se não existir, o programa se encerra. Se
existir, executa código de entrada no novo laço:

\iniciocodigo
@<API Weaver: Definições@>+=
void _exit_loop(void){
  if(_number_of_loops <= 1){
    Wexit();
    exit(1); // Esta linha apenas previne aviso na compilação
  }
  else{
    @<Código após sairmos de Subloop@>
    _number_of_loops --;
    @<Código antes de Loop e Subloop@>
#if defined(__EMSCRIPTEN__)
    emscripten_cancel_main_loop();
    emscripten_set_main_loop(_loop_stack[_number_of_loops - 1], 0, 1);
#else
    while(1)
      _loop_stack[_number_of_loops - 1]();
#endif
  }
}
@
\fimcodigo

\secao{Definições Vazias}

As seguintes definições ainda estão em branco e estão aqui para manter
o projeto compilando com sucesso. Esta parte não existirá na versão
final deste documento e as partes que aparecem aqui vazias estarão
posteriormente preenchidas e melhor explicadas.

\iniciocodigo
@<Código andes de Subloop, não de Loop@>=
@
\fimcodigo

\iniciocodigo
@<Código após sairmos de Subloop@>=
@
\fimcodigo

\iniciocodigo
@<API Weaver: Funções@>=
@
\fimcodigo

\iniciocodigo
@<Código a executar todo loop@>=
@
\fimcodigo

\iniciocodigo
@<Código de renderização@>=
@
\fimcodigo

\iniciocodigo
@<Código andes de Loop, não de Subloop@>=
@
\fimcodigo

\iniciocodigo
@<Weaver Macros@>=
@
\fimcodigo


\secao{Referências}

\referencia{Fiedler, G. (2004) ``Fix Your Timestep!'', acessado em
\monoespaco{https://gafferongames. com/post/fix\_your\_timestep/}
em 2020.}


\fim
