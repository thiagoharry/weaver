\font\sixteen=cmbx16
\font\twelve=cmr12
\font\fonteautor=cmbx12
\font\fonteemail=cmtt10
\font\twelvenegit=cmbxti12
\font\twelvebold=cmbx12
\font\trezebold=cmbx13
\font\twelveit=cmsl12
\font\monodoze=cmtt12
\font\it=cmti12
\voffset=0,959994cm % 3,5cm de margem superior e 2,5cm inferior
\parskip=6pt

\def\titulo#1{{\noindent\sixteen\hbox to\hsize{\hfill#1\hfill}}}
\def\autor#1{{\noindent\fonteautor\hbox to\hsize{\hfill#1\hfill}}}
\def\email#1{{\noindent\fonteemail\hbox to\hsize{\hfill#1\hfill}}}
\def\negrito#1{{\twelvebold#1}}
\def\italico#1{{\twelveit#1}}
\def\monoespaco#1{{\monodoze#1}}
\def\iniciocodigo{\lineskip=0pt\parskip=0pt}
\def\fimcodigo{\twelve\parskip=0pt plus 1pt\lineskip=1pt}

\long\def\abstract#1{\parshape 10 0.8cm 13.4cm 0.8cm 13.4cm
0.8cm 13.4cm 0.8cm 13.4cm 0.8cm 13.4cm 0.8cm 13.4cm 0.8cm 13.4cm
0.8cm 13.4cm 0.8cm 13.4cm 0.8cm 13.4cm
\noindent{{\twelvenegit Abstract: }\twelveit #1}}

\def\resumo#1{\parshape  10 0.8cm 13.4cm 0.8cm 13.4cm
0.8cm 13.4cm 0.8cm 13.4cm 0.8cm 13.4cm 0.8cm 13.4cm 0.8cm 13.4cm
0.8cm 13.4cm 0.8cm 13.4cm 0.8cm 13.4cm
\noindent{{\twelvenegit Resumo: }\twelveit #1}}

\def\secao#1{\vskip12pt\noindent{\trezebold#1}\parshape 1 0cm 15cm}
\def\subsecao#1{\vskip12pt\noindent{\twelvebold#1}}
\def\referencia#1{\vskip6pt\parshape 5 0cm 15cm 0.5cm 14.5cm 0.5cm 14.5cm
0.5cm 14.5cm 0.5cm 14.5cm {\twelve\noindent#1}}

%@* .

\twelve
\vskip12pt
\titulo{Gerenciador de Memória Weaver}
\vskip12pt
\autor{Thiago Leucz Astrizi}
\vskip6pt
\email{thiago@@bitbitbit.com.br}
\vskip6pt

\abstract{This article describes using literary programming a memory
manager written for Weaver Game Engine. It aims to be a very simple
and fast memory manager for programs where memory is allocated and
freed in a stack-based order and we know the maximum ammount of memory
that the program will need. It allows users creating markings during
the program execution that after allows them to free at once all the
memory allocated after the last marking. It also warns users about
memory leaks. After the memory manager creation, we also run some
benchmarks comparing its performance against malloc and free from
GNU's Glibc running at Linux and also running in a web browser using
Web Assembly.}

\vskip 0.5cm plus 3pt minus 3pt

\resumo{Este artigo descreve usando programação literária um
gerenciador de memória escrito para o Motor de Jogos Weaver. Ele é
voltado a programas onde a memória é alocada e liberada seguindo uma
ordem baseada em pilha e a quantidade máxima de memória é
conhecida. Ele fornece a possibilidade de criar marcações durante a
execução do programa de modo que depois possam liberar toda a memória
alocada desde a última marcação. Ele também avisa usuários
sobre vazamentos de memória. Depois da criação do gerenciador de
memória, também comparamos o desempenho dele com o {\it malloc} e {\it
free} da biblioteca C GNU rodando emLinux e em um navegador de
Internet usando {\it Web Assembly}.}

\secao{1. Introdução}

\subsecao{1.1. Gerenciadores de Memória em Motores de Jogos}

Muitos motores de jogos implementam seus próprios gerenciadores de
memória ao invés de utilizar as funções da biblioteca de sistema para
obter memória dinamicamente. Segundo [Gregory 2019], isso ocorre
porque, a implementação de funções como \italico{malloc}
e \italico{free} costumam ser relativamente lentas quando utilizadas
em jogos quando comparados a gerenciadores feitos sob medida, além de
poderem causar fragmentação de memória.

Gregory identifica cinco padrões de projeto comuns utilizados em
gerenciadores de memória de jogos:

\negrito{Alocadores Baseados em Pilha:} Esses alocadores
sempre alocam novas regiões de memória sequencialmente e toda
desalocação deve ser feita em ordem inversa à alocação. A
implementação é bastante simples e busca manter a localidade espacial
das regiões de memória usadas. Toda a memória alocada à partir de
qualquer ponto de execução pode ser desalocada muito rapidamente com
uma única chamada de função.

\negrito{Alocadores de Reservatório:} Essa técnica pode ser usada quando
sabemos que podemos precisar de até $n$ tipos de elementos, todos eles
com o mesmo tamanho. Pode-se então alocar com antecedência o espaço
para todos os elementos e manter ele sempre à disposição usando
variáveis para definir quando o elemento realmente existe e quando
podemos considerar sua posição com livre.

\negrito{Alocações Alinhadas:} Diferentes tipos de dados e arquiteturas podem
ter diferentes restrições de alinhamento de memória alocada. A
plataforma de jogos \italico{Playstation 3}, por exemplo, requer que
qualquer posição de memória a ser passada por meio de DMA
(\italico{Direct Memory Access}) tenha um alinhamento de 128 bits. Ou seja,
o endereço precisa ser um múltiplo de 128 bits.

\negrito{Alocações de Quadro Único e Duplo:} Um motor de jogos divide
sua computação em quadros, sendo que em cada quadro uma nova imagem é
enviada para a tela. Podem existir então variáveis alocadas que devem
ter um tempo de vida de um ou dois quadros somente, podendo ser
desalocadas automaticamente depois disso.

\negrito{Desfragmentação de Memória:} Quando uma memória é alocada e
liberada em ordem qua não pode ser controladas, podem se formar
lacunas entre espaços de memória alocados, as quais podem ser muitas e
ao mesmo tempo com um tamanho pequeno demais para serem reaproveitadas.
Para evitar isso é comum usar ao invés de ponteiros para objetos alocados,
índices de referência para os ponteiros em si. Desta forma, regiões de
memória alocadas podem ser movidas incremental e periodicamente de modo a
evitar a fragmentação.

O objetivo deste artigo será definir um alocador baseado em pilha, o
qual consegue armazenar duas pilhas diferentes na região de memória
gerenciada por ele. Gerenciar qual das pilhas a ser usada será
responsabilidae do usuário. No caso do Motor de Jogos Weaver, a
primeira pilha (que chamamos de ``pilha externa'') será reservada para
alocar memória para elementos criados pelo programador que utiliza o
motor de jogos e a segunda pilha (``pilha interna'') será reservada
para uso interno do motor. Essa técnica é descrita em [Ranck, 2000]
que mostra que ela foi usada no jogo \italico{Hydro Thunder} da
empresa \italico{Midway}.

Alocadores de reservatório não serão tratados aqui, mas podem ser
construídos sobre o alocador presente aqui como uma segunda camada de
gerenciamento. Como o objetivo é fornecer um gerenciador de memória
para rodar em computadores pessoais para a linguagem C e C++, iremos
manter um alinhamento automático de memória sempre igual ao tamanho de
um \monoespaco{long int}. O gerenciador confiará na informação
fornecida pelo compilador para obter o alinhamento ideal. E como
estamos definindo um alocador baseado em pilha, não precisaremos nos
preocupar com desfragmentação.

\subsecao{1.2. Ambientes de Execução}

Neste artigo vamos nos preocupar em garantir que as funções definidas
rodem com sucesso em quatro ambientes diferentes: Windows 10, macOS
Sierra, Linux e Web Assembly. Os três primeiros são os três sistemas
operacionais para computadores pessoais. O último é uma especificação
de máquina virtual especializada em interpretar um subconjunto
otimizado de Javascript, criada para permitir a execução de programas
de computador completos dentro de ambientes como navegadores de
Internet. Seu desenvolvimento partiu do método apresentado por [Zakai,
2011] que ofereceu um modo de compilar código em C e C++ para
Javascript de maneira otimizada.

Para desenvolver de maneira portável nos quatro ambientes, serão
usadas macros condicionais e as diferentes formas de obter memória no
diferentes sistemas será comparada.

\subsecao{1.3. Programação Literária e Notação Usada no Artigo}

Este artigo utiliza a técnica de ``Programação Literária'' para
desenvolver o seu gerenciador de memória. Esta técnica foi apresentada
em [Knuth, 1984] e consiste e uma filosofia de desenvolvimento de
\italico{software} na qual um programador desenvolve um programa escrevendo
e explicando didaticamente o código necessário, se preocupando em
deixar o seu funcionamento claro para as pessoas que lerem a
explicação. Ferramentas automáticas são então utilizadas para extrair
o código existente na expicação, mudar a ordem do código conforme for
mais adequado para o compilador e produzir à partir do código extraído
um programa executável.

Por exemplo, neste artigo serão definidos dois arquivos
diferentes: \monoespaco{memory.c} e \monoespaco{memory.h}, os quais
podem ser inseridos estaticamente em qualquer projeto, ou compilados
como uma biblioteca compartilhada. O que colocaremos
em \monoespaco{memory.h} será:

\iniciocodigo
@(src/memory.h@>=
#ifndef WEAVER_MEMORY_MANAGER
#define WEAVER_MEMORY_MANAGER
#ifdef __cplusplus
extern "C" {
#endif
@<Declarações de Memória@>
#ifdef __cplusplus
}
#endif
#endif
@
\fimcodigo

As duas primeiras linhas assim como a última são macros de segurança
que impedem que as funções e variáveis declaradas ali sejam declaradas
redundantemente se alguém incluir mais de uma vez o arquivo em um
código-fonte. As demais linhas contém macros que checam se estamos
compilando usando C++ ao invés de C. Se for o caso, nós declaramos
todas as funções que existem neste arquivo como funções do tipo C,
para que o compilador C++ saiba que elas não serão modificadas por
meio de sobrecarga de operadores e que por isso não é necessário
armazenar informações adicionais além do nome da função para
reconhecê-la.

No meio do código acima, deixamos indicado em letras vermelhas que
iremos adicionais mais tarde ali um novo trecho de código chamado
``Declarações de Memória'', com a declaraçãol das funções que iremos
usar. Folheando o artigo, você encontrará nas páginas seguintes um
outro trecho de código cujo título não será \monoespaco{memory.h}, mas
sim ``Declarações de Memória''. Será ali que o código que vai nesta
parte do arquivo será encontrado. Pdem existir mais de um trecho de
código com este título. Isso significa que para produzir o código
funcional utilizado pelo compilador, devemos concatenar todos estes
trechos com o mesmo título e colocar na parte indicada deste
arquivo. Isso nos permite colocar a declaração de funções à medida que
formos explicando elas no código, sem precisar declarar todas de uma
vez só porque elas pertencem a um mesmo trecho de código.

\subsecao{1.4. Funções a serem Definidas}

Nosso gerenciador de memória irá definir um total de 6 novas
funções. A primeira recebe um tamanho em bytes e retorna uma região
contínua de memória a ser gerenciada (que chamamos de ``arena''):

\iniciocodigo
@<Declarações de Memória@>=
#include <stdlib.h> // Include 'size_t'
void *Wcreate_arena(size_t size);
@
\fimcodigo

A segunda recebe uma arena criada pela primeira e libera toda a memória
reservada nela, retornando se ela foi bem-sucedida ou não:

\iniciocodigo
@<Declarações de Memória@>+=
#include <stdbool.h> // Include 'bool'
bool Wdestroy_arena(void *);
@
\fimcodigo

A terceira equivale a um \monoespaco{malloc} e recebe uma arena, um
número que indica se queremos alocar da pilha externa (0) ou interna
(1) da arena e um tamanho em bytes. Se uma macro de depuração estiver
definida, ela também deve receber o nome do arquivo em que é chamada e
o número de linha para que quando um vazamento de memória for
detectado, uma mensagem de erro informativa possa ser impressa. Por
fim, definimos também uma macro para preencher automaticamente tais
informações sem que o usuário precise intervir caso o modo de
depuração esteja ativo:

\iniciocodigo
@<Declarações de Memória@>+=
#if W_DEBUG_MEMORY
void *_alloc_(void *arena, int flag,size_t size, char *filename,
              unsigned long line);
#define Walloc(a, b, c) _alloc_(a, b, c, __FILE__, __LINE__)
#else
void *_alloc_(void *arena, int flag, size_t size);
#define Walloc(a, b, c) _alloc_(a, b, c)
#endif
@
\fimcodigo

A quarta função é a equivalente ao \monoespaco{free} e desaloca uma
região previamente alocada com o \monoespaco{Walloc}. Se estivermos em
modo de depuração, ela também deve ser informada do nome do arquivo e
número de linha para imprimir mensagens úteis em caso de depuração:

\iniciocodigo
@<Declarações de Memória@>+=
#if defined(W_DEBUG_MEMORY)
void _free_(void *mem, char *filename, unsigned long line);
#define Wfree(a) _free_(a, __FILE__, __LINE__)
#else
void *_free_(void *mem);
#define Wfree(a) _free_(a)
#endif
@
\fimcodigo

A quinta função serve para colocar uma marcação, a qual chamamos de
``breakpoint'' dentro de uma arena onde alocamos memória. Essa
marcação pode ser usada para determinar que alocações ocorreram antes
e quais ocorreram depois que ela foi feita. A flag indica se essa
marcação deve ser colocada na região de alocação externa (0) ou
interna (1). A função retorna se foi bem-sucedida ou não.

\iniciocodigo
@<Declarações de Memória@>+=
bool Wbreakpoint(void *arena, int regiao);
@
\fimcodigo

A última função usa a marcação criada pela função anterior e desaloca
de uma só vez toda a memória alocada com \monoespaco{Walloc} desde que
a última marcação foi criada. Ela recebe uma flag para saber se isso
deve ser feito na região externa (0) ou interna (1):

\iniciocodigo
@<Declarações de Memória@>+=
void Wtrash(void *arena, int regiao);
@
\fimcodigo

\secao{2. Implementação}

\subsecao{2.1. Obtendo uma Região de Memória Inicial}

Em um gerenciador de memória de propósito geral nós não temos como
saber qual a quantidade máxima de memória que iremos precisar. Mas em
um gerenciador de memória usado em jogos, é essencial que
estabeleçamos um limite máximo de uso de memória. Em tais casos,
estamos menos interessados em obter resultados precisos de computação
e mais interessados em garantir um desempenho contínuo, sem perda de
performance súbita como quando esgotamos a memória principal e temos
que usar memória \italico{swap}. Nestes casos é interessante alocar de
uma só vez a quantidade máxima de memória que nos comprometemos a usar
e gerenciar esta mesma quantidade durante o tempo de execução do
programa.

O modo que utilizaremos para obter essa quantidade de memória inicial
varia dependendo do ambiente de execução. Pode-se usar até mesmo
um \monoespaco{malloc} para fazer isso, embora aqui nós iremos
preferir usar a alternativa que desperdice menos memória possível. Se
possível uma que retorne exatamente o valor que queremos sem gastar
qualquer valor adicional preenchendo informações e estruturas de dado
adicionais.

Tanto em qualquer sistema Unix (Linux, OpenBSD, macOS) como quando
compilamos para WebAssembly com o compilador Emscripten, a escolha
mais econômica é usar o \monoespaco{mmap}. Ee é uma função bastante
ampla que permite mapear para a memória qualquer coisa, de arquivos em
disco até simplesmente alocar uma região nova de memória para um
programa.

Para podermos usar \monoespaco{mmap}, só temos que inserir o cabeçalho
adequado:

\iniciocodigo
@<Incluir Cabeçalhos Necessários@>=
#if defined(__EMSCRIPTEN__) || defined(__unix__) || defined(__APPLE__)
#include <sys/mman.h>
#endif
@
\fimcodigo

Tendo definido o cabeçalho, para usar a função para alocar uma região
de tamanho $M$ que não está associada a nenhum arquivo, somente à
memória física e que pode tanto ser lida como escrita, invocamos ela
da seguinte forma:

\iniciocodigo
@<Alocar em `arena' região de `M' bytes@>=
#if defined(__EMSCRIPTEN__) || defined(__unix__) || defined(__APPLE__)
arena = mmap(NULL, M, PROT_READ|PROT_WRITE, MAP_PRIVATE|MAP_ANON,
             -1, 0);
#endif
@
\fimcodigo

Os argumentos que foram setados para nulo, zero ou -1 são apenas
argumentos que não são necessários no modo que estamos usando
o \monoespaco{mmap}.

Assim como nós criamos uma nova região de memória para usarmos, depois
vai ser necessário desfazer ela. Neste caso, usamos
o \monoespaco{munmap}:

\iniciocodigo
@<Desalocar `arena' de tamanho `M' bytes@>=
#if defined(__EMSCRIPTEN__) || defined(__unix__) || defined(__APPLE__)
munmap(arena, M);
#endif
@
\fimcodigo

Em ambientes Windows, a função equivalente a isso é a
a \monoespaco{CreateFileMapping}, com a diferença de que ela retorna
um controlador que precisa de uma função adicional para por fim obter
um ponteiro para a região alocada. Felizmente, segundo a documentação
da API do Windows, é permitido fechar tal controlador com
o \monoespaco{CloseHandle} antes de desalocar e desfazer a região
para a qual o ponteiro aponta. Graças à isso, conseguimos manter a
simetria entre o código Windows e das demais plataformas, pois como o
controlador será fechado aqui, não precisamos memorizá-lo com uma
estrutura adicional a ser encerrada futuramente.

\iniciocodigo
@<Alocar em `arena' região de `M' bytes@>+=
#if defined(_WIN32)
{
  HANDLE handle;
  handle = CreateFileMappingA(INVALID_HANDLE_VALUE, NULL,
                              PAGE_READWRITE,
                              (DWORD) ((DWORDLONG) M) / ((DWORDLONG) 4294967296),
                              (DWORD) ((DWORDLONG) M) % ((DWORDLONG) 4294967296),
                              NULL);
  arena = MapViewOfFile(handle, FILE_MAP_READ | FILE_MAP_WRITE, 0, 0, 0);
  CloseHandle(handle);
}
#endif
@
\fimcodigo

Para desalocar a região alocada usaremos o \monoespaco{UnmapViewOfFile}:

\iniciocodigo
@<Desalocar 'arena' de tamanho `M' bytes@>+=
#if defined(_WIN32)
UnmapViewOfFile(arena);
#endif
@
\fimcodigo

Usar as funções acima requer os seguintes cabeçalhos:

\iniciocodigo
@<Incluir Cabeçalhos Necessários@>+=
#if defined(_WIN32)
#include <windows.h>  // Include 'CreateFileMapping', 'MapViewOfFIle',
#include <memoryapi.h> // 'UnmapViewOfFile', 'CloseHandle'
#endif
@
\fimcodigo

\subsecao{2.2. Obtendo o Tamanho da Página}

Em um computador real, em contraste com uma máquina virtual, quando um
programa pede memória para o Sistema Operacional, ele sempre irá
receber memória em múltiplos do tamanho da página usada internamente
pela máquina. Tipicamente o tamanho de uma página é de 4 KiB. Sendo
assim, não adianta pedirmos quantidade de memória que não seja
múltiplo de 4 KiB. Se pedirmos apenas 2 KiB, continuaremos recebendo 4
KiB. Se pedirmos 5 KiB, receberemos 8 KiB.

É importante então que nestes ambientes, nosso gerenciador esteja
ciente disso e que mesmo que um usuário peça uma quantidade de memória
que não seja múltipla do tamanho da página, ele sempe irá ajustar o
pedido para um valor múltiplo para assim não desperdiçar memória.

Na maioria dos sistemas Unix, desde que compatíveis com o POSIX,
podemos obter facilmente o tamanho de uma página por meio da
função \monoespaco{sysconf}:

\iniciocodigo
@<Obter tamanho de página `p'@>=
#if defined(__unix__)
p = sysconf(_SC_PAGESIZE);
#endif
@
\fimcodigo

A documentação do macOS afirma que ele possui tal função que é
compatível com o POSIX. Contudo, a documentação não menciona a opção
de obter o tamanho da página por ela. Ao invés disso, a documentação
menciona usar a função BSD \monoespaco{getpagesize} para obter tal
informação. Para nos mantermos seguindo a documentação, faremos isso
então:

\iniciocodigo
@<Obter tamanho de página `p'@>+=
#if defined(__APPLE__)
p = getpagesize();
#endif
@
\fimcodigo

Tanto a função BSD acima como a função POSIX estão definidas no mesmo
cabeçalho:

\iniciocodigo
@<Incluir Cabeçalhos Necessários@>+=
#if defined(__APPLE__) || defined(__unix__)
#include <unistd.h>
#endif
@
\fimcodigo

No Windows, o modo de obter o tamanho da página é por meio da função
mais complexa \monoespaco{GetSystemInfo} que retorna também uma série
de informações adicionais sobre o sistema que não precisaremos usar no
momento.

\iniciocodigo
@<Obter tamanho de página `p'@>+=
#if defined(_WIN32)
{
  SYSTEM_INFO info;
  GetSystemInfo(&info);
  p = info.dwPageSize;
}
#endif
@
\fimcodigo

E para usar esta função, a documentação nos aconselha incluir
diretamente o cabeçalho \monoespaco{windows.h}:

\iniciocodigo
@<Incluir Cabeçalhos Necessários@>+=
#if defined(_WIN32)
#include <windows.h> // Include 'GetSystemInfo'
#endif
@
\fimcodigo


Por fim, o caso do ambiente WebAssembly. Neste ambiente, a memória
dinâmica é obtida por meio de uma seção de memória linear que começa
com um tamaho padrão e pode crescer por meio de um
operador \monoespaco{grow\_memory}, o qual também é configurado com um
tamanho máximo. De qualquer forma, aqui também a quantidade de memória
alocada é múltipla de uma página. O tamanho da página na máquina
virtual WebAssembly é documentado como sendo sempre o mesmo:

\iniciocodigo
@<Obter tamanho de página `p'@>+=
#if defined(__EMSCRIPTEN__)
p = 64 * 1024; // 64 KiB
#endif
@
\fimcodigo

\subsecao{2.3. Sobre Execução em Diferentes Threads}

É importante garantir que o código definido aqui não deixe de
funcionar quando invocado simultaneamente por mais de um trecho de
código. Para isso, iremos apenas garantir um \italico{mutex} para que
o nosso gerenciador de memória não tenha a mesma região de memória
sendo acessada por mais de uma \italico{thread}. Caso tenhamos um caso
no qual várias threads precise alocar memória simultaneamente, pode
ser mais vantajoso dar para cada uma delas a sua própria região de
memória para evitar que cada uma delas bloqueie todas as outras
durante sua alocação.

Nem todos os ambientes suportam threads. A máquina virtual Web
Assembly, ao menos a versão implementada nos navegadores de Internet
não oferece suporte à elas, exceto em versões experimentais. Sendo
assim, definiremos código para nosos mutex somente para os demais
ambientes. No caso do Windows, iremos preferir usar a API para
``Seções Críticas'' ao invés de usar diretamente um mutex. O principal
motivo é evitar uma chamada de sistema para o Kernel caso ninguém
esteja usando nosso Mutex. As seções críticas conseguem isso apenas
trazendo a restrição de não poderem ser compartilhadas com outros
programas. No mais, as seções críticas funcionam de maneira análoga a
um mutex.

Em sistemas baseados em Unix, incluimos o cabeçalho da biblioteca
POSIX ``pthreads''. No Windows, os cabeçalhos relevantes já foram
incluídos no \monoespaco{windows.h}.

\iniciocodigo
@<Incluir Cabeçalhos Necessários@>+=
#if defined(__unix__) || defined(__APPLE__)
#include <pthread.h>
#endif
@
\fimcodigo

Um mutex é declarado da seguinte forma:

\iniciocodigo
@<Declaração de Mutex@>=
#if defined(__unix__) || defined(__APPLE__)
pthread_mutex_t mutex;
#endif
#if defined(_WIN32)
CRITICAL_SECTION mutex;
#endif
@
\fimcodigo

Quando o mutex é inicializado, colocamos em uma variável booleana
chamada \monoespaco{error} se ocorreu algum problema. No caso da
biblioteca \italico{pthreads}, sua função de inicialização já retorna
valor não-nulo se ocorreu um problema. No Windows, o sistema garante
que a função de inicialização nunca falhe. Nos dois casos assumimos
que temos um ponteiro genérico para o nosso mutex.

\iniciocodigo
@<Inicialização de `*mutex'@>=
#if defined(__unix__) || defined(__APPLE__)
error = pthread_mutex_init((pthread_mutex_t *) mutex, NULL);
#endif
#if defined(_WIN32)
InitializeCriticalSection((CRITICAL_SECTION *) mutex);
#endif
@
\fimcodigo

O código para destruir um mutex é o exposto abaixo. Neste caso não
iremos nos preocupar com casos de erro.

\iniciocodigo
@<Finaliza `*mutex'@>=
#if defined(__unix__) || defined(__APPLE__)
pthread_mutex_destroy((pthread_mutex_t *) mutex);
#endif
#if defined(_WIN32)
DeleteCriticalSection((CRITICAL_SECTION *) mutex);
#endif
@
\fimcodigo

A operação clássica de \italico{wait} para requerer o uso do mutex:

\iniciocodigo
@<`*mutex':WAIT()@>=
#if defined(__unix__) || defined(__APPLE__)
pthread_mutex_lock((pthread_mutex_t *) mutex);
#endif
#if defined(_WIN32)
EnterCriticalSection((CRITICAL_SECTION *) mutex);
#endif
@
\fimcodigo

E a função de \italico{signal} para liberar um mutex que foi pedido
pela thread:

\iniciocodigo
@<`*mutex':SIGNAL()@>=
#if defined(__unix__) || defined(__APPLE__)
pthread_mutex_unlock((pthread_mutex_t *) mutex);
#endif
#if defined(_WIN32)
LeaveCriticalSection((CRITICAL_SECTION *) mutex);
#endif
@
\fimcodigo

\subsecao{2.3. Cabeçalho de Arenas de Memória}

Logo após alocarmos uma região de memória (ou ``arena''), a primeira
coisa a fazer é reservar os seus bytes iniciais para armazenar
informações gerais sobre ela. As informações que desejamos são: o
tamanho total de espaço que temos (\monoespaco{remaining\_space}),
quanto de espaço realmente foi alocado com ela
(\monoespaco{total\_size}), um ponteiro para o começo de uma região
livre na pilha de memória interna e externa que iremos alocar
(\monoespaco{internal\_free}, \monoespaco{external\_free}) e um
ponteiro para o último elemento que alocamos em cada uma das pilhas
(\monoespaco{internal\_last}, \monoespaco{external\_last}).

Outra coisa que também deixaremos no cabeçalho é um \italico{array} de
elementos de tamanho igual a \monoespaco{unsigned long} reservados
para alocar quantidades pequenas de memória. Para alocações minúsculas
de curta duração não valerá à pena gastar espaço da região livre,
tendo que armazenar para cada uma delas um cabeçalho de
alocação. Iremos alocar em tais casos, sempre que possível, um espaço
deste \monoespaco{array} do cabeçalho. E também precisaremos de um
índice para a próxima região do \italico{array} para a região interna
e para a externa. Ambas as regiões compartilharão o
mesmo \italico{array}, mas uma começará ocupando as primeiras posições
e a outra as últimas posições.

Precisaremos de um mutex como o que definimos para proteger a arena de
ser manipulada simultaneamente por duas threads.

Outra informação que podemos precisar só em momentos de depuração:
qual a quantidade mínima de memória que chegamos a ter ao longo do
tempo de vida da arena? Isso é útil de armazenar em tais momentos
porque após encerrarmos a arena e descobrirmos que nós usamos muito
pouco dela, podemos ter interesse em diminuir o tamanho que demos a
ela, já que nossa aplicação não está precisando de tanta memória
quanto reservamos. Usaremos a
variável \monoespaco{smallest\_remaining\_space} para armazenar isso e
ela só estará definida se a macro \monoespaco{W\_DEBUG\_MEMORY}
estiver definida.

O cabeçalho de nossa arena de memória terá então a seguinte forma:

\iniciocodigo
@<Cabeçalho da Arena@>=
struct arena_header{
  @<Declaração de Mutex@>
  void *internal_free, *external_free;
  void *internal_last, *external_last;
  unsigned long array[1024];
  int internal_index, external_index;
  size_t total_size, remaining_space;
#if defined(W_DEBUG_MEMORY)
  size_t smallest_remaining_space;
#endif
};
@
\fimcodigo

Assumimos que nós temos um ponteiro para a arena de memória
recém-obtida, o qual chamamos de \monoespaco{arena} e que nós temos o
tamanho total em bytes que está alocado nela na
variável \monoespaco{M}, então conseguimos inicializar o cabeçalho ali
com o código abaixo. Para calcular as próximas posições onde iremos
inserir, convertemos o ponteiro para o começo da arena para um
ponteiro de caractere para podermos fazer nossos cálculos com
aritmética de ponteiro com múltiplos de 1 byte. O próximo espaço que
temos livre para a pilha de memória externa é o byte imediatamente
após o cabeçalho. Já a pilha interna fica com o último byte da arena.

\iniciocodigo
@<Inicializa cabeçalho em `arena' de tamanho `M'@>=
{
  struct arena_header *header = (struct arena_header *) arena;
  header -> internal_free = ((char *) header) + M - 1;
  header -> external_free = ((char *) header) + sizeof(struct arena_header);
  header -> internal_last = header -> external_last = NULL;
  header -> internal_index = 1023;
  header -> external_index = 0;
  header -> remaining_space = M - sizeof(struct arena_header);
  header -> total_size = M;
#if defined(W_DEBUG_MEMORY)
  header ->  smallest_remaining_space = header -> remaining_space;
#endif
  { // Mutex initialization
    void *mutex = &(header -> mutex);
    @<Inicialização de `*mutex'@>
  }
}
@
\fimcodigo

\subsecao{2.4. O Código de Alocação de Nova Arena}

Com o código que definirmos já temos como definir a
função \monoespaco{Wcreate\_arena}. O que a função terá que fazer
são as 6 operações abaixo:

1. Receber do usuário um tamanho \monoespaco{t} para alocar.

2. Descobrir qual o tamanho da página \monoespaco{p} no sistema atual.

3. Obter \monoespaco{M}, sendo igual ao menor múltiplo
de \monoespaco{p} maior que \monoespaco{t}. Se \monoespaco{M} for
menor que o espaço necessário para o cabeçalho da arena, ele se torna
igual ao menor múltiplo de \monoespaco{p} que é maior que o tamanho do
cabeçalho.

4. Alocamos uma nova arena de tamanho \monoespaco{M}.

5. Inicializamos o seu cabeçalho.

6. Retornamos o ponteiro para o começo da arena, onde está seu cabeçalho,
ou \monoespaco{NULL} em caso de problemas.

O código para fazer isso será então:

\iniciocodigo
@<Definição de `Wcreate\_arena'@>=
void *Wcreate_arena(size_t t){
  bool error = false;
  void *arena;
  size_t p, M, header_size = sizeof(struct arena_header);
  // Operation 2:
  @<Obter tamanho de página `p'@>
  // Operation 3:
  M = (((t - 1) / p) + 1) * p;
  if(M < header_size)
    M = (((header_size - 1) / p) + 1) * p;
  // Operation 4:
  @<Alocar em `arena' região de `M' bytes@>
  // Operation 5:
  @<Inicializa cabeçalho em `arena' de tamanho `M'@>
  // Operation 6:
  if(error) return NULL;
  return arena;
}
@
\fimcodigo

\subsecao{2.5. O Cabeçalho de uma Nova Alocação}

Quando alocamos uma nova região de memória e retornamos seu endereço
para o usuário, nós precisamos adicionar nela informação adicional que
nos permite gerenciá-la. Mais tarde, quando precisarmos desalocá-la
podemos ler tais informações para saber o quanto vamos liberar de
memória e o que fazer com o espaço.

Contudo, nosso gerenciador não armazenará apenas alocações. Como
mencionamos, iremos armazenar também marcações que nos permitem
desalocar muitas regiões de uma só vez. Então a primeira coisa que
vamos querer colocar é uma variável chamada \monoespaco{type} que
indica que tipo de elemento está posicionado ali. No momento definiremos
somente dois tipos de elementos: memória alocada e buracos (memória
anteriormente alocada, mas que foi desalocada):

\iniciocodigo
@<Definições de Tipos@>=
#define ALLOCATION 0x01010101
#define HOLE       0x02020202
@
\fimcodigo

Usamos um padrão tão curioso de números ao invés de usar simplesmente 1 e 2
porque isso pode permitir detectar mais facilmente ocorrências de
\italico{buffer overflow} caso tenhamos interesse em fazer tais testes de
sanidade.

Como queremos que todos os elementos formem uma lista encadeada para
podermos percorrer eles mais rapidamente, precisamos também de um
ponteiro para o elemento anterior (\monoespaco{last\_element}). E
queremos também um ponteiro para a própria arena de memória à qual
pertence aquela alocação (\monoespaco{arena}).

É importante que todos os elementos mencionados acima venham primeiro
e estejam na mesma ordem, independente do tipo de elemento armazenado
pelo alocador. Assim podemos percorrer todos eles, verificar seu tipo
e voltar para o começo da arena de memória sem nos preocupar em
tratá-los de forma separada.

Come relação às variáveis que são específicas à alocações de memória:

\monoespaco{size}: Armazena o tamanho em bytes da região alocada,
incluindo o cabeçalho.

\monoespaco{file}: Opcional. Só será armazenado se o gerenciador for
compilado em modo de depuração ou modo de checar por vazamentos de
memória. É o nome do arquivo em que a função está sendo invocada.

\monoespaco{file}: Opcional. Só será armazenado se o gerenciador for
compilado em modo de depuração ou modo de checar por vazamentos de
memória. É o número da linha em que a função está sendo chamada.

Com isso terminamos de descrever as variáveis que precisamos e
mostramos o código de nosso cabeçalho de alocação:

\iniciocodigo
@<Cabeçalho da Arena@>=
struct alloc_header{
  unsigned long type;
  void *last_element;
  struct arena_header *arena;
  /////////////////////////
  size_t size;
#if defined(W_DEBUG_MEMORY)
  char file[32];
  unsigned long line;
#endif
};
@
\fimcodigo



\subsecao{2.?. Organização Final do Arquivo-Fonte}

Salvaremos todo o código de definição de funções que fizemos no
arquivo abaixo que poderá então ser compilado:

\iniciocodigo
@(src/memory.c@>=
@<Incluir Cabeçalhos Necessários@>
#include "memory.h"
@<Definições de Tipos@>
@<Cabeçalho da Arena@>
@<Definição de `Wcreate\_arena'@>
@
\fimcodigo


\secao{Referências}

\referencia{Gregory, J. (2019) ``Game Engine Architecture'', CRC Press, terceira
edição.}

\referencia{Zakai, A. (2011) ``Emscripten: an LLVM-to-JavaScript compiler'',
Proceedings of the ACM international conference companion on Object
oriented programming systems languages and applications companion,
p. 301--312.}

\referencia{Knuth, D. E. (1984) ``Literate Programming'', The Computer Journal,
volume 27, edição 2, p. 97--111}

\referencia{Ranck, S. (2000) ``Game Programming Gems'', Charles River Media,
volume 1, edição 1, p. 92--100}


\fim
