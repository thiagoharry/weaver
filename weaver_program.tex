\font\sixteen=cmbx16
\font\twelve=cmr12
\font\fonteautor=cmbx12
\font\fonteemail=cmtt10
\font\twelvenegit=cmbxti12
\font\twelvebold=cmbx12
\font\trezebold=cmbx13
\font\twelveit=cmsl12
\font\monodoze=cmtt12
\font\it=cmti12
\voffset=0,959994cm % 3,5cm de margem superior e 2,5cm inferior
\parskip=6pt

\def\titulo#1{{\noindent\sixteen\hbox to\hsize{\hfill#1\hfill}}}
\def\autor#1{{\noindent\fonteautor\hbox to\hsize{\hfill#1\hfill}}}
\def\email#1{{\noindent\fonteemail\hbox to\hsize{\hfill#1\hfill}}}
\def\negrito#1{{\twelvebold#1}}
\def\italico#1{{\twelveit#1}}
\def\monoespaco#1{{\monodoze#1}}
\def\iniciocodigo{\lineskip=0pt\parskip=0pt}
\def\fimcodigo{\twelve\parskip=0pt plus 1pt\lineskip=1pt}

\long\def\abstract#1{\parshape 10 0.8cm 13.4cm 0.8cm 13.4cm
0.8cm 13.4cm 0.8cm 13.4cm 0.8cm 13.4cm 0.8cm 13.4cm 0.8cm 13.4cm
0.8cm 13.4cm 0.8cm 13.4cm 0.8cm 13.4cm
\noindent{{\twelvenegit Abstract: }\twelveit #1}}

\def\resumo#1{\parshape  10 0.8cm 13.4cm 0.8cm 13.4cm
0.8cm 13.4cm 0.8cm 13.4cm 0.8cm 13.4cm 0.8cm 13.4cm 0.8cm 13.4cm
0.8cm 13.4cm 0.8cm 13.4cm 0.8cm 13.4cm
\noindent{{\twelvenegit Resumo: }\twelveit #1}}

\def\secao#1{\vskip12pt\noindent{\trezebold#1}\parshape 1 0cm 15cm}
\def\subsecao#1{\vskip12pt\noindent{\twelvebold#1}}
\def\referencia#1{\vskip6pt\parshape 5 0cm 15cm 0.5cm 14.5cm 0.5cm 14.5cm
0.5cm 14.5cm 0.5cm 14.5cm {\twelve\noindent#1}}

%@* .

\twelve
\vskip12pt
\titulo{O Programa Weaver}
\vskip12pt
\autor{Thiago Leucz Astrizi}
\vskip6pt
\email{thiago@@bitbitbit.com.br}
\vskip6pt

\abstract{This article describes using literary programming the
  program Weaver. This program is a project manager for the Weaver
  Game Engine. If a user wants to create a new game with the Weaver
  Game Engine, they use this program to create the directory structure
  for a new game project. They also use this program to add new source
  files and shader files to a game project. And to update a project
  with a more recent Weaver version installed in the computer. The
  presenting code in C is cross-platform and should work under
  Windows, Linux, OpenBSD and possibly other Unix variants.}

\vskip 0.5cm plus 3pt minus 3pt

\resumo{Este artigo descreve usando programação literária o programa
  Weaver. Este programa é um gerenciador de projetos para o Motor de
  Jogos Weaver. Se alguém deseja criar um novo projeto com o motor de
  jogos, usará este programa para criar a estrutura de diretórios
  desejada. Também usará o programa para adicionar novos arquivos de
  código-fonte e shaders. Para atualizar um projeto pré-existente com
  uma nova versão de Weaver, o programa também é necessário. O código
  seguinte em C será multi-plataforma e deverá funcionar em Windows,
  Linux, OpenBSD e possivelmente outras variantes de Unix.}

\secao{1. Introdução}

Um motor de jogos é formado por um conjunto de bibliotecas e funções
que auxiliam na criação de jogos fornecendo as funcionalidades mais
comus para este tipo de desenvolvimento. Mas além das bibliotecas e
funções, deve existir um gerenciador responsável por fazer com que o
seu código utilize as bibliotecas a maneira adequada e faça as
inicializações necessárias.

O motor de jogos Weaver tem pré-requisitos bastante estritos de como o
diretório que contém um projeto Weaver deve estar organizado. É para
cumprir erstes requisitos que o programa que será apresentado é
necessário. Ele inicializa da maneira correta a estrutura de
diretórios de um novo projeto. Ele adiciona novos arquivos fonte já
com quaisquer código necessário para sua integração. E por controlar o
projeto desta forma, ele saberá atualizar as bibliotecas para versões
mais recentes se necessário.

O uso deste programa será por mieo de linha de comando. Por exemplo,
se um usuário usar o comando ``\monoespaco{weaver pong}'', será criada
uma estrutura de diretórios semelhante à mostrada na imagem que
ilustra o fim da seção com um novo projeto chamado ``pong''.

\imagem{cweb/diagrams/project_dir.eps}

As seguintes seções do artigo estão organizadas da seguinte forma. A
seção 2 abordará a licensa do software. A seção 3 listará as variáveis
usadas para controlar seu comportamento. A seção 4 trará algumas
macros e quais as bibliotecas incluídas. A seção 5 listará todos os
casos de uso e como implementá-los.

\secao{2. Copyright e licenciamento}

Segue abaixo a licença do programa e sua tradução não-oficial:

\espaco{5mm}\linha
\alinhaverbatim
Copyright (c) Thiago Leucz Astrizi 2015

This program is free software: you can redistribute it and/or
modify it under the terms of the GNU Affero General Public License as
published by the Free Software Foundation, either version 3 of
the License, or (at your option) any later version.

This program is distributed in the hope that it will be useful,
but WITHOUT ANY WARRANTY; without even the implied warranty of
MERCHANTABILITY or FITNESS FOR A PARTICULAR PURPOSE.  See the
GNU Affero General Public License for more details.

You should have received a copy of the GNU Affero General Public
License along with this program.  If not, see
<http://www.gnu.org/licenses/>.

\linha

Copyright (c) Thiago Leucz Astrizi 2015

Este programa é um software livre; você pode redistribuí-lo e/ou
modificá-lo dentro dos termos da Licença Pública Geral GNU Affero como
publicada pela Fundação do Software Livre (FSF); na versão 3 da
Licença, ou (na sua opinião) qualquer versão.

Este programa é distribuído na esperança de que possa ser útil,
mas SEM NENHUMA GARANTIA; sem uma garantia implícita de ADEQUAÇÃO
a qualquer MERCADO ou APLICAÇÃO EM PARTICULAR. Veja a
Licença Pública Geral GNU Affero para maiores detalhes.

Você deve ter recebido uma cópia da Licença Pública Geral GNU Affero
junto com este programa. Se não, veja
<http://www.gnu.org/licenses/>.
\alinhanormal
\linha\espaco{5mm}

A versão completa da licença pode ser obtida junto ao código-fonte
Weaver ou consultada no link mencionado.

\secao{Variáveis e Estrutura do Programa Weaver}

O comportamento de Weaver deve depender das seguintes variáveis:

|inside_weaver_directory|: Indicará se o programa está sendo
  invocado de dentro de um projeto Weaver.

|argument|: O primeiro argumento, ou NULL se ele não existir

|argument2|: O segundo argumento, ou NULL se não existir.

|project_version_major|: Se estamos em um projeto Weaver, qual o
  maior número da versão do Weaver usada para gerar o
  projeto. Exemplo: se a versão for 0.5, o número maior é 0. Em
  versões de teste, o valor é sempre 0.

|project_version_minor|: Se estamos em um projeto Weaver, o valor
  do menor número da versão do Weaver usada para gerar o
  projeto. Exemplo, se a versão for 0.5, o número menor é 5. Em
  versões de teste o valor é sempre 0.

|weaver_version_major|: O número maior da versão do Weaver sendo
  usada no momento.

|weaver_version_minor|: O número menor da versão do Weaver sendo
  usada no momento.

|arg_is_path|: Se o primeiro argumento é ou não um caminho
  absoluto ou relativo para um projeto Weaver.

|arg_is_valid_project|: Se o argumento passado seria válido como
  nome de projeto Weaver.

|arg_is_valid_module|: Se o argumento passado seria válido como
  um novo módulo no projeto Weaver atual.

|arg_is_valid_plugin|: Se o segundo argumento existe e se ele é um
 nome válido para um novo plugin.

|arg_is_valid_function|: Se o segundo argumento existe e se ele seria
 um nome válido para um loop principal e também para um arquivo.

|project_path|: Se estamos dentro de um diretório de projeto
  Weaver, qual o caminho para a sua base (onde há o Makefile)

|have_arg|: Se o programa é invocado com argumento.

|shared_dir|: Deverá armazenar o caminho para o diretório onde
  estão os arquivos compartilhados da instalação de Weaver. Por
  padrão, será igual à "\monoespaco{/usr/local/share/weaver}", mas caso exista a
  variável de ambiente \monoespaco{WEAVER\_DIR}, então este será
  considerado o endereço dos arquivos compartilhados.

|author_name|,|project_name| e |year|: Conterão respectivamente o
  nome do usuário que está invocando Weaver, o nome do projeto atual
  (se estivermos no diretório de um) e o ano atual. Isso será
  importante para gerar as mensagens de Copyright em novos projetos
  Weaver.

|return_value|: Que valor o programa deve retornar caso o programa
  seja interrompido no momento atual.

A estrutura geral do programa com a declaração de todas as variáveis
será:

\iniciocodigo
@(src/weaver.c@>=
@<Cabeçalhos Incluídos no Programa Weaver@>
@<Macros do Programa Weaver@>
@<Funções auxiliares Weaver@>
int main(int argc, char **argv){@/
  int return_value = 0; /* Valor de retorno. */
  bool inside_weaver_directory = false, arg_is_path = false,
    arg_is_valid_project = false, arg_is_valid_module = false,
    have_arg = false, arg_is_valid_plugin = false,
    arg_is_valid_function = false; /* Variáveis booleanas. */
  unsigned int project_version_major = 0, project_version_minor = 0,
    weaver_version_major = 0, weaver_version_minor = 0,
    year = 0;
  /* Strings UTF-8: */
  char *argument = NULL, *project_path = NULL, *shared_dir = NULL,
    *author_name = NULL, *project_name = NULL, *argument2 = NULL;
  @<Inicialização@>
  @<Caso de uso 1: Imprimir ajuda (criar projeto)@>
  @<Caso de uso 2: Imprimir ajuda de gerenciamento@>
  @<Caso de uso 3: Mostrar versão@>
  @<Caso de uso 4: Atualizar projeto Weaver@>
  @<Caso de uso 5: Criar novo módulo@>
  @<Caso de uso 6: Criar novo projeto@>
  @<Caso de uso 7: Criar novo plugin@>
  @<Caso de uso 8: Criar novo shader@>
  @<Caso de uso 9: Criar novo loop principal@>
END_OF_PROGRAM:
  @<Finalização@>
  return return_value;
}
@
\fimcodigo

\secao{4. Macros e Cabeçalhos do Programa Weaver}

O programa precisará de algumas macros. A primeira delas deverá conter
uma string com a versão do programa. A versão pode ser formada só por
letras (no caso de versões de teste) ou por um número seguido de um
ponto e de outro número (sem espaços) no caso de uma versão final do
programa.

Para a segunda macro, observe que na estrutura geral do programa vista
acima existe um rótulo chamado |END_OF_PROGRAM| logo na parte de
finalização. Uma das formas de chegarmos lá é por meio da execução
normal do programa, caso nada dê errado. Entretanto, no caso de um
erro, nós podemos também chegar lá por meio de um desvio incondicional
após imprimirmos a mensagem de erro e ajustarmos o valor de retorno do
programa. A responsabilidade de fazer isso será da segunda macro.

Por outro lado, podemos também querer encerrar o programa previamente,
mas sem que tenha havido um erro. A responsabilidade disso é da
terceira macro que definimos.

\iniciocodigo
@<Macros do Programa Weaver@>=
#define VERSION "Alpha"
#define ERROR() {perror(NULL); return_value = 1; goto END_OF_PROGRAM;}
#define END() goto END_OF_PROGRAM;
@
\fimcodigo

Vamos também mostrar agora os cabeçahos incluídos:

\iniciocodigo
@<Cabeçalhos Incluídos no Programa Weaver@>=
#include <sys/types.h> // stat, getuid, getpwuid, mkdir
#include <sys/stat.h> // stat, mkdir
#include <stdbool.h> // bool, true, false
#include <unistd.h> // get_current_dir_name, getcwd, stat, chdir, getuid
#include <string.h> // strcmp, strcat, strcpy, strncmp
#include <stdlib.h> // free, exit, getenv
#include <dirent.h> // readdir, opendir, closedir
#include <libgen.h> // basename
#include <stdarg.h> // va_start, va_arg
#include <stdio.h> // printf, fprintf, fopen, fclose, fgets, fgetc, perror
#include <ctype.h> // isanum
#include <time.h> // localtime, time
#include <pwd.h> // getpwuid
@
\fimcodigo

\secao{5. Casos de Uso do Programa Weaver}

\fim
