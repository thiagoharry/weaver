\font\sixteen=cmbx16
\font\twelve=cmr12
\font\fonteautor=cmbx12
\font\fonteemail=cmtt10
\font\twelvenegit=cmbxti12
\font\twelvebold=cmbx12
\font\trezebold=cmbx13
\font\twelveit=cmsl12
\font\monodoze=cmtt12
\font\it=cmti12
\voffset=0,959994cm % 3,5cm de margem superior e 2,5cm inferior
\parskip=6pt

\def\titulo#1{{\noindent\sixteen\hbox to\hsize{\hfill#1\hfill}}}
\def\autor#1{{\noindent\fonteautor\hbox to\hsize{\hfill#1\hfill}}}
\def\email#1{{\noindent\fonteemail\hbox to\hsize{\hfill#1\hfill}}}
\def\negrito#1{{\twelvebold#1}}
\def\italico#1{{\twelveit#1}}
\def\monoespaco#1{{\monodoze#1}}
\def\iniciocodigo{\lineskip=0pt\parskip=0pt}
\def\fimcodigo{\twelve\parskip=0pt plus 1pt\lineskip=1pt}

\long\def\abstract#1{\parshape 10 0.8cm 13.4cm 0.8cm 13.4cm
0.8cm 13.4cm 0.8cm 13.4cm 0.8cm 13.4cm 0.8cm 13.4cm 0.8cm 13.4cm
0.8cm 13.4cm 0.8cm 13.4cm 0.8cm 13.4cm
\noindent{{\twelvenegit Abstract: }\twelveit #1}}

\def\resumo#1{\parshape  10 0.8cm 13.4cm 0.8cm 13.4cm
0.8cm 13.4cm 0.8cm 13.4cm 0.8cm 13.4cm 0.8cm 13.4cm 0.8cm 13.4cm
0.8cm 13.4cm 0.8cm 13.4cm 0.8cm 13.4cm
\noindent{{\twelvenegit Resumo: }\twelveit #1}}

\def\secao#1{\vskip12pt\noindent{\trezebold#1}\parshape 1 0cm 15cm}
\def\subsecao#1{\vskip12pt\noindent{\twelvebold#1}}
\def\referencia#1{\vskip6pt\parshape 5 0cm 15cm 0.5cm 14.5cm 0.5cm 14.5cm
0.5cm 14.5cm 0.5cm 14.5cm {\twelve\noindent#1}}

%@* .

\twelve
\vskip12pt
\titulo{O Programa Weaver}
\vskip12pt
\autor{Thiago Leucz Astrizi}
\vskip6pt
\email{thiago@@bitbitbit.com.br}
\vskip6pt

\abstract{This article describes using literary programming the
  program Weaver. This program is a project manager for the Weaver
  Game Engine. If a user wants to create a new game with the Weaver
  Game Engine, they use this program to create the directory structure
  for a new game project. They also use this program to add new source
  files and shader files to a game project. And to update a project
  with a more recent Weaver version installed in the computer. The
  presenting code in C is cross-platform and should work under
  Windows, Linux, OpenBSD and possibly other Unix variants.}

\vskip 0.5cm plus 3pt minus 3pt

\resumo{Este artigo descreve usando programação literária o programa
  Weaver. Este programa é um gerenciador de projetos para o Motor de
  Jogos Weaver. Se alguém deseja criar um novo projeto com o motor de
  jogos, usará este programa para criar a estrutura de diretórios
  desejada. Também usará o programa para adicionar novos arquivos de
  código-fonte e shaders. Para atualizar um projeto pré-existente com
  uma nova versão de Weaver, o programa também é necessário. O código
  seguinte em C será multi-plataforma e deverá funcionar em Windows,
  Linux, OpenBSD e possivelmente outras variantes de Unix.}

\secao{1. Introdução}

Um motor de jogos é formado por um conjunto de bibliotecas e funções
que auxiliam na criação de jogos fornecendo as funcionalidades mais
comus para este tipo de desenvolvimento. Mas além das bibliotecas e
funções, deve existir um gerenciador responsável por fazer com que o
seu código utilize as bibliotecas a maneira adequada e faça as
inicializações necessárias.

O motor de jogos Weaver tem pré-requisitos bastante estritos de como o
diretório que contém um projeto Weaver deve estar organizado. É para
cumprir erstes requisitos que o programa que será apresentado é
necessário. Ele inicializa da maneira correta a estrutura de
diretórios de um novo projeto. Ele adiciona novos arquivos fonte já
com quaisquer código necessário para sua integração. E por controlar o
projeto desta forma, ele saberá atualizar as bibliotecas para versões
mais recentes se necessário.

O uso deste programa será por mieo de linha de comando. Por exemplo,
se um usuário usar o comando ``\monoespaco{weaver pong}'', será criada
uma estrutura de diretórios semelhante à mostrada na imagem que
ilustra o fim da seção com um novo projeto chamado ``pong''.

\imagem{cweb/diagrams/project_dir.eps}

As seguintes seções do artigo estão organizadas da seguinte forma. A
seção 2 abordará a licensa do software. A seção 3 listará as variáveis
usadas para controlar seu comportamento. A seção 4 trará algumas
macros que usaremos, algumas das quais apareceram na estrutura do
programa. A seção 5 apresentará algumas funções auxiliares que
utilizaremos. A seção 6 mostrará a inicialização das variáveis do
programa. A seção 7 mostrará os casos de uso do programa e como
implementá-los após termos as variáveis com os valores certos.

\secao{2. Copyright e licenciamento}

Segue abaixo a licença do programa e sua tradução não-oficial:

\espaco{5mm}\linha
\alinhaverbatim
Copyright (c) Thiago Leucz Astrizi 2015

This program is free software: you can redistribute it and/or
modify it under the terms of the GNU Affero General Public License as
published by the Free Software Foundation, either version 3 of
the License, or (at your option) any later version.

This program is distributed in the hope that it will be useful,
but WITHOUT ANY WARRANTY; without even the implied warranty of
MERCHANTABILITY or FITNESS FOR A PARTICULAR PURPOSE.  See the
GNU Affero General Public License for more details.

You should have received a copy of the GNU Affero General Public
License along with this program.  If not, see
<http://www.gnu.org/licenses/>.

\linha

Copyright (c) Thiago Leucz Astrizi 2015

Este programa é um software livre; você pode redistribuí-lo e/ou
modificá-lo dentro dos termos da Licença Pública Geral GNU Affero como
publicada pela Fundação do Software Livre (FSF); na versão 3 da
Licença, ou (na sua opinião) qualquer versão.

Este programa é distribuído na esperança de que possa ser útil,
mas SEM NENHUMA GARANTIA; sem uma garantia implícita de ADEQUAÇÃO
a qualquer MERCADO ou APLICAÇÃO EM PARTICULAR. Veja a
Licença Pública Geral GNU Affero para maiores detalhes.

Você deve ter recebido uma cópia da Licença Pública Geral GNU Affero
junto com este programa. Se não, veja
<http://www.gnu.org/licenses/>.
\alinhanormal
\linha\espaco{5mm}

A versão completa da licença pode ser obtida junto ao código-fonte
Weaver ou consultada no link mencionado.

\secao{3. Variáveis e Estrutura do Programa Weaver}

O comportamento de Weaver deve depender das seguintes variáveis:

|inside_weaver_directory|: Indicará se o programa está sendo
  invocado de dentro de um projeto Weaver.

|argument|: O primeiro argumento, ou NULL se ele não existir

|argument2|: O segundo argumento, ou NULL se não existir.

|project_version_major|: Se estamos em um projeto Weaver, qual o
  maior número da versão do Weaver usada para gerar o
  projeto. Exemplo: se a versão for 0.5, o número maior é 0. Em
  versões de teste, o valor é sempre 0.

|project_version_minor|: Se estamos em um projeto Weaver, o valor
  do menor número da versão do Weaver usada para gerar o
  projeto. Exemplo, se a versão for 0.5, o número menor é 5. Em
  versões de teste o valor é sempre 0.

|weaver_version_major|: O número maior da versão do Weaver sendo
  usada no momento.

|weaver_version_minor|: O número menor da versão do Weaver sendo
  usada no momento.

|arg_is_path|: Se o primeiro argumento é ou não um caminho
  absoluto ou relativo para um projeto Weaver.

|arg_is_valid_project|: Se o argumento passado seria válido como
  nome de projeto Weaver.

|arg_is_valid_module|: Se o argumento passado seria válido como
  um novo módulo no projeto Weaver atual.

|arg_is_valid_plugin|: Se o segundo argumento existe e se ele é um
 nome válido para um novo plugin.

|arg_is_valid_function|: Se o segundo argumento existe e se ele seria
 um nome válido para um loop principal e também para um arquivo.

|project_path|: Se estamos dentro de um diretório de projeto
  Weaver, qual o caminho para a sua base (onde há o Makefile)

|have_arg|: Se o programa é invocado com argumento.

|shared_dir|: Deverá armazenar o caminho para o diretório onde
  estão os arquivos compartilhados da instalação de Weaver. Por
  padrão, será igual à "\monoespaco{/usr/local/share/weaver}", mas caso exista a
  variável de ambiente \monoespaco{WEAVER\_DIR}, então este será
  considerado o endereço dos arquivos compartilhados.

|author_name|,|project_name| e |year|: Conterão respectivamente o
  nome do usuário que está invocando Weaver, o nome do projeto atual
  (se estivermos no diretório de um) e o ano atual. Isso será
  importante para gerar as mensagens de Copyright em novos projetos
  Weaver.

|return_value|: Que valor o programa deve retornar caso o programa
  seja interrompido no momento atual.

A estrutura geral do programa com a declaração de todas as variáveis
será:

\iniciocodigo
@(src/weaver.c@>=
@<Cabeçalhos Incluídos no Programa Weaver@>
@<Macros do Programa Weaver@>
@<Funções auxiliares Weaver@>
int main(int argc, char **argv){@/
  int return_value = 0; /* Valor de retorno. */
  bool inside_weaver_directory = false, arg_is_path = false,
    arg_is_valid_project = false, arg_is_valid_module = false,
    have_arg = false, arg_is_valid_plugin = false,
    arg_is_valid_function = false; /* Variáveis booleanas. */
  unsigned int project_version_major = 0, project_version_minor = 0,
    weaver_version_major = 0, weaver_version_minor = 0,
    year = 0;
  /* Strings UTF-8: */
  char *argument = NULL, *project_path = NULL, *shared_dir = NULL,
    *author_name = NULL, *project_name = NULL, *argument2 = NULL;
  @<Inicialização@>
  @<Caso de uso 1: Imprimir ajuda (criar projeto)@>
  @<Caso de uso 2: Imprimir ajuda de gerenciamento@>
  @<Caso de uso 3: Mostrar versão@>
  @<Caso de uso 4: Atualizar projeto Weaver@>
  @<Caso de uso 5: Criar novo módulo@>
  @<Caso de uso 6: Criar novo projeto@>
  @<Caso de uso 7: Criar novo plugin@>
  @<Caso de uso 8: Criar novo shader@>
  @<Caso de uso 9: Criar novo loop principal@>
END_OF_PROGRAM:
  @<Finalização@>
  return return_value;
}
@
\fimcodigo

\secao{4. Macros e Cabeçalhos do Programa Weaver}

O programa precisará de algumas macros. A primeira delas deverá conter
uma string com a versão do programa. A versão pode ser formada só por
letras (no caso de versões de teste) ou por um número seguido de um
ponto e de outro número (sem espaços) no caso de uma versão final do
programa.

Para a segunda macro, observe que na estrutura geral do programa vista
acima existe um rótulo chamado |END_OF_PROGRAM| logo na parte de
finalização. Uma das formas de chegarmos lá é por meio da execução
normal do programa, caso nada dê errado. Entretanto, no caso de um
erro, nós podemos também chegar lá por meio de um desvio incondicional
após imprimirmos a mensagem de erro e ajustarmos o valor de retorno do
programa. A responsabilidade de fazer isso será da segunda macro.

Por outro lado, podemos também querer encerrar o programa previamente,
mas sem que tenha havido um erro. A responsabilidade disso é da
terceira macro que definimos.

\iniciocodigo
@<Macros do Programa Weaver@>=
#define VERSION "Alpha"
#define ERROR() {perror(NULL); return_value = 1; goto END_OF_PROGRAM;}
#define END() goto END_OF_PROGRAM;
@
\fimcodigo

Como estamos usando a função de biblioteca \monoespaco{perror},
devemos incluir o cabeçalho \monoespaco{stdio.h}, o que também nos
trará s funções de imprimir na tela, abrir e fechar arquivos e
escrever neles, o que nos será útil. Vamos inserir suporte à valores
booleanos que usamos na própria estrutura do programa e também a
biblioteca padrão, que tem funções como |exit| usadas na estrutura do
programa:

\iniciocodigo
@<Cabeçalhos Incluídos no Programa Weaver@>=
#include <stdio.h> // printf, fprintf, fopen, fclose, fgets, fgetc, perror
#include <stdbool.h> // bool, true, false
#include <stdlib.h> // free, exit, getenv
@
\fimcodigo

\secao{5. Funções Auxiliares}

Listemos aqui algumas funções que usaremos ao longo do programa para
facilitar sua descrição.

\subsecao{5.1. path\_up: Manipula Caminho}

Para manipularmos o caminho da árvore de diretórios, usaremos uma
função auxiliar que recebe como entrada uma string com um caminho na
árvore de diretórios e apaga todos os últimos caracteres até apagar
dois ``/''. Assim em ``/home/alice/projeto/diretorio/'' ele retornaria
``/home/alice/projeto'' efetivamente subindo um nível na árvore de
diretórios.

É importante lembrar que no Windows o separador não é o ``/'', mas o
``\\''. Então vamos tratar o separador de forma diferente de acordo
com o sistema:

\iniciocodigo
@<Funções auxiliares Weaver@>=
void path_up(char *path){
#if !defined(_WIN32)
  char separator = '/';
#else
  char separator = '\\';
#endif
  int erased = 0;
  char *p = path;
  while(*p != '\0') p ++; // Vai até o fim
  while(erased < 2 && p != path){
    p --;
    if(*p == separator) erased ++;
    *p = '\0'; // Apaga
  }
}
@
\fimcodigo

Note que caso a função receba uma string que não possua dois ``/'' em
seu nome, acabamos apagando toda a string. Neste programa limitaremos
o uso desta função a strings com caminhos de arquivos que não estão na
raíz e diretórios diferentes da própria raíz que terminam sempre com
``/'', então não teremos problemas pois a restrição do número de
barras será cumprida. Ex: ``/etc/'' e ``/tmp/file.txt''.

\subsecao{5.2. directory\_exists: Arquivo existe e é diretório}

Para checar se o diretório \monoespaco{.weaver} existe, definimos
|directory_exist(x)| como uma função que recebe uma string
correspondente à localização de um arquivo e que deve retornar 1 se
|x| for um diretório existente, -1 se |x| for um arquivo existente e 0
caso contrário. Primeiro criamos as macros para não nos esquecermos do
que significa cada número de retorno:

\iniciocodigo
@<Macros do Programa Weaver@>+=
#define NAO_EXISTE             0
#define EXISTE_E_EH_DIRETORIO  1
#define EXISTE_E_EH_ARQUIVO   -1
@
\fimcodigo

\iniciocodigo
@<Funções auxiliares Weaver@>+=
int directory_exist(char *dir){
#if !defined(_WIN32)
  // Unix:
  struct stat s; // Armazena status se um diretório existe ou não.
  int err; // Checagem de erros
  err = stat(dir, &s); // .weaver existe?
  if(err == -1) return NAO_EXISTE;
  if(S_ISDIR(s.st_mode)) return EXISTE_E_EH_DIRETORIO;
  return EXISTE_E_EH_ARQUIVO;
#else
  // Windows:
  DWORD dwAttrib = GetFileAttributes(dir);
  if(dwAttrib == INVALID_FILE_ATTRIBUTES) return NAO_EXISTE;
  if(!(dwAttrib & FILE_ATTRIBUTE_DIRECTORY)) return EXISTE_E_EH_ARQUIVO;
  else return EXISTE_E_EH_DIRETORIO
#endif
}
@
\fimcodigo

Dependendo de estarmos no Windows ou em sistemas Unix, usamos funções
diferentes e vamos precisar de cabeçalhos diferentes:

\iniciocodigo
@<Cabeçalhos Incluídos no Programa Weaver@>=
#if !defined(_WIN32)
#include <sys/types.h> // stat, getuid, getpwuid, mkdir
#include <sys/stat.h> // stat, mkdir
#else
#include <windows.h> // GetFileAttributes, ...
#endif
@
\fimcodigo

\subsecao{5.3. concatenate: Concatena strings}

A última função auxiliar da qual precisaremos é uma função para
concatenar strings. Ela deve receber um número arbitrário de srings
como argumento, mas a última string deve ser uma string vazia. E irá
retornar a concatenação de todas as strings passadas como argumento.

A função irá alocar sempre uma nova string, a qual deverá ser
desalocada antes do programa terminar. Como exemplo,
|concatenate("tes", " ", "te", "")| retorna |"tes te"|.

\iniciocodigo
@<Funções auxiliares Weaver@>+=
char *concatenate(char *string, ...){
  va_list arguments;
  char *new_string, *current_string = string;
  size_t current_size = strlen(string) + 1;
  char *realloc_return;
  va_start(arguments, string);
  new_string = (char *) malloc(current_size);
  if(new_string == NULL) return NULL;
   // Copia primeira string de acordo com o indicado pelo sistema operacional
#ifdef __OpenBSD__
  strlcpy(new_string, string, current_size);
#else
  strcpy(new_string, string);
#endif
  while(current_string[0] != '\0'){ // Pára quando copiamos o ""
    current_string = va_arg(arguments, char *);
    current_size += strlen(current_string);
    realloc_return = (char *) realloc(new_string, current_size);
    if(realloc_return == NULL){
      free(new_string);
      return NULL;
    }
    new_string = realloc_return;
     // Copia próxima string de acordo com o recomendado pelo sistema
#ifdef __OpenBSD__
    strlcat(new_string, current_string, current_size);
#else
    strcat(new_string, current_string);
#endif
  }
  return new_string;
}
@
\fimcodigo

É importante lembrarmos que a função |concatenate| sempre deve receber
como último argumento uma string vazia ou teremos um \italico{buffer
  overflow}. Esta função é perigosa e deve ser usada sempre tomando-se
este cuidado.

O uso desta função requer que usemos o seguinte cabeçalho:

\iniciocodigo
@<Cabeçalhos Incluídos no Programa Weaver@>=
#include <string.h> // strcmp, strcat, strcpy, strncmp
#include <stdarg.h> // va_start, va_arg
@
\fimcodigo

\subsecao{5.4. basename: Obtém o caminho do diretório de arquivo}

Esta função já existe em sistemas Unix. Dado o caminho completo para
um arquivo, ela retorna uma string apenas com o nome do arquivo. Ela
não precisa alocar uma nova string, ela retorna um ponteiro para o
nome do arquivo dentro do próprio caminho recebido como argumento.
Vamos defini-la agora para sistemas Windows:

\iniciocodigo
@<Funções auxiliares Weaver@>+=
#if defined(_WIN32)
char *basename(char *path){
  char *p = path;
  char *last_delimiter = NULL;
  while(*p != '\0'){
    if(*p == '\\')
      last_delimiter = p;
    p ++;
  }
  if(last_delimiter != NULL)
    return last_delimiter + 1;
  else
    return path;
}
#endif
@
\fimcodigo

Mesmo que não precisemos definir esta função em sistemas Unix, ainda
precisamos incluí-la com o cabeçalho:

\iniciocodigo
@<Cabeçalhos Incluídos no Programa Weaver@>=
#include <libgen.h>
@
\fimcodigo

\subsecao{5.5. copy\_single\_file: Copia um único arquivo para diretório}

A função |copy_single_file| tenta copiar o
arquivo cujo caminho é o primeiro argumento para o diretório cujo
caminho é o segundo argumento. Se ela conseguir, retorna 1 e retorna 0
caso contrário.

\iniciocodigo
@<Funções auxiliares Weaver@>+=
int copy_single_file(char *file, char *directory){
  int block_size, bytes_read;
  char *buffer, *file_dst;
  FILE *orig, *dst;
  // Inicializa 'block_size':
  @<Descobre tamanho do bloco do sistema de arquivos@>
  buffer = (char *) malloc(block_size); // Aloca buffer de cópia
  if(buffer == NULL) return 0;
  file_dst = concatenate(directory, "/", basename(file), "");
  if(file_dst == NULL) return 0;
  orig = fopen(file, "r"); // Abre arquivo de origem
  if(orig == NULL){
    free(buffer);
    free(file_dst);
    return 0;
  }
  dst = fopen(file_dst, "w"); // Abre arquivo de destino
  if(dst == NULL){
    fclose(orig);
    free(buffer);
    free(file_dst);
    return 0;
  }
  while((bytes_read = fread(buffer, 1, block_size, orig)) > 0){
    fwrite(buffer, 1, bytes_read, dst); // Copia origem -> buffer -> destino
  }
  fclose(orig);
  fclose(dst);
  free(file_dst);
  free(buffer);
  return 1;
}
@
\fimcodigo

O mais eficiente é que o buffer usado para copiar arquivos tenha o
mesmo tamanho do bloco do sistema de arquivos. Para obter o valor
correto deste tamanho, usamos o seguinte trecho de código em sistemas
Unix:

\iniciocodigo
@<Descobre tamanho do bloco do sistema de arquivos@>=
#if !defined(_WIN32)
{
  struct stat s;
  stat(directory, &s);
  block_size = s.st_blksize;
  if(block_size <= 0){
    block_size = 4096;
  }
}
#endif
@
\fimcodigo

No Windows o código equivalente seria:

\iniciocodigo
@<Descobre tamanho do bloco do sistema de arquivos@>+=
#if defined(_WIN32)
{
  DWORD dummy;
  DISK_GEOMETRY s;
  file = CreateFileW(".temp.dat", 0, FILE_SHARE_READ | FILE_SHARE_WRITE,
                     NULL, OPEN_EXISTING, FILE_FLAG_DELETE_ON_CLOSE, NULL);
  DeviceIoControl(file, IOCTL_DISK_GET_DRIVE_GEOMETRY, NULL, 0, &s,
                  sizeof(s), &dummy, (LPOVERLAPPED) NULL);
  CloseHandle(file);
  block_size = (ULONG) s.BytesPerSector;
}
#endif
@
\fimcodigo

\subsecao{5.6. copy\_files: Copia todos os arquivos de origem para destino}

De posse da função que copia um só arquivo, precisamos definir uma
função para copiar todos os arquivos de dentro d eum diretório para
outro recursivamente. Isso é algo trabalhoso, pois precisamos listar
todo o conteúdo de um diretório para obter seus arquivos. Como fazer
isso varia dependendo do sistema operacional.


Em sistemas Unix a função usará |readdir| para ler o conteúdo de
arquivos:

\iniciocodigo
@<Funções auxiliares Weaver@>+=
#if !defined(_WIN32)
int copy_files(char *orig, char *dst){
  DIR *d = NULL;
  struct dirent *dir;
  d = opendir(orig);
  if(d){
    while((dir = readdir(d)) != NULL){ // Loop para ler cada arquivo
      char *file;
      file = concatenate(orig, "/", dir -> d_name, "");
      if(file == NULL){
        return 0;
      }
#if (defined(__linux__) || defined(_BSD_SOURCE)) && defined(DT_DIR)
      // Se suportamos DT_DIR, não precisamos chamar a função 'stat':
      if(dir -> d_type == DT_DIR){
#else
      struct stat s;
      int err;
      err = stat(file, &s);
      if(err == -1) return 0;
      if(S_ISDIR(s.st_mode)){
#endif
      // Se concluirmos estar lidando com subdiretório via 'stat' ou 'DT_DIR':
        char *new_dst;
        new_dst = concatenate(dst, "/", dir -> d_name, "");
        if(new_dst == NULL){
          return 0;
        }
        if(strcmp(dir -> d_name, ".") && strcmp(dir -> d_name, "..")){
          if(directory_exist(new_dst) == NAO_EXISTE) mkdir(new_dst, 0755);
          if(copy_files(file, new_dst) == 0){
            free(new_dst);
            free(file);
            closedir(d);
            return 0;
          }
        }
        free(new_dst);
      }
      else{
        // Se concluimos estar diante de um arquivo usual:
        if(copy_single_file(file, dst) == 0){
          free(file);
          closedir(d);
          return 0;
        }
      }
    free(file);
    } // Fim do loop para ler cada arquivo
    closedir(d);
  }
  return 1;
}
#endif
@
\fimcodigo

E isso requer inserir o cabeçalho:

\iniciocodigo
@<Cabeçalhos Incluídos no Programa Weaver@>=
#if !defined(_WIN32)
#include <dirent.h> // readdir, opendir, closedir
#endif
@
\fimcodigo

No Windows, não é necessário inserir nenhum cabeçalho novo que já não
está inserido. A definição da função fica assim:

\iniciocodigo
@<Funções auxiliares Weaver@>+=
#if defined(_WIN32)
int copy_files(char *orig, char *dst){
  WIN32_FIND_DATA file;
  HANDLE dir = NULL;
  dir = FindFirstFile(orig, &file));
  if(dir !== INVALID_HANDLE_VALUE){
    // The first file shall be '.' and should be safely ignored
    do{
      if(strcmp(file.cFileName, ".") && strcmp(file.cFileName, "..")){
        char *path;
        path = concatenate(orig, "\\", file.cFileName, "");
        if(path == NULL){
          return 0;
        }
        if(file.dwFileAttributes & FILE_ATTRIBUTE_DIRECTORY){
          char *dst_path;
          dst_path = concatenate(dst, "\\", file.cFileName, "");
          if(directory_exist(dst_path) == NAO_EXISTE)
            CreateDirectoryW(dst_path, NULL);
          if(copy_files(path, dst_path) == 0){
            free(dst_path);
            free(path);
            FindClose(dir);
            return 0;
          }
          free(dst_path);
        }
        else{ // file
          if(copy_single_file(path, dst) == 0){
            free(path);
            FindClose(dir);
            return 0;
          }
        }
        free(path);
      }
    }while(FindNextFile(dir, &file));
  }
  FindClose(dir);
  return 1;
}
#endif
@
\fimcodigo

\subsecao{5.7. write\_copyright: Escreve mensagem de copyright em arquivo}

Por padrão, projetos Weaver utilizam a licença GNU GPL3. Como códigos
sob esta licença são copiados e usados estaticamente em novos
projetos, eles precisam necessariamente ter uma licença igual ou
compatível.

O código é bastante simples e requer apenas alguns parâmetros com o
nome do autor e ano atual:

\iniciocodigo
@<Funções auxiliares Weaver@>+=
void write_copyright(FILE *fp, char *author_name, char *project_name, int year){
  char license[] = "/*\nCopyright (c) %s, %d\n\nThis file is part of %s.\n\n%s\
 is free software: you can redistribute it and/or modify\nit under the terms of\
 the GNU Affero General Public License as published by\nthe Free Software\ 
 Foundation, either version 3 of the License, or\n(at your option) any later\
 version.\n\n\
%s is distributed in the hope that it will be useful,\nbut WITHOUT ANY\
  WARRANTY; without even the implied warranty of\nMERCHANTABILITY or FITNESS\
  FOR A PARTICULAR PURPOSE.  See the\nGNU Affero General Public License for more\
  details.\n\nYou should have received a copy of the GNU Affero General Public License\
\nalong with %s. If not, see <http://www.gnu.org/licenses/>.\n*/\n\n";
  fprintf(fp, license, author_name, year, project_name, project_name,
          project_name, project_name);
}
@
\fimcodigo

\subsecao{5.8. create\_dir: Cria novos diretórios}

Esta é a função responsável por criar uma lista de diretórios. Isso é
algo bastante simples, mas deve ficar encapsulado em sua própria
função por causa das diferenças entre Sistemas Operacionais sobre como
realizar a tarefa.

Esta função deve receber uma lista variável de strings como argumento,
sendo que o último argumento precisa ser uma string vazia ou
NULL. Para cada argumento representando um caminho, a função criará o
diretório no caminho especificado. Por padrão, o separador em caminhos
será o ``\monoespaco{/}'' para podermos usar esta função da mesma
forma independente do Sistema Operacional. Em sistemas em que o
separador é outro, como no caso do Windows que usa o
``\monoespaco{\\}'', a função fará a conversão da maneira apropriada.

Em sistemas Unix nós precisamos também especificar as permissões
máximas que o diretório terá em termos de leitura, escrita e
execução. Tais permissões podem ser mais restritas de acordo com a
configuração do sistema. No Windows, que tem um sistema de permissões
mais hierárquica, nós apenas herdamos as permissões do diretório pai.

Em caso de erro, nós retornaremos -1. Se tudo deu certo, retornamos 1.

A definição de nossa função será então:

\iniciocodigo
@<Funções auxiliares Weaver@>+=
int create_dir(char *string, ...){
  char *current_string;
  va_list arguments;
  va_start(arguments, string);
  int err = 1;
  current_string = va_arg(arguments, char *);
  while(current_string != NULL && current_string[0] != '\0' && err != -1){
#if !defined(_WIN32)
    err = mkdir(current_string, S_IRWXU | S_IRWXG | S_IROTH);
#else
    {
      char *p = current_string;
      while(*p != '\0'){
        if(*p == '/') *p = '\\';
        p ++;
      }
      if(!CreateDirectoryW(current_string, NULL))
        err = -1;
    }
#endif
    current_string = va_arg(arguments, char *);
  }
  return err;
}
@
\fimcodigo

\subsecao{5.9. append\_file: Concatena um arquivo no outro}

Esta será uma função diferente, pois ela será mais focada em resolver
de maneira eficiente um único caso de uso do que apresentar uma
interface intuitiva e consistente. Ela irá receber como entrada um
ponteiro para um arquivo já aberto (iremos chamar esta função depois
de já termos aberto o arquivo para escrever o copyright) e receberá
dois outros argumentos: o diretório em que está o arquivo de origem e
o nome do arquivo de origem, separados em duas strings. Dessa forma, o
trabalho de concatenar as duas coisas fica com esta função, não com
quem a está invocando.

Sua definição será:

\iniciocodigo
@<Funções auxiliares Weaver@>+=
int append_file(FILE *fp, char *dir, char *file){
  int block_size, bytes_read;
  char *buffer, *directory = ".";
  char *path = concatenate(dir, file, "");
  if(path == NULL) return 0;
  FILE *origin;
  @<Descobre tamanho do bloco do sistema de arquivos@>
  buffer = (char *) malloc(block_size);
  if(buffer == NULL){
    free(path);
    return 0;
  }
  origin = fopen(path, "r");
  if(origin == NULL){
    free(buffer);
    free(path);
    return 0;
  }
  while((bytes_read = fread(buffer, 1, block_size, origin)) > 0){
    fwrite(buffer, 1, bytes_read, fp);
  }
  fclose(origin);
  free(buffer);
  free(path);
  return 1;
}
@
\fimcodigo

\secao{6. Inicialização das Variáveis}

\subsecao{6.1. inside\_weaver\_directory e project\_path: Onde estamos}

A primeira das variáveis é |inside_weaver_directory|, que deve valer
|false| se o programa foi invocado de fora de um diretório de projeto
Weaver e |true| caso contrário.

Como definir se estamos em um diretório que pertence à um projeto
Weaver? Simples. São diretórios que contém dentro de si ou em um
diretório ancestral um diretório oculto
chamado \monoespaco{.weaver}. Caso encontremos este diretório oculto,
também podemos aproveitar e ajustar a variável |project_path| para
apontar para o local onde ele está. Se não o encontrarmos, estaremos
fora de um diretório Weaver e não precisamos mudar nenhum valor das
duas variáveis, pois elas deverão permanecer com o valor padrão
|NULL|.

Em suma, o que precisamos é de um loop com as seguintes
características:

\negrito{Invariantes}: A variável |complete_path| deve sempre
  possuir o caminho completo do diretório \monoespaco{.weaver} se ele
  existisse no diretório atual.

\negrito{Inicialização:} Inicializamos tanto o |complete_path|
  para serem válidos de acordo com o diretório em que o programa é
  invocado.

\negrito{Manutenção:} Em cada iteração do loop nós verificamos se
  encontramos uma condição de finalização. Caso contrário, subimos
  para o diretório pai do qual estamos, sempre atualizando as
  variáveis para que o invariante continue válido.

\negrito{Finalização}: Interrompemos a execução do loop se uma das
  três condições ocorrerem:

a) |complete_path == "/.weaver"|: Neste caso não podemos subir mais na
árvore de diretórios, pois estamos na raiz do sistema de arquivos. Não
encontramos um diretório \monoespaco{.weaver}. Isso significa que não
estamos dentro de um projeto Weaver.

b) |complete_path == "C:\\.weaver"|: A letra inicial pode não ser um
``C''. De qualquer forma, estamos na raíz do sistema dos arquivos e
não podemos subir mais como no caso acima. Com a diferença de estarmos
no Windows.

c) |complete_path == "./.weaver"| e tal arquivo existe e é diretório:
Neste caso encontramos um diretório \monoespaco{.weaver} e descobrimos
que estamos dentro de um projeto Weaver. Podemos então atualizar a
variável |project_path| para o diretório em que paramos.

O código de inicialização destas variáveis será então:

\iniciocodigo
@<Inicialização@>=
char *path = NULL, *complete_path = NULL;
#if !defined(_WIN32)
path = getcwd(NULL, 0); // Unix
#else
{ // Windows
  DWORD bsize;
  bsize = GetCurrentDirectory(0, NULL);
  path = (char *) malloc(bsize);
  GetCurrentDirectory(bsize, path);
}
#endif
if(path == NULL) ERROR();
complete_path = concatenate(path, "/.weaver", "");
free(path);
if(complete_path == NULL) ERROR();
@
\fimcodigo

Para obtermos o diretório atual, vamos precisar do cabeçalho:

\iniciocodigo
@<Cabeçalhos Incluídos no Programa Weaver@>=
#if !defined(_WIN32)
#include <unistd.h> // get_current_dir_name, getcwd, stat, chdir, getuid
#endif
@
\fimcodigo

Agora iniciamos um loop que terminará quando |complete_path| for igual
à \monoespaco{/.weaver} (chegamos no fim da árvore de diretórios e não
encontramos nada) ou quando realmente existir o
diretório \monoespaco{.weaver/} no diretório examinado. E no fim do
loop, sempre vamos para o diretório-pai do qual estamos:

\iniciocodigo
@<Inicialização@>+=
{
  size_t tmp_size = strlen(complete_path);
  // Testa se chegamos ao fim:
  while(strcmp(complete_path, "/.weaver") &&
	strcmp(complete_path, "\\.weaver") &&
	strcmp(complete_path + 1, ":\\.weaver")){
    if(directory_exist(complete_path) == EXISTE_E_EH_DIRETORIO){
      inside_weaver_directory = true;
      complete_path[strlen(complete_path) - 7] = '\0'; // Apaga o '.weaver'
      project_path = concatenate(complete_path, "");
      if(project_path == NULL){ free(complete_path); ERROR(); }
      break;
    }
    else{
      path_up(complete_path);
#ifdef __OpenBSD__
      strlcat(complete_path, "/.weaver", tmp_size);
#else
      strcat(complete_path, "/.weaver");
#endif
    }
  }
  free(complete_path);
}
@
\fimcodigo

Como alocamos memória para |project_path| armazenar o endereço do
projeto atual se estamos em um projeto Weaver, no final do programa
teremos que desalocar a memória:

\iniciocodigo
@<Finalização@>=
if(project_path != NULL) free(project_path);
@
\fimcodigo

\subsecao{6.2. weaver\_version\_major e weaver\_version\_minor: Versão do Programa}

Para descobrirmos a versão atual do Weaver que temos, basta consultar
o valor presente na macro |VERSION|. Então, obtemos o número de versão
maior e menor que estão separados por um ponto (se existirem). Note
que se não houver um ponto no nome da versão, então ela é uma versão
de testes. Mesmo neste caso o código abaixo vai funcionar, pois a
função |atoi| iria retornar 0 nas duas invocações por encontrar
respectivamente uma string sem dígito algum e um fim de string sem
conteúdo:

\iniciocodigo
@<Inicialização@>+=
{
  char *p = VERSION;
  while(*p != '.' && *p != '\0') p ++;
  if(*p == '.') p ++;
  weaver_version_major = atoi(VERSION);
  weaver_version_minor = atoi(p);
}
@
\fimcodigo

\subsecao{6.3. project\_version\_major e project\_version\_minor: Versão do Projeto}

Se estamos dentro de um projeto Weaver, temos que inicializar
informação sobre qual versão do Weaver foi usada para atualizá-lo pela
última vez. Isso pode ser obtido lendo o arquivo
\italico{.weaver/version} localizado dentro do diretório Weaver. Se não
estamos em um diretório Weaver, não precisamos inicializar tais
valores. O número de versão maior e menor é separado por um ponto.

\iniciocodigo
@<Inicialização@>+=
if(inside_weaver_directory){
  FILE *fp;
  char *p, version[10];
  char *file_path = concatenate(project_path, ".weaver/version", "");
  if(file_path == NULL) ERROR();
  fp = fopen(file_path, "r");
  free(file_path);
  if(fp == NULL) ERROR();
  p = fgets(version, 10, fp);
  if(p == NULL){ fclose(fp); ERROR(); }
  while(*p != '.' && *p != '\0') p ++;
  if(*p == '.') p ++;
  project_version_major = atoi(version);
  project_version_minor = atoi(p);
  fclose(fp);
}
@
\fimcodigo

\subsecao{6.4. have\_arg, argument e argument2: Argumentos de Invocação}

Uma das variáveis mais fáceis e triviais de se inicializar. Basta
consultar |argc| e |argv|.

\iniciocodigo
@<Inicialização@>+=
have_arg = (argc > 1);
if(have_arg) argument = argv[1];
if(argc > 2) argument2 = argv[2];
@
\fimcodigo

\subsecao{6.5. arg\_is\_path: Se argumento é diretório}

Agora temos que verificar se no caso de termos um argumento, se ele é
um caminho para um projeto Weaver existente ou não. Para isso,
checamos se ao concatenarmos \monoespaco{/.weaver} no argumento
encontramos o caminho de um diretório existente ou não.

\iniciocodigo
@<Inicialização@>+=
if(have_arg){
  char *buffer = concatenate(argument, "/.weaver", "");
  if(buffer == NULL) ERROR();
  if(directory_exist(buffer) == EXISTE_E_EH_DIRETORIO){
    arg_is_path = 1;
  }
  free(buffer);
}
@
\fimcodigo

\subsecao{6.6. shared\_dir: Onde arquivos estão instalados}

A variável |shared_dir| deverá conter onde estão os arquivos
compartilhados da instalação de Weaver. Tais arquivos são as próprias
bibliotecas a serem inseridas estaticamente e modelos de código
fonte. Se existir a macro passada durante a
compilação \monoespaco{WEAVER\_DIR}, este será o caminho em que estão
os arquivos. Caso contrário, assumiremos o valor padrão
de \monoespaco{/usr/local/share/weaver} em sistemas baseados em Unix e
o local apontado pela variável de ambiente ProgramFiles em ambientes
Windows.

@<Inicialização@>+=
{
#ifdef WEAVER_DIR
  shared_dir = concatenate(WEAVER_DIR, "");
#else
#if !defined(_WIN32)
  shared_dir = concatenate("/usr/local/share/weaver/", ""); // Unix
#else
  { // Windows
    char *temp_buf;
    DWORD bsize = GetEnvironmentVariable("ProgramFiles", temp_buf, 0);
    temp_buf = (char *) malloc(bsize);
    GetEnvironmentVariable("ProgramFiles", temp_buf, bsize);
    shared_dir = concatenate(temp_buf, "\weaver", "");
    free(temp_buf);
  }
#endif
#endif
  if(shared_dir == NULL) ERROR();
}
@
\fimcodigo

Com isso damos poder durante a compilação para determinar onde os
dados do motor Weaver serão armazenados no sistema. Algo mais comum de
ser alterado em sistemas Unix que no Windows, onde espera-se que os
programas sejam armazenados no mesmo lugar.

No Windows o código é mais longo principalmente por termos que
determinar manualmente o nome do local padrão de se armazenar os
programas. O endereço pode variar de acordo com o idioma do sistema,
com a unidade de volume em que ele está ou com o fato do programa ter
sido compilado em máquina com 32 ou 64 bits.

No fim do programa devemos desalocar a memória alocada para
|shared_dir|:

\iniciocodigo
@<Finalização@>+=
if(shared_dir != NULL) free(shared_dir);
@
\fimcodigo

\subsecao{6.7. arg\_is\_valid\_project: Se o argumento é um nome de projeto}

A próxima questão que deve ser averiguada é se o que recebemos como
argumento, caso haja argumento, pode ser o nome de um projeto Weaver
válido ou não. Para isso, três condições precisam ser
satisfeitas:

1) O nome base do projeto deve ser formado somente por caracteres
alfanuméricos e underline (embora uma barra possa aparecer para passar
o caminho completo de um projeto).

2) Não pode existir um arquivo com o mesmo nome do projeto no local
indicado para a criação.

3) O projeto não pode ter o nome de nenhum arquivo que costuma ficar
no diretório base de um projeto Weaver (como ``Makefile''). Do
contrário, na hora da compilação comandos como ``\monoespaco{gcc
game.c -o Makefile}'' poderiam ser executados e sobrescreveriam
arquivos importantes.

Para isso, usamos o seguinte código:

\iniciocodigo
@<Inicialização@>+=
if(have_arg && !arg_is_path){
  char *buffer;
  char *base = basename(argument);
  int size = strlen(base);
  int i;
  // Checando caracteres inválidos no nome:
  for(i = 0; i < size; i ++){
    if(!isalnum(base[i]) && base[i] != '_'){
      goto NOT_VALID;
    }
  }
  // Checando se arquivo existe:
  if(directory_exist(argument) != NAO_EXISTE){
    goto NOT_VALID;
  }
  // Checando se conflita com arquivos de compilação:
  buffer = concatenate(shared_dir, "project/", base, "");
  if(buffer == NULL) ERROR();
  if(directory_exist(buffer) != NAO_EXISTE){
    free(buffer);
    goto NOT_VALID;
  }
  free(buffer);
  arg_is_valid_project = true;
}
NOT_VALID:
@
\fimcodigo

Para podermos checar se um caractere é alfanumérico, incluimos a
seguinte biblioteca:

\iniciocodigo
@<Cabeçalhos Incluídos no Programa Weaver@>=
#include <ctype.h> // isalnum
@
\fimcodigo

\subsecao{6.8. arg\_is\_valid\_module: Se o argumento pode ser um nome de módulo}

Checar se o argumento que recebemos pode ser um nome válido para um
módulo só faz sentido se estivermos dentro de um diretório Weaver e se
um argumento estiver sendo passado. Neste caso, o argumento é um nome
válido se ele contiver apenas caracteres alfanuméricos, underline e se
não existir no projeto um arquivo \monoespaco{.c} ou \monoespaco{.h}
em
\monoespaco{src/} que tenha o mesmo nome do argumento passado:

\iniciocodigo
@<Inicialização@>+=
if(have_arg && inside_weaver_directory){
  char *buffer;
  int i, size;
  size = strlen(argument);
  // Checando caracteres inválidos no nome:
  for(i = 0; i < size; i ++){
    if(!isalnum(argument[i]) && argument[i] != '_'){
      goto NOT_VALID_MODULE;
    }
  }
  // Checando por conflito de nomes:
  buffer = concatenate(project_path, "src/", argument, ".c", "");
  if(buffer == NULL) ERROR();
  if(directory_exist(buffer) != NAO_EXISTE){
    free(buffer);
    goto NOT_VALID_MODULE;
  }
  buffer[strlen(buffer) - 1] = 'h';
  if(directory_exist(buffer) != NAO_EXISTE){
    free(buffer);
    goto NOT_VALID_MODULE;
  }
  free(buffer);
  arg_is_valid_module = true;
}
NOT_VALID_MODULE:
@
\fimcodigo

\subsecao{6.9. arg\_is\_valid\_plugin: Se o argumento pode ser um nome de plugin}

Para que um argumento seja um nome válido para plugin, ele deve ser
composto só por caracteres alfanuméricos ou underline e não existir no
diretório
\monoespaco{plugin} um arquivo com a extensão \monoespaco{.c} de mesmo
nome. Também precisamos estar naturalmente, em um diretório Weaver.

\iniciocodigo
@<Inicialização@>+=
if(argument2 != NULL && inside_weaver_directory){
  int i, size;
  char *buffer;
  size = strlen(argument2);
  // Checando caracteres inválidos no nome:
  for(i = 0; i < size; i ++){
    if(!isalnum(argument2[i]) && argument2[i] != '_'){
      goto NOT_VALID_PLUGIN;
    }
  }
  // Checando se já existe plugin com mesmo nome:
  buffer = concatenate(project_path, "plugins/", argument2, ".c", "");
  if(buffer == NULL) ERROR();
  if(directory_exist(buffer) != NAO_EXISTE){
    free(buffer);
    goto NOT_VALID_PLUGIN;
  }
  free(buffer);
  arg_is_valid_plugin = true;
}
NOT_VALID_PLUGIN:
@
\fimcodigo

\subsecao{6.10. arg\_is\_valid\_function: Se o argumento pode ser um nome de função de loop principal}

Para que essa variável seja verdadeira, é preciso existir um segundo
argumento e ele deve ser formado somente por caracteres alfanuméricos
ou underline. Além disso, o primeiro caractere precisa ser uma letra e
ele não pode ter o mesmo nome de alguma palavra reservada em C.

\iniciocodigo
@<Inicialização@>+=
if(argument2 != NULL && inside_weaver_directory &&
   !strcmp(argument, "--loop")){
  int i, size;
  char *buffer;
  // Primeiro caractere não pode ser dígito
  if(isdigit(argument2[0]))
    goto NOT_VALID_FUNCTION;
  size = strlen(argument2);
  // Checando caracteres inválidos no nome:
  for(i = 0; i < size; i ++){
    if(!isalnum(argument2[i]) && argument2[i] != '_'){
      goto NOT_VALID_PLUGIN;
    }
  }
  // Checando se existem arquivos com o nome indicado:
  buffer = concatenate(project_path, "src/", argument2, ".c", "");
  if(buffer == NULL) ERROR();
  if(directory_exist(buffer) != NAO_EXISTE){
    free(buffer);
    goto NOT_VALID_FUNCTION;
  }
  buffer[strlen(buffer)-1] = 'h';
  if(directory_exist(buffer) != NAO_EXISTE){
    free(buffer);
    goto NOT_VALID_FUNCTION;
  }
  free(buffer);
  // Checando se recebemos como argumento uma palavra reservada em C:
  if(!strcmp(argument2, "auto") || !strcmp(argument2, "break") ||
     !strcmp(argument2, "case") || !strcmp(argument2, "char") ||
     !strcmp(argument2, "const") || !strcmp(argument2, "continue") ||
     !strcmp(argument2, "default") || !strcmp(argument2, "do") ||
     !strcmp(argument2, "int") || !strcmp(argument2, "long") ||
     !strcmp(argument2, "register") || !strcmp(argument2, "return") ||
     !strcmp(argument2, "short") || !strcmp(argument2, "signed") ||
     !strcmp(argument2, "sizeof") || !strcmp(argument2, "static") ||
     !strcmp(argument2, "struct") || !strcmp(argument2, "switch") ||
     !strcmp(argument2, "typedef") || !strcmp(argument2, "union") ||
     !strcmp(argument2, "unsigned") || !strcmp(argument2, "void") ||
     !strcmp(argument2, "volatile") || !strcmp(argument2, "while") ||
     !strcmp(argument2, "double") || !strcmp(argument2, "else") ||
     !strcmp(argument2, "enum") || !strcmp(argument2, "extern") ||
     !strcmp(argument2, "float") || !strcmp(argument2, "for") ||
     !strcmp(argument2, "goto") || !strcmp(argument2, "if"))
    goto NOT_VALID_FUNCTION;
  arg_is_valid_function = true;
}
NOT_VALID_FUNCTION:
@

\subsecao{6.11. author\_name: Nome do criador do código}

A variável |author_name| deve conter o nome do usuário que está
invocando o programa. Esta informação é útil para gerar uma mensagem
de Copyright nos arquivos de código fonte de novos módulos.

Isso será feito de maneira diferente em sistemas Unix e Windows. Em
sistemas Unix, começamos obtendo o seu UID. De posse dele, obtemos
todas as informações de login com um |getpwuid|. Se o usuário tiver
registrado um nome em \monoespaco{/etc/passwd}, obtemos tal nome na
estrutura retornada pela função. Caso contrário, assumiremos o login
como sendo o nome:

\iniciocodigo
@<Inicialização@>+=
#if !defined(_WIN32)
{
  struct passwd *login;
  int size;
  char *string_to_copy;
  login = getpwuid(getuid()); // Obtém dados de usuário
  if(login == NULL) ERROR();
  size = strlen(login -> pw_gecos);
  if(size > 0)
    string_to_copy = login -> pw_gecos;
  else
    string_to_copy = login -> pw_name;
  size = strlen(string_to_copy);
  author_name = (char *) malloc(size + 1);
  if(author_name == NULL) ERROR();
#ifdef __OpenBSD__
  strlcpy(author_name, string_to_copy, size + 1);
#else
  strcpy(author_name, string_to_copy);
#endif
}
#endif
@
\fimcodigo

No Windows, o nome pode ser obtido com a função |GetUserNameExA|. Na
primeira invocação tentamos obter o tamanho do buffer necessário para
armazenarmos o nome e na segunda obtemos o nome em si. Em caso de
erro, assumimos um tamanho padrão de 64 bytes e usamos a variável de
ambiente \monoespaco{USERNAME}.

\iniciocodigo
@<Inicialização@>+=
#if defined(_WIN32)
{
  int size = 0;
  GetUserNameExA(NameDisplay, author_name, &size);
  if(size == 0)
    size = 64;
  author_name = (char *) malloc(size);
  if(GetUserNameExA(NameDisplay, author_name, &size) == 0){
    strncpy(author_name, getenv("USERNAME"), size);
    author_name[size - 1] = '\0';
  }
}
#endif
@
\fimcodigo

Depois, precisaremos desalocar a memória ocupada por |author_name|:

\iniciocodigo
@<Finalização@>+=
if(author_name != NULL) free(author_name);
@
\fimcodigo

Para que o código Unix funcione, devemos inserir a biblioteca abaixo
para termos acesso ao \monoespaco{getpwuid}:

@<Cabeçalhos Incluídos no Programa Weaver@>=
#if !defined(_WIN32)
#include <pwd.h> // getpwuid
#endif
@

\subsecao{6.12. project\_name: Nome do projeto}

Só faz sendido falarmos no nome do projeto se estivermos dentro de um
projeto Weaver. Neste caso, o nome do projeto pode ser encontrado em
um dos arquivos do diretório base de tal projeto em
\monoespaco{.weaver/name}:

\iniciocodigo
@<Inicialização@>+=
if(inside_weaver_directory){
  FILE *fp;
  char *c;
#if !defined(_WIN32)
  char *filename = concatenate(project_path, ".weaver/name", "");
#else
  char *filename = concatenate(project_path, ".weaver\name", "");
#endif
  if(filename == NULL) ERROR();
  project_name = (char *) malloc(256);
  if(project_name == NULL){
    free(filename);
    ERROR();
  }
  fp = fopen(filename, "r");
  if(fp == NULL){
    free(filename);
    ERROR();
  }
  c = fgets(project_name, 256, fp);
  fclose(fp);
  free(filename);
  if(c == NULL) ERROR();
  project_name[strlen(project_name)-1] = '\0';
  project_name = realloc(project_name, strlen(project_name) + 1);
  if(project_name == NULL) ERROR();
}
@
\fimcodigo

Depois, precisaremos desalocar a memória ocupada por |project_name|:

\iniciocodigo
@<Finalização@>+=
if(project_name != NULL) free(project_name);
@
\fimcodigo

\subsecao{6.13. year: Ano atual}

O ano atual é trivial de descobrir usando a função |localtime|,
independente do sistema operacional:

\iniciocodigo
@<Inicialização@>+=
{
  time_t current_time;
  struct tm *date;
  time(&current_time);
  date = localtime(&current_time);
  year = date -> tm_year + 1900;
}
@
\fimcodigo

O único pré-requisito é incluirmos antes a biblioteca com funções de
tempo:

\iniciocodigo
@<Cabeçalhos Incluídos no Programa Weaver@>=
#include <time.h> // localtime, time
@
\fimcodigo

\secao{7. Casos de Uso}

\subsecao{7.1. Imprimir ajuda de criação de projeto}

O primeiro caso de uso sempre ocorre quando Weaver é invocado fora de
um diretório de projeto e a invocação é sem argumentos ou com
argumento \monoespaco{--help}. Nesse caso assumimos que o usuário não sabe
bem como usar o programa e imprimimos uma mensagem de ajuda. A mensagem
de ajuda terá uma forma semelhante a esta:

\alinhaverbatim
    .  .   You are outside a Weaver Directory.
   ./  \\.  The following command uses are available:
   \\\\  //
   \\\\()//  weaver
   .={}=.      Print this message and exits.
  / /`'\\ \\
  ` \\  / '  weaver PROJECT_NAME
     `'        Creates a new Weaver Directory with a new
               project.
\alinhanormal

O que é feito com o código abaixo:


\iniciocodigo
@<Caso de uso 1: Imprimir ajuda (criar projeto)@>=
if(!inside_weaver_directory && (!have_arg || !strcmp(argument, "--help"))){
  printf("    .  .     You are outside a Weaver Directory.\n"
  "   .|  |.    The following command uses are available:\n"
  "   ||  ||\n"
  "   \\\\()//  weaver\n"
  "   .={}=.      Print this message and exits.\n"
  "  / /`'\\ \\\n"
  "  ` \\  / '  weaver PROJECT_NAME\n"
  "     `'        Creates a new Weaver Directory with a new\n"
  "                project.\n");
  END();
}
@
\fimcodigo


\subsecao{7.2. Imprimir ajuda de gerenciamento}

O segundo caso de uso também é bastante simples. Ele é invocado quando
já estamos dentro de um projeto Weaver e invocamos Weaver sem
argumentos ou com um \monoespaco{--help}. Assumimos neste caso que o
usuário quer instruções sobre a criação de um novo módulo. A mensagem
que imprimiremos é semelhante à esta:

\alinhaverbatim
       \\              You are inside a Weaver Directory.
        \\______/      The following command uses are available:
        /\\____/\\
       / /\\__/\\ \\       weaver
    __/_/_/\\/\\_\\_\\___     Prints this message and exits.
      \\ \\ \\/\\/ / /
       \\ \\/__\\/ /       weaver NAME
        \\/____\\/          Creates NAME.c and NAME.h, updating
        /      \\          the Makefile and headers
       /
                          weaver --loop NAME
                           Creates a new main loop in a new file src/NAME.c

                          weaver --plugin NAME
                           Creates new plugin in plugin/NAME.c

                          weaver --shader NAME
                           Creates a new shader directory in shaders/
\alinhanormal

O que é obtido com o código:

\iniciocodigo
@<Caso de uso 2: Imprimir ajuda de gerenciamento@>=
if(inside_weaver_directory && (!have_arg || !strcmp(argument, "--help"))){
  printf("       \\                You are inside a Weaver Directory.\n"
  "        \\______/        The following command uses are available:\n"
  "        /\\____/\\\n"
  "       / /\\__/\\ \\       weaver\n"
  "    __/_/_/\\/\\_\\_\\___     Prints this message and exits.\n"
  "      \\ \\ \\/\\/ / /\n"
  "       \\ \\/__\\/ /       weaver NAME\n"
  "        \\/____\\/          Creates NAME.c and NAME.h, updating\n"
  "        /      \\          the Makefile and headers\n"
  "       /\n"
  "                        weaver --loop NAME\n"
  "                         Creates a new main loop in a new file src/NAME.c\n\n"
  "                        weaver --plugin NAME\n"
  "                         Creates a new plugin in plugin/NAME.c\n\n"
  "                        weaver --shader NAME\n"
  "                         Creates a new shader directory in shaders/\n");
  END();
}
@
\fimcodigo

\subsecao{7.3. Mostrar a versão instalada de Weaver}

Um caso de uso ainda mais simples. Ocorrerá toda vez que o usuário
invocar Weaver com o argumento \monoespaco{--version}:

\iniciocodigo
@<Caso de uso 3: Mostrar versão@>=
if(have_arg && !strcmp(argument, "--version")){
  printf("Weaver\t%s\n", VERSION);
  END();
}
@
\fimcodigo

\subsecao{7.4. Atualizar projetos Weaver já existentes}

Este caso de uso ocorre quando o usuário passar como argumento para
Weaver um caminho absoluto ou relativo para um diretório Weaver
existente. Assumimos então que ele deseja atualizar o projeto passado
como argumento. Talvez o projeto tenha sido feito com uma versão muito
antiga do motor e ele deseja que ele passe a usar uma versão mais
nova da API.

Naturalmente, isso só será feito caso a versão de Weaver instalada
seja superior à versão do projeto ou se a versão de Weaver instalada
for uma versão instável para testes. Entende-se neste caso que o
usuário deseja testar a versão experimental de Weaver no projeto. Fora
isso, não é possível fazer \italico{downgrades} de projetos, passando
da versão 0.2 para 0.1, por exemplo.

Versões experimentais sempre são identificadas como tendo um nome
formado somente por caracteres alfabéticos. Versões estáveis serão
sempre formadas por um ou mais dígitos, um ponto e um ou mais dígitos
(o número de versão maior e menor). Como o número de versão é
interpretado com um |atoi|, isso significa que se estamos usando uma
versão experimental, então o número de versão maior e menor serão
sempre identificados como zero.

Projetos em versões experimentais de Weaver sempre serão atualizados,
independente da versão ser mais antiga ou mais nova.

Uma atualização consiste em copiar todos os arquivos que estão no
diretório de arquivos compartilhados Weaver dentro de
\monoespaco{project/src/weaver} para o diretório \monoespaco{src/weaver}
do projeto em questão. Para isso podemos contar com as funções de
cópia de arquivos definidos na seção de funções auxiliares.

\iniciocodigo
@<Caso de uso 4: Atualizar projeto Weaver@>=
if(arg_is_path){
  if((weaver_version_major == 0 && weaver_version_minor == 0) ||
     (weaver_version_major > project_version_major) ||
     (weaver_version_major == project_version_major &&
      weaver_version_minor > project_version_minor)){
    char *buffer, *buffer2;
    // |buffer| <- SHARED_DIR/project/src/weaver
    buffer = concatenate(shared_dir, "project/src/weaver/", "");
    if(buffer == NULL) ERROR();
    // |buffer2| <- PROJECT_DIR/src/weaver/
    buffer2 = concatenate(argument, "/src/weaver/", "");
    if(buffer2 == NULL){
      free(buffer);
      ERROR();
    }
    if(copy_files(buffer, buffer2) == 0){
      free(buffer);
      free(buffer2);
      ERROR();
    }
    free(buffer);
    free(buffer2);
  }
  END();
}
@
\fimcodigo

\subsecao{7.5. Adicionando um módulo ao projeto Weaver}

Se estamos dentro de um diretório de projeto Weaver, e o programa
recebeu um argumento, então estamos inserindo um novo módulo no nosso
jogo. Se o argumento é um nome válido, podemos fazer isso. Caso
contrário,devemos imprimir uma mensagem de erro e sair.

Criar um módulo basicamente envolve:


a) Criar arquivos \monoespaco{.c} e \monoespaco{.h} base, deixando seus
nomes iguais ao nome do módulo criado.

b) Adicionar em ambos um código com copyright e licenciamento com o
nome do autor, do projeto e ano.

c) Adicionar no \monoespaco{.h} código de macro simples para evitar que
o cabeçalho seja inserido mais de uma vez e fazer com que o
\monoespaco{.c} inclua o \monoespaco{.h} dentro de si.

d) Fazer com que o \monoespaco{.h} gerado seja inserido
em \monoespaco{src/includes.h} e assim suas estruturas sejam
acessíveis de todos os outros módulos do jogo.

O código para isso é:

\iniciocodigo
@<Caso de uso 5: Criar novo módulo@>=
if(inside_weaver_directory && have_arg &&
   strcmp(argument, "--plugin") && strcmp(argument, "--shader") &&
   strcmp(argument, "--loop")){
  if(arg_is_valid_module){
    char *filename;
    FILE *fp;
    // Criando modulo.c
    filename = concatenate(project_path, "src/", argument, ".c", "");
    if(filename == NULL) ERROR();
    fp = fopen(filename, "w");
    if(fp == NULL){
      free(filename);
      ERROR();
    }
    write_copyright(fp, author_name, project_name, year);
    fprintf(fp, "#include \"%s.h\"", argument);
    fclose(fp);
    filename[strlen(filename)-1] = 'h'; // Criando modulo.h
    fp = fopen(filename, "w");
    if(fp == NULL){
      free(filename);
      ERROR();
    }
    write_copyright(fp, author_name, project_name, year);
    fprintf(fp, "#ifndef _%s_h_\n", argument);
    fprintf(fp, "#define _%s_h_\n\n#include \"weaver/weaver.h\"\n",
            argument);
    fprintf(fp, "#include \"includes.h\"\n\n#endif");
    fclose(fp);
    free(filename);

    // Atualizando src/includes.h para inserir modulo.h:
    fp = fopen("src/includes.h", "a");
    fprintf(fp, "#include \"%s.h\"\n", argument);
    fclose(fp);
  }
  else{
    fprintf(stderr, "ERROR: This module name is invalid.\n");
    return_value = 1;
  }
  END();
}
@
\fimcodigo

\subsecao{7.6. Criar novo projeto}

Criar um novo projeto Weaver consiste em criar um novo diretório com o
nome do projeto, copiar para lá tudo o que está no diretório
\monoespaco{project} do diretório de arquivos compartilhados e criar um
diretório \monoespaco{.weaver} com os dados do projeto. Além disso,
criamos um \monoespaco{src/game.c} e \monoespaco{src/game.h} adicionando o
comentário de Copyright neles e copiando a estrutura básica dos
arquivos do diretório compartilhado \monoespaco{basefile.c} e
\monoespaco{basefile.h}. Também criamos um
\monoespaco{src/includes.h} que por hora estará vazio, mas será modificado
na criação de futuros módulos.

\iniciocodigo
@<Caso de uso 6: Criar novo projeto@>=
if(! inside_weaver_directory && have_arg){
  if(arg_is_valid_project){
    int err;
    char *dir_name;
    FILE *fp;
    err = create_dir(argument, NULL);
    if(err == -1) ERROR();
#if !defined(_WIN32) //cd
    err = chdir(argument);
#else
    err = _chdir(argument);
#endif
    if(err == -1) ERROR();
    err = create_dir(".weaver", "conf", "tex", "src", "src/weaver",
                     "fonts", "image", "sound", "models", "music",
                     "plugins", "src/misc", "src/misc/sqlite",
                     "compiled_plugins", "shaders", "");
    if(err == -1) ERROR();
    dir_name = concatenate(shared_dir, "project", "");
    if(dir_name == NULL) ERROR();
    if(copy_files(dir_name, ".") == 0){
      free(dir_name);
      ERROR();
    }
    free(dir_name); //Criando arquivo com número de versão:
    fp = fopen(".weaver/version", "w");
    fprintf(fp, "%s\n", VERSION);
    fclose(fp); // Criando arquivo com nome de projeto:
    fp = fopen(".weaver/name", "w");
    fprintf(fp, "%s\n", basename(argv[1]));
    fclose(fp);
    fp = fopen("src/game.c", "w");
    if(fp == NULL) ERROR();
    write_copyright(fp, author_name, argument, year);
    if(append_file(fp, shared_dir, "basefile.c") == 0) ERROR();
    fclose(fp);
    fp = fopen("src/game.h", "w");
    if(fp == NULL) ERROR();
    write_copyright(fp, author_name, argument, year);
    if(append_file(fp, shared_dir, "basefile.h") == 0) ERROR();
    fclose(fp);
    fp = fopen("src/includes.h", "w");
    write_copyright(fp, author_name, argument, year);
    fprintf(fp, "\n#include \"weaver/weaver.h\"\n");
    fprintf(fp, "\n#include \"game.h\"\n");
    fclose(fp);
  }
  else{
    fprintf(stderr, "ERROR: %s is not a valid project name.", argument);
    return_value = 1;
  }
  END();
}
@
\fimcodigo

\subsecao{7.7. Criar novo plugin}

Este aso de uso é invocado quando temos dois argumentos, o primeiro é
|"--plugin"| e o segundo é o nome de um novo plugin, o qual deve ser
um nome único, sem conflitar com qualquer outro dentro de
\monoespaco{plugins/}. Devemos estar em um diretório Weaver para fazer
isso.

\iniciocodigo
@<Caso de uso 7: Criar novo plugin@>=
if(inside_weaver_directory && have_arg && !strcmp(argument, "--plugin") &&
   arg_is_valid_plugin){
  char *buffer;
  FILE *fp;
  /* Criando o arquivo: */
  buffer = concatenate("plugins/", argument2, ".c", "");
  if(buffer == NULL) ERROR();
  fp = fopen(buffer, "w");
  if(fp == NULL) ERROR();
  write_copyright(fp, author_name, project_name, year);
  fprintf(fp, "#include \"../src/weaver/weaver.h\"\n\n");
  fprintf(fp, "void _init_plugin_%s(W_PLUGIN){\n\n}\n\n", argument2);
  fprintf(fp, "void _fini_plugin_%s(W_PLUGIN){\n\n}\n\n", argument2);
  fprintf(fp, "void _run_plugin_%s(W_PLUGIN){\n\n}\n\n", argument2);
  fprintf(fp, "void _enable_plugin_%s(W_PLUGIN){\n\n}\n\n", argument2);
  fprintf(fp, "void _disable_plugin_%s(W_PLUGIN){\n\n}\n", argument2);
  fclose(fp);
  free(buffer);
  END();
}
@
\fimcodigo

\subsecao{7.8. Criar novo shader}

Este caso de uso é similar ao anterior, mas possui algumas
diferenças. Todo shader será um novo arquivo no formato GLSL dentro do
diretório \monoespaco{shaders}. E além disso, seu nome terá sempre o
formato dado pela expressão regular \monoespaco{[0-9][0-9]*-.*}. O(s)
dígito(s) na primeira parte do nome deve ser único para cada shader de
um mesmo projeto. E os números representados por tais dígitos devem
ser sempre sequenciais, começando no 1 e incrementando-o a cada novo
shader.

Este caso de uso será invocado somente quando o nosso primeiro
argumento for |``--shader''| e o segundo for um nome qualquer. Não
precisamos realmente forçar uma restrição nos nomes dos shaders, pois
sua convenção numérica garante que cada um terá um nome único e
não-conflitante.

Para garantir isso, o código deverá contar quantos arquivos com
extensão GLSL existem no diretório dos shaders e criar um novo shader
com nome \monoespaco{DD-XX.glsl}, onde \monoespaco{DD} é o número de
arquivos que existia mais 1 e \monoespaco{XX} é o nome escolhido
passado como segundo argumento para o programa. Mas se existirem
lacunas na numeração de shaders, por exemplo existir um shader 1 e um
3 sem existir o 2, daremos preferência para cobrir a lacuna. O
conteúdo base de um shader será obtido de um arquivo onde o programa
Weaver está instalado.

Depois de descobrir a numeração do novo shader, basta criarmos ele
como um arquivo vazio e depois copiarmos o conteúdo de um modelo já
existente em nosso diretório de instalação.

O código deste caso de uso é então:

\iniciocodigo
@<Caso de uso 8: Criar novo shader@>=
if(inside_weaver_directory && have_arg && !strcmp(argument, "--shader") &&
   argument2 != NULL){
    FILE *fp;
    size_t tmp_size, number = 0;
    int shader_number;
    char *buffer;
    @<Shader: Conta número de arquivos e obtém número do shader@>
    // Criando o shader:
    tmp_size = number / 10 + 7 + strlen(argument2);
    buffer = (char *) malloc(tmp_size);
    if(buffer == NULL) ERROR();
    buffer[0] = '\0';
    snprintf(buffer, tmp_size, "%d-%s.glsl", (int) number, argument2);
    fp = fopen(buffer, "w");
    if(fp == NULL){
        free(buffer);
        ERROR();
    }
    if(append_file(fp, shared_dir, "shader.glsl") == 0) ERROR();
    fclose(fp);
    free(buffer);
    END();
}
@
\fimcodigo

A parte de contar onter o número do novo shader ocorre de maneira
diferente no Unix e no Windows devido à API diferente para lidar com o
sistema de arquivos. Tirando as particularidades de como iterar sobre
arquivos em um diretório, o que faremos é iterar em cada arquivo no
diretório \monoespaco{shaders} de nosso projeto que não seja um
diretório, tenha extensão GLSL e tenha seu nome começado com um número
positivo. Chamaremos tal número de \monoespaco{number}. Teremos um
vetor booleano inicialmente marcado inteiramente como falso. Ao chegar
em cada um destes arquivos, marcamos no vetor booleano a informação de
que o shader de número \monoespaco{number} existe colocando o valor
verdadeiro na posição do vetor reservada para ele. Depois de iterarmos
sobre cada um dos arquivos, acharemos a primeira posição do vetor que
ainda está marcada como falsa. Sua posição indica qual número de shadr
ainda não foi usado e é o número que escolheremos.

Complexidades adicionais neste código envolvem apenas tomarmos o
cuidado para que o nosso vetor booleano sempre tenha um tamanho
adequado. Para isso tentamos alocar ele inicialmente com 128 espaços,
mas se acharmos shaders com números altos o bastante, o realocaremos
para lidar com o número maior.

O código para isso no Linux será:

\iniciocodigo
@<Shader: Conta número de arquivos e obtém número do shader@>=
#if !defined(_WIN32)
{
  size_t i, max_number = 0;
  DIR *shader_dir;
  struct dirent *dp;
  char *p;
  bool *exists;
  size_t exists_size = 128;
  shader_dir = opendir("shaders/");
  if(shader_dir == NULL)
    ERROR();
  exists = (bool *) malloc(sizeof(bool) * exists_size);
  if(exists == NULL){
    closedir(shader_dir);
    ERROR();
  }
  for(i = 0; i < exists_size; i ++)
    exists[i] = false;
  while((dp = readdir(shader_dir)) != NULL){
    if(dp -> d_name == NULL) continue;
    if(dp -> d_name[0] == '.') continue;
    if(dp -> d_name[0] == '\0') continue;
    buffer = concatenate("shaders/", dp -> d_name, "");
    if(buffer == NULL) ERROR();
    if(directory_exist(buffer) != EXISTE_E_EH_ARQUIVO){
      free(buffer);
      continue;
    }
    for(p = buffer; *p != '\0'; p ++);
    p -= 5;
    if(strcmp(p, ".glsl") && strcmp(p, ".GLSL")){
      free(buffer);
      continue;
    }
    number = atoi(buffer);
    if(number == 0){
      free(buffer);
      continue;
    }
    if(number > max_number)
      max_number = number;
    if(number > exists_size){
      if(number > exists_size * 2)
        exists_size = number;
      else
        exists_size *= 2;
      exists = (bool *) realloc(exists, exists_size * sizeof(bool));
      if(exists == NULL){
        free(buffer);
        closedir(shader_dir);
        ERROR();
      }
      for(i = exists_size / 2; i < exists_size; i ++)
        exists[i] = false;
    }
    exists[number - 1] = true;
    free(buffer);
  }
  closedir(shader_dir);
  for(i = 0; i <= max_number; i ++)
    if(exists[i] == false){
      shader_number = i + 1;
      break;
    }
  free(exists);
}
#endif
@
\fimcodigo

No Windows, o código para iterar sobre arquivos é diferente, mas o
restante não muda:

\iniciocodigo
@<Shader: Conta número de arquivos e obtém número do shader@>=
#if defined(_WIN32)
{
  int i;
  char *p;
  bool *exists;
  size_t exists_size = 128;
  int number, max_number = 0;
  WIN32_FIND_DATA file;
  HANDLE shader_dir = NULL;
  number_of_files = 0;
  shader_dir = FindFirstFile("shaders\\", &file));
  if(shader_dir == INVALID_HANDLE_VALUE)
    ERROR();
  exists = (bool *) malloc(sizeof(bool) * exists_size);
  if(exists == NULL){
    closedir(shader_dir);
    ERROR();
  }
  for(i = 0; i < exists_size; i ++)
    exists[i] = false;
  do{
    if(file.cFileName == NULL) continue;
    if(file.cFileName[0] == '.') continue;
    if(file.cFileName[0] == '\0') continue;
    buffer = concatenate("shaders\\", file.cFileName, "");
    if(buffer == NULL) ERROR();
    if(directory_exist(buffer) != EXISTE_E_EH_ARQUIVO){
      free(buffer);
      continue;
    }
    for(p = buffer; *p != '\0'; p ++);
    p -= 5;
    if(strcmp(p, ".glsl") && strcmp(p, ".GLSL")){
      free(buffer);
      continue;
    }
    number = atoi(buffer);
    if(number == 0){
      free(buffer);
      continue;
    }
    if(number > max_number)
      max_number = number;
    if(number > exists_size){
      if(number > exists_size * 2)
        exists_size = number;
      else
        exists_size *= 2;
      exists = (bool *) realloc(exists, exists_size * sizeof(bool));
      if(exists == NULL){
        free(buffer);
        closedir(shader_dir);
        ERROR();
      }
      for(i = exists_size / 2; i < exists_size; i ++)
        exists[i] = false;
    }
    exists[number - 1] = true;
    free(buffer);
  }while(FindNextFile(shader_dir, &file) != 0);
  FindClose(shader_dir);
  for(i = 0; i <= max_number; i ++)
  if(exists[i] == false){
    shader_number = i + 1;
    break;
  }
  free(exists);
}
#endif
@
\fimcodigo

\subsecao{7.9. Criar novo loop principal}

Este caso de uso ocorre quando o segundo argumento é
\monoespaco{--loop} e quando o próximo argumento for um nome válido
para uma função. Se não for, imprimimos uma mensagem de erro para
avisar.

Neste caso não podemos apenas copiar o conteúdo de um arquivo base
para formar o arquivo com um novo módulo do projeto Weaver, pois esse
novo arquivo terá definida uma função com um nome fornecido pelo
usuário. Então apenas criamos e preenchemos o arquivo na hora com
conteúdo defiido no próprio código abaixo.

\iniciocodigo
@<Caso de uso 9: Criar novo loop principal@>=
if(inside_weaver_directory && !strcmp(argument, "--loop")){
  if(!arg_is_valid_function){
    if(argument2 == NULL)
      fprintf(stderr,
              "ERROR: You should pass a name for your new loop.\n");
    else
      fprintf(stderr, "ERROR: %s not a valid loop name.\n", argument2);
    ERROR();
  }
  char *filename;
  FILE *fp;
  // Criando LOOP_NAME.c
  filename = concatenate(project_path, "src/", argument2, ".c", "");
  if(filename == NULL) ERROR();
  fp = fopen(filename, "w");
  if(fp == NULL){
    free(filename);
    ERROR();
  }
  write_copyright(fp, author_name, project_name, year);
  fprintf(fp, "#include \"%s.h\"\n\n", argument2);
  fprintf(fp, "MAIN_LOOP %s(void){\n", argument2);
  fprintf(fp, " LOOP_INIT:\n\n");
  fprintf(fp, " LOOP_BODY:\n");
  fprintf(fp, "  if(W.keyboard[W_ANY])\n");
  fprintf(fp, "    Wexit_loop();\n");
  fprintf(fp, " LOOP_END:\n");
  fprintf(fp, "  return;\n");
  fprintf(fp, "}\n");
  fclose(fp);
  // Criando LOOP_NAME.h
  filename[strlen(filename)-1] = 'h';
  fp = fopen(filename, "w");
  if(fp == NULL){
    free(filename);
    ERROR();
  }
  write_copyright(fp, author_name, project_name, year);
  fprintf(fp, "#ifndef _%s_h_\n", argument2);
  fprintf(fp, "#define _%s_h_\n#include \"weaver/weaver.h\"\n\n", argument2);
  fprintf(fp, "#include \"includes.h\"\n\n");
  fprintf(fp, "MAIN_LOOP %s(void);\n\n", argument2);
  fprintf(fp, "#endif\n");
  fclose(fp);
  free(filename);
  // Atualizando src/includes.h
  fp = fopen("src/includes.h", "a");
  fprintf(fp, "#include \"%s.h\"\n", argument2);
  fclose(fp);  
}
@
\fimcodigo

\secao{8. Conclusão}

Isso finaliza todo o código necessário para que o programa Weaver
possa gerenciar projetos de jogos feitos com o motor Weaver.

O programa apresentado aqui ainda não representa todo o gerenciamento
de um projeto. Uma parte não retratada aqui é um instalador que em
sistemas Unix é representado por um \monoespaco{Makefile} responsável
por instalar o motor Weaver no local adequado e em sistemas Windows
tem a forma de um pacote MSIX.

Além disso, o código das bibliotecas em si também fazem parte do motor
Weaver, mas terão o seu código descrito em outros artigos.

Alguns códigos como o código-base para shaders e novos projetos podem
ser encontrados junto com o código-fonte de Weaver, no
diretório \monoespaco{project}.

Por fim, em sistemas Unix há um Makefile para cada projeto, que também
realiza muito do desenvolvimento. No Windows, se utiliza-se o Visual
Studio ao invés de ferramentas Unix, esta parte do gerenciamento será
feita por um conjunto de regras.

\fim
