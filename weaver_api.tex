\font\sixteen=cmbx16
\font\twelve=cmr12
\font\fonteautor=cmbx12
\font\fonteemail=cmtt10
\font\twelvenegit=cmbxti12
\font\twelvebold=cmbx12
\font\trezebold=cmbx13
\font\twelveit=cmsl12
\font\monodoze=cmtt12
\font\it=cmti12
\voffset=0,959994cm % 3,5cm de margem superior e 2,5cm inferior
\parskip=6pt

\def\titulo#1{{\noindent\sixteen\hbox to\hsize{\hfill#1\hfill}}}
\def\autor#1{{\noindent\fonteautor\hbox to\hsize{\hfill#1\hfill}}}
\def\email#1{{\noindent\fonteemail\hbox to\hsize{\hfill#1\hfill}}}
\def\negrito#1{{\twelvebold#1}}
\def\italico#1{{\twelveit#1}}
\def\monoespaco#1{{\monodoze#1}}
\def\iniciocodigo{\lineskip=0pt\parskip=0pt}
\def\fimcodigo{\twelve\parskip=0pt plus 1pt\lineskip=1pt}

\long\def\abstract#1{\parshape 10 0.8cm 13.4cm 0.8cm 13.4cm
0.8cm 13.4cm 0.8cm 13.4cm 0.8cm 13.4cm 0.8cm 13.4cm 0.8cm 13.4cm
0.8cm 13.4cm 0.8cm 13.4cm 0.8cm 13.4cm
\noindent{{\twelvenegit Abstract: }\twelveit #1}}

\def\resumo#1{\parshape  10 0.8cm 13.4cm 0.8cm 13.4cm
0.8cm 13.4cm 0.8cm 13.4cm 0.8cm 13.4cm 0.8cm 13.4cm 0.8cm 13.4cm
0.8cm 13.4cm 0.8cm 13.4cm 0.8cm 13.4cm
\noindent{{\twelvenegit Resumo: }\twelveit #1}}

\def\secao#1{\vskip12pt\noindent{\trezebold#1}\parshape 1 0cm 15cm}
\def\subsecao#1{\vskip12pt\noindent{\twelvebold#1}}
\def\referencia#1{\vskip6pt\parshape 5 0cm 15cm 0.5cm 14.5cm 0.5cm 14.5cm
0.5cm 14.5cm 0.5cm 14.5cm {\twelve\noindent#1}}

%@* .

\twelve
\vskip12pt
\titulo{O Programa Weaver}
\vskip12pt
\autor{Thiago Leucz Astrizi}
\vskip6pt
\email{thiago@@bitbitbit.com.br}
\vskip6pt

\abstract{This article describes using literary programming the
  Weaver API. Weaver is a game engine and this API are how programmers
  interact with the engine in their game projects. Besides the API,
  this article also covers how the configuration file is interpreted
  and how game loops should be managed in a game project. The API is
  portable code which should work under OpenBSD, Linux, Windows and
  Web Assembly environments.}

\vskip 0.5cm plus 3pt minus 3pt

\resumo{Este artigo utiliza programação literária para descrever a
API Weaver. weaver é um motor de jogos e esta API é como os
programadores interagem com ela em seus projetos. Além da API, este
artigo também cobre como o arquivo de configuração é interpretado e
como os laços de execução do jogo devem ser organizados. A API é
código portável que deve funcionar sob ambientes OpenBSD, Linux,
Windows e Web Assembly.}

\secao{1. Introdução}

Quando um usuário digita \monoespaco{weaver PROJETO} na linha de
comando, um diretório com um novo projeto Weaver e criado. Dentro
deste diretório, o arquivo que contém o loop principal está
em \monoespaco{src/game.c} e nele encontramos:

\alinhaverbatim
#include "game.h"

MAIN_LOOP(main_loop){ // The game loop
 LOOP_INIT: // Code executed during loop initialization

 LOOP_BODY: // Code executed every loop iteration
    if(W.keyboard[W_ANY])
        Wexit_loop();
 LOOP_END: // Code executed at the end of the loop
    return;
}

int main(void){
  Winit(); // Initializes Weaver
  Wloop(main_loop); // Enter a new game loop
  return 0;
}
\alinhanormal

E dentro de \monoespaco{src/game.h}, nós encontramos:

\alinhaverbatim
#ifndef _game_h_
#define _game_h_

#include "weaver/weaver.h"
#include "includes.h"

struct _game_struct{
  // You can personalize this struct putting your variables here. But
  // don't change it's name. Access it in W.game variable.
  int whatever; // <- A variablejust  to prevent errors in some compilers
} _game;

MAIN_LOOP main_loop(void);

#endif
\alinhanormal

Notar que neste arquivo existe uma estrutura que pode ser
personalizada pelo usuário e cumpre o papel de centralizar algumas
variáveis com estados globais. O tipo de coisa que para um jogo deve
ser preservada ao salvar. Ou que deve ser acessível para plugins. Ou
que deve ser transmitida para informar o estado do jogo a clientes
conectados à rede. Segundo o comentário acima, esta estrutura deve ser
referenciada pela variável \monoespaco{W.game}, o que já nos indica
que a API será organizada de modo a fornecer um \monoespaco{struct}
chamado \monoespaco{W} onde serão centralizados os recursos da API.

O arquivo \monoespaco{includes.h} é apenas um cabeçalho que inclui em
um projeto todos os cabeçalho de módulos criados pelo usuário (cada
módulo é um arquivo de código C e um cabeçalho).

Todo o código da API deve então estar presente ou ser incluída por
macros dentro dos arquivos \monoespaco{weaver.c}
e \monoespaco{weaver.h}. A organização do \monoespaco{weaver.h} é:

\iniciocodigo
@(project/src/weaver/weaver.h@>=
#ifndef _weaver_h_
#define _weaver_h_
#ifdef __cplusplus
  extern "C" {
#endif
@<Estrutura Global@>
@<Cabeçalhos Weaver@>
@<Macros Weaver@>
#ifdef __cplusplus
  }
#endif
#endif
@
\fimcodigo

A tal ``estrutura global'' referenciada acima nada mais é que
o \monoespaco{struct} chamado \monoespaco{W} onde já mencionamos no
comentário que colocaremos o \monoespaco{struct \_game\_struct \_game}
que definimos em \monoespaco{game.h}. Podemos então começar definindo
ele:

\iniciocodigo
@<Estrutura Global@>=
// Esta estrutura conterá todas as variáveis e funções definidas pela
// API Weaver:
extern struct _weaver_struct{
  struct _game_struct *game;
  @<Variáveis Weaver@>
  //@<Funções Weaver@>
} W;
@
\fimcodigo

Notar que além de \monoespaco{W.game}, existirão outras variáveis
presentes dentro desta estrutura. Basicamente iremos centralizar
dentro dela todas as funções públicas da nossa API. Só não estarão
nela funções que começam com ``\_'', e que são consideradas internas e
não deveriam ser usadas por programadores utilizando a API. Desta
forma deixamos bem delimitado o que faz parte da API e também evitamos
poluir com nomes o ``namespace'' global de programas em C.

Também definiremos aqui a estrutura geral de nosso
arquivo \monoespaco{weaver.c}:

\iniciocodigo
@(project/src/weaver/weaver.c@>=
#include "weaver.h"
#include "../game.h"
@<API Weaver: Definições@>
//@<API Weaver: Funções@>
@<API Weaver: Base@>
@
\fimcodigo

Nas definições declaramos novos tipos de estruturas de dados que forem
necessárias. A primeira coisa que já podemos definir é a
estrutura \monoespaco{W}, a qual é apenas declarada no cabeçalho:

\iniciocodigo
@<API Weaver: Definições@>=
struct _weaver_struct W;
@
\fimcodigo

Na parte de funções definimos as funções a serem usadas. Já a última
partedo arquivo contém as funções mais básicas da API. As únicas que
não são colocadas dentro da estrutura \monoespaco{W}. Elas são a
função de inicialização e a função que encerra o funcionamento do
motor.

Uma das coisas que a função de inicialização faz é inicializar os
valores da estrutura \monoespaco{W}:

\iniciocodigo
@<Cabeçalhos Weaver@>=
void Winit(void);
@
\fimcodigo

\iniciocodigo
@<API Weaver: Base@>=
void Winit(void){
  W.game = &_game;
  @<API Weaver: Inicialização@>
}
@
\fimcodigo

A função de finalização deve desalocar qualquer memória pendente,
finalizar o uso de recursos, e deve também fechar o programa
informando que tudo correu bem se assim realmente ocorreu:

\iniciocodigo
@<Cabeçalhos Weaver@>=
void Wexit(void);
@
\fimcodigo

\iniciocodigo
@<API Weaver: Base@>=
void Wexit(void){
  //@<API Weaver: Finalização@>
  exit(0);
}
@
\fimcodigo

O uso da função \monoespaco{exit} nos obriga a inserir o cabeçalho:

\iniciocodigo
@<Cabeçalhos Weaver@>=
#include <stdlib.h>
@
\fimcodigo

\subsecao{1.1. Os loops principais.}

Todos os jogos são organizados dentro de loops, ou laços
principais. Eles são basicamente um código que fica iterando
indefinidamente até que uma condição nos leve a outro loop principal,
ou então ao fim do programa.

Como foi mostrado no código inicial do \monoespaco{game.c}, um loop
principal deve ser declarado como:

\inicioverbatim
MAIN\_LOOP(nome\_do\_loop){
  // Código do loop
}
\fimverbatim

E podemos entrar em um novo loop principal, ou iniciar o primeiro loop
principal por meio de uma chamada \monoespaco{Wloop(nome\_do\_loop)}.

Com relação à função \monoespaco{Wloop}, a primeira coisa com a qual
devemos nos preocupar é se não existe algum arquivo ou recurso
pendente que deve ser obtido antes de iniciarmos um novo loop. Se
existir, não poderemos passar para o novo loop. Por causa disso, a
função \monoespaco{Wloop} na verdade é uma macro que verifica isso
antes de chamar a verdadeira função de loop:

\iniciocodigo
@<Macros Weaver@>+=
#define Wloop(a) ((W.pending_files)?(false):(_Wloop(a)))
@
\fimcodigo

A estrutura |W| deve então possuir esta variável que armazena se temos
arquivos pendentes ou não:

\iniciocodigo
@<Variáveis Weaver@>=
  // Isso vai dentro da estrutura W:
  unsigned int pending_files;
@
\fimcodigo

E ela deve ser inicializada como 0, incrementar se começamos a obter
um recurso relevante assíncrono e decrementar quando terminamos.

\iniciocodigo
@<API Weaver: Inicialização@>=
W.pending_files = 0;
@
\fimcodigo

Com relação à verdadeira função de loop, ela é declarada da seguinte
forma:

\iniciocodigo
@<Cabeçalhos Weaver@>+=
#if !defined(_MSC_VER)
void _Wloop(void (*f)(void)) __attribute__ ((noreturn));
#else
__declspec(noreturn) void _Wloop(void (*f)(void));
#endif
@
\fimcodigo

Notar que a função de loop nunca retornará e a macro condicional serve
para especificar isso tanto em uma notação que o GCC e Clang entendem
como na notação do Visual Studio. Qualquer código presente
imediatamente após ela será ignorado, a menos que falhemos ao iniciar
um novo loop devido a algum recurso assíncrono pendente.



%%%%%%%%%%%%%%%%%%%%%%%%%%%%%%%%%%%% TODO %%%%%%%%%%%%%%%%%%%%%%%%%%%%%%



@<Funções Weaver@>
@

@<API Weaver: Finalização@>
@

@<API Weaver: Funções@>
@

@<API Weaver: Inicialização@>
@


\fim
